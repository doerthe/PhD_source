By now, we have discussed three of the four characteristics we identified as crucial for a \emph{Unifying Logic} for the Semantic Web (Section~\ref{req}): 
A clearly defined semantics, 
syntactic and semantic compatibility with \rdf, and the power to connect the logical building blocks \emph{Querying}, \emph{Ontologies/Taxonomies} and \emph{Rules}. 
The fourth aspect in this list, the support of the layer \emph{Proof}, is the topic of the following two chapters. It should be possible to express and exchange proofs
 by making use of the \emph{Unifying Logic}. % of the Semantic Web should support proofs:
% It needs to be possible 
% to express proofs in that logic in 
These proofs should not only support the layer of \emph{Trust} but should also be suitable to be used in practical applications.  

In the Semantic Web there are different ways how reasoners provide explanations for their conclusions:
OWL DL reasoners mostly either provide a \emph{justification}~\cite{just1,just2} -- a set of facts from which the new knowledge follows -- 
or a \emph{proof}~\cite{owlproof2,owlproof} -- 
a list of the aforementioned facts together with the concrete reasoning steps applied on them leading to the new data. 
While verifications leave the work of constructing a more concrete description of how inference steps have been applied to the user who needs to know the underlying calculus to make sense of it,
DL-proofs have another disadvantage: 
they are mostly represented in formats 
which are not native to the Semantic Web and can therefore not easily be exchanged, reused or automatically interpreted in this context.

For \nthree this is different: \nthree reasoners produce proofs using the SWAP~\cite{SWAP} proof 
vocabulary\footnote{\url{http://www.w3.org/2000/10/swap/reason\#}}. This vocabulary enables them to explain all their derivations in \nthree. 
By that the proofs themselves become again part of the Semantic Web, they can be exchanged between different parties. They can be checked or
used further by other reasoners.
%We discuss the relationship of \nthreelogic and the layer of \emph{Proofs} in two steps: Firstly, 
% In this chapter, 
% we further elaborate in this chapter on the SWAP~\cite{SWAP} proof vocabulary\footnote{\url{http://www.w3.org/2000/10/swap/reason\#}} 
% which we already briefly discussed in Section~\ref{proofintro}. 

In this chapter we focus on the formal aspects of proofs:
We identify a set of formulas for which the semantics of \nthree is not ambiguous, \emph{simple formulas}.
For these \emph{simple formulas} we define the direct semantics. This definition also covers the semantics according to EYE.
We then formally define the calculus the SWAP proof vocabulary expresses and prove its correctness. 
In the following chapter we then investigate the practical applications of \nthree proofs.
% Secondly, in the following chapter we will then take a closer look to the practical applications of proofs: next to establishing 
% trust, proofs can, for example, also be used to do planning or to identify the potentially critical streams in a bigger set-up of streaming sensors. 
%Below, we explain these two use cases in more detail.
% vocabulary
% introduced in the context of the Semantic Web Application Platform (SWAP)~\cite{SWAP}

% In the previous chapters we discussed how the semantics of \nthreelogic can be defined, its relation to \rdf and its ability to connect the logical building blocks
% of the Semantic Web Layer cake. 
% Having dons that, we covered three of the four requirements we had on a unifying logic 
% As discussed in Section~\ref{req}, all these aspects are crucial for a logic to become the unifying logic of the Semantic Web.
% we identified all these aspects as crucial properties for a \emph{Unifying logic} for the Semantic Web but there 
% was one more relevant aspect for such a logic: it should support the \emph{Proof} layer. 
% \section{Proofs and explanations in the Semantic Web}\todo{this part need to either be written or removed}
% Before we discuss proofs in \nthree we first give a brief overview of the different approaches to explain the derivation results produced by the different reasoning systems
% in the Semantic Web.
% Semantic Web reasoning typically either applies Description Logics (DL)~\cite{dl} as formalised in the context of the Web Ontology language (OWL)~%~\cite{owl} 
% or one of the several rule-based approaches,
% e.g., swrl~\cite{swrl} or Notation3 Logic (N3 Logic)~\cite{N3Logic}. 

% Together with their reasoning results most OWL DL reasoners either provide a   a \emph{justification} -- a set of facts from which the new knowledge follows --
% or a proof in the classical sense -- a list of the aforementioned facts together with the concrete reasoning steps applied on them leading to the new data.
% 





\section{Simple formulas}
To be able to discuss about the possible applications of proofs in \nthree, we first need to 
clarify the calculus behind it and -- most important -- prove its correctness based on the semantics of \nthree. We already know, 
that this semantics of \nthree is 
rather problematic: In Section~\ref{semofn3} we formalised two possible alternatives to understand the meaning of nested implicitly universally quantified variables
and it is very likely, that there are even more ways to understand implicit quantification in \nthree. 
%Even though we stated our own preferences, we  rather would like this matter to be discussed in the community instead of making a concrete choice here. 
% In the context of the Semantic Web, we want proofs to be exchangeable: different parties should be able to exchange and also test each others proofs, this is 
% only possible if we agree on the semantics. Since this agreement should be based on discussions within the Semantic Web community and not simply be taken by 
% one single author of a dissertation, we first specify a set of formulas which is interpreted in the same way by 
% 
Since we want to keep our findings valid for all implementations -- proofs should 
be exchangeable among the different reasoners -- we 
%limit our considerations about the proof calculus to those 
first define a set of formulas whose implicitly universally
quantified variables are not ambiguous.  If we make sure that the formulas we consider do not contain deeply nested universals, we can guarantee that Cwm and EYE 
quantify these variables at the same position: on the top formula.
%These are the formulas in which 
% If all universal variables in a formula  %such variables 
% do not occur in deeply nested expressions,  . 
We call formulas fulfilling this condition \emph{simple formulas}. Below, we explain the idea of such \emph{simple formulas} in more detail.


%\subsection{Semantics of simple formulas}
% If we go back to the example formulas containing universals we saw so far
% In our previous examples, 
Based on our considerations in the previous chapters we know at this point that implicit universal quantification in \nthree always leads to contradicting interpretations 
if the quantified variables only occur deeply nested in a formula.
While example 
Formula~\ref{uni} from Section~\ref{vars}
\begin{multline}\notag
 \texttt{\{:Kurt :knows ?x.\} => }%} \\ \texttt{
 \texttt{\{?x :knows :Kurt.\}.}%\nonumber
\end{multline}
had a clear meaning in both interpretations we considered, namely
\[
 \forall \texttt{x}: \texttt{<Kurt knows x>}\rightarrow \texttt{<x knows Kurt>}
\]
Formula~\ref{fff} led to discrepancies:
\begin{multline}\notag
\texttt{\{\{?x :q ?y.\} => \{?x :r :c.\}.\} =>}
\texttt{\{?x :p :a.\}.}
\end{multline}
The problem with that last formula was that Cwm and EYE differ in their understanding of the concept \emph{parent}  
for universal variables which only occur nested -- ie surrounded by at least two pairs of curly brackets \texttt{\{~\}} -- in a formula.  Here, that is the case 
for the universal variable
\texttt{?y}. For Cwm, the next higher formula surrounded by curly brackets is the parent, it carries the quantifier -- that is the formula \texttt{\{?x :q ?y.\} => \{?x :r :c.\}.} in the example --
while for EYE the top formula is always the parent. \emph{Simple formulas} are now the formulas where universal variables do not occur nested 
and which are therefore interpreted 
equally by both reasoners. 

In order to formally define \emph{simple formulas}, we first introduce the concept of \emph{nested components}.
All definitions in this chapter rely on the syntax definitions we gave in Section~\ref{n3synsec}, in particular the \nthree alphabet (Definition~\ref{alphabet}) 
and the grammar given in 
Figure~\ref{N3S}. 
We will furthermore make use of two different types of brackets: brackets occurring in \nthree's syntax \texttt{\{}, \texttt{\}}, \texttt{(}, \texttt{)} and auxiliary 
brackets we use in mathematical expressions.
 To emphasize the difference between these two kinds of brackets,
%  brackets which form part of the \nthree vocabulary, i.e. ``\verb!(!'', ``\verb!)!'', ``\verb!{!'', and ``\verb!}!'', 
%  and the brackets occurring in mathematical language,
 we will underline the \nthree brackets in all definitions where both kinds of brackets occur.

% Using that notation, we define:
% The scope of this quantification depends on how deeply nested
% a variable occurs in a formula. To be able to make statements about that we define:



\begin{definition}[Components of a formula]\label{formulacomponents}
Given an \nthree alphabet $\mathcal{A}$.
Let $\mathcal{F}$ be the set of \nthree formulas and $\mathcal{T}$ 
the set of terms over $\mathcal{A}$. Let $f\in \mathcal{F}$ be a formula and $c:\mathcal{T} \rightarrow 2^\mathcal{T}$ a~function such that:
\[c(t)=\begin{cases}
  
  c(t_1)\cup\ldots\cup c(t_n) & \text{if }t=\underline{\texttt{(}}t_1 \ldots t_n\underline{\texttt{)}}\text{ is a list,}\\
  \{t\}  & \text{otherwise.}
\end{cases}\]

We define the set $\textit{comp}(f)\subset \mathcal{T}$ of \emph{direct components} of $f$ as follows:
 \begin{itemize}
  \item If $f$ is an atomic formula of the form $t_1~ t_2~ t_3.$, $\textit{comp}(f)=c(t_1)\cup c(t_2)\cup c(t_3)$.
  \item If $f$ is an implication of the form $e_{1} \verb!=>!~e_{2}.$, then $\textit{comp}(f)=\{e_1, e_2\}$.
  \item If $f$ is a conjunction of the form $f_1 f_2$, then $\textit{comp}(f)=\textit{comp}(f_1)\cup \textit{comp}(f_2)$.
 \end{itemize}
 Likewise, for $n\in \mathbb{N}_{>0}$, we define the \emph{components of level $n$} as:
 \begin{flalign*} 
  \textit{comp}^n(f):= &  
  \{t\in \mathcal{T}|\exists f_1,\ldots, f_{n-1}\in \mathcal{F}: 
   t\in \textit{comp}(f_1)\wedge  \underline{\texttt{\{}}f_1\underline{\texttt{\}}}\in \textit{comp}(f_2)\wedge \ldots\\& \wedge  \underline{\texttt{\{}}f_{n-1}\underline{\texttt{\}}}\in 
  \textit{comp}(f)\} 
\end{flalign*} 
For all $n>1$ we call the components  $t\in\textit{comp}^n(f)$ \emph{nested components} of $f$.
\end{definition}


Now, we can distinguish between direct components and 
nested components. Formula~\ref{fff} from above has  \texttt{\{\{?x :q ?y.\} => \{?x :r :c.\}.\}} and
 \texttt{\{?x :p :a.\}} as direct components, nested components of level  two are  \texttt{\{?x :q ?y.\}}, \texttt{ \{?x :r :c.\}}, \texttt{?x}, \texttt{:p}, and \texttt{:a}, 
 and nested components of level three are \texttt{?x}, \texttt{:q}, \texttt{?y}, \texttt{:r} and \texttt{:c}.
Our definition of \emph{simple formulas} now excludes all formulas having universal variables at level three or deeper:

% \texttt{\{\{?x :q ?y.\} => \{?x :r :c.\}.\} =>}
% \texttt{\{?x :p :a.\}.}

% In example Formula~\ref{ref} from Section~\ref{vars}:
% \[
%   \texttt{:John :says \{:Kurt :knows :Albert.\}.}
% \]
%  \verb!:John!, \verb!:says! and \verb!{:Kurt :knows :Albert.}! are direct components while \verb!:Kurt!,  \verb!:knows! and  \verb!:Albert! are nested components of
% level two. If we go back two the two example containing universals from above, we have 

%Using the above definition, we can now limit the level of nesting for universal variables:

\begin{definition}[Simple formulas]
Let $\mathcal{A}$ be an \nthree alphabet and let $U$ be the set of universal %and $E$ be the set of existential
variables.
We call an \nthree formula $f$ \emph{simple} iff  $\operatorname{comp}^n(f)\cap U =\emptyset$ %and  $\operatorname{comp}^n(f)\cap E =\emptyset$ 
for all $n \in \mathbb{N}$, $n>2$.
\end{definition}

For $f=\text{ Formula~\ref{fff} }$ we have:  \[\textit{comp}^3(f)=\{\texttt{?x}, \texttt{:q}, \texttt{?y}, \texttt{:r}, \texttt{:c}\}\]
and thus:
\[\textit{comp}^3(f)\cap U=\{\texttt{?x}, \texttt{:q}, \texttt{?y}, \texttt{:r}, \texttt{:c}\}\cap U = \{\texttt{?x}, \texttt{?y}\}\neq \emptyset\]
Formula~\ref{fff} is not a \emph{simple formula}. Since Formula~\ref{uni} does not contain any components of a higher level than 2, this formula is \emph{simple}.
For \emph{simple formulas} we now have:

\begin{lemma}[Universal quantification of simple formulas]
The \nthree Core Logic interpretations according to Cwm and EYE always quantify all universal variables occurring in \emph{simple formulas} on top level. 
\end{lemma}

\begin{proof}
 For the interpretation according to EYE the claim is trivial since EYE quantifies all universal variables exclusively on top level.
 
 To see that Cwm also quantifies all universal variables occurring in a \emph{simple formula} on top level we take a closer look at the 
 attributes from Section~\ref{unicwm} defined  in Table~\ref{uniatt}. Remember that for each formula node \texttt{f}
 the value $\texttt{f}.q$ of attribute $q$ is exactly the set of universal variables for which \texttt{f} carries a universal quantifier 
 in the \nthree Core Logic translation.  We use the definition of this attribute in our proof.
 
 Let us assume that $g_1$ is a simple formula containing a variable \texttt{?x} as nested component which is not quantified on top level. 
 This means, that for some $k\geq 1$ there exists a  
 component $\texttt{\{}g_2\texttt{\}}\in \textit{comp}^k(g_1)$ for which $\texttt{?x} \in g_2.q$. Based on the fact that $g_2$ is surrounded by brackets, 
 we know that the value  $g_2.q$ is computed using the attribute rule  
 $\texttt{f}.q \leftarrow \texttt{f}.v_2 \setminus \texttt{e}.s$ on the production rule \texttt{e ::=\{f\}}. We therefore know that $\texttt{?x}\in g_2.v_2$. 
 If we now look at the attribute rules for $v_2$, we furthermore see that this value can only be not empty if 
 the attribute rule $\texttt{e}.v_2\leftarrow\texttt{f}.v_1$ has been applied
 on a lower level in the syntax tree. As this rule is an attribute rule for the production rule
 \texttt{e::=\{f\}} we know that there exists a direct component $\texttt{\{}g_3\texttt{\}}\in \textit{comp}(g_2)\subset\textit{comp}^{k+1}(g_1)$ for which
 %for some $l\geq1$ there must be a $\texttt{\{}g_3\texttt{\}}\in \textit{comp}^l(g_2)$ with 
 $\texttt{?x} \in g_3.v_1$. According to the definition of $v_1$ which only passes values upwards to the next expression \texttt{e} in the syntax tree this means
 that \texttt{?x} is direct  component of $g_3$. We therefore know that $\texttt{?x}\in \textit{comp}(g_3)\subset \textit{comp}^2(g_2)\subset \textit{comp}^{k+2}(g_1)$. This means for 
 $k+2>2$ we have $\textit{comp}^{k+2}(g_1)\cap U \neq \emptyset$ which contradicts the assumption that $g_1$ is a \emph{simple formula}.
 %But since both $\texttt{\{}g_2\texttt{\}}$ and $\texttt{\{}g_3\texttt{\}}$ are nested components
% This means that \texttt{?x} is a direct or nested component
%  If we now look at the attribute rules for $v_1$, 
%  we see that this means that there must be a direct or nested component $x$ of $g_3$ which is a universal variable. 
%  But as  $\texttt{\{}g_3\texttt{\}}$ is a direct or nested component of $g_2$ there must be a $k.0$ such that $x\in \textit{comp}^{k+1}(g_2)$ 
 
%  For each sub-formula for which this value is not the empty set, this value is produced by the attribute rule
%  $\texttt{f}.q \leftarrow \texttt{f}.v_2 \setminus \texttt{e}.s$ on the production rule \texttt{e ::=\{f\}}. If $\texttt{f}.q\neq \emptyset$ then we also know that 
%  $\texttt{f}.v_2\neq \emptyset$. This means that there must be direct or nested component of 
 
 
%  To see that Cwm also quantifies all universals occurring in simple formulas on top level, we make use of the attributes from Section~\ref{unicwm} defined  in Table~\ref{uniatt}.
%  For each formula node \texttt{f} in an \nthree syntax tree the value of the inherited attribute $q$ is the set of universal variables for which \texttt{f} carries a universal quantifier 
%  in the \nthree Core Logic translation of the formula. The top formula is always produced by the production rule \texttt{s ::= f}. The corresponding attribute rule 
%  for $q$ is: $\texttt{f}.q\leftarrow \texttt{f}.v_1 \cup \texttt{f}.v_2$. 
%  
%  Let \texttt{?x} be a universal variable occurring in the simple \nthree formula $g$. Then $\texttt{?x}\in \textit{comp}(g)\cup \textit{comp}^2(g)$.
%  \begin{enumerate}
%   \item \label{ppp1} If $\texttt{?x}\in \textit{comp}(g)$, then $g$ is an atomic formula, ie $g= t_1 t_2 t_3$ and $\texttt{?x}= t_i$ for at least one $i$ with $1\leq i\leq 3$. 
%   For the production rule 
%   $\texttt{f ::= t}_1\texttt{t}_2\texttt{t}_3$ for atomic formulas the value for $v_1$ is computed by the rule  $ \texttt{f}.v_1 \leftarrow \texttt{t}_1.v_1\cup \texttt{t}_2.v_1\cup \texttt{t}_3.v_1$.
%   For $t_i=\texttt{?x}$ we furthermore know because of the attribute rule $\texttt{t}.v_1\leftarrow\{\texttt{uv}\}$ on the production rule  \texttt{t ::= uv} that 
%   $t_i.v_1=\{\texttt{?x}\}$. Thus
%   $\{\texttt{?x}\}\subset g.q$. 
%   \item If $\texttt{?x}\in \textit{comp}^2(g)$ then $g$ is either a atomic formula, an implication or a conjunction. We consider these three cases separately:
%   \begin{enumerate}
%   \item If $g$ is an atomic formula of the form $t_1 t_2 t_3$, then $\texttt{?x}\in \textit{comp}(g_2)$ for at least one $t_i$ with $t_i=\underline{\texttt{\{}}f_2\underline{\texttt{\}}}$.
%   This $g_2$ then needs to be an atomic formula and therefore be produced by the rule $\texttt{f ::= t}_1\texttt{t}_2\texttt{t}_3$. For this rule we have the attribute rules
%   \item
%   \item If $f$ is a conjunction,
%   \end{enumerate}
%  
%  \end{enumerate}
\end{proof}

% We thus showed that \emph{simple formulas} are the formulas which Cwm and EYE interpret equally. Both reasoners will come to the same conclusions 
% when provided with simple formulas. But  the goal of our restriction was to also make the proofs produced by these and other reasoners exchangeable. We therefore 
% need to also take the cases into account where reasoners translate implicitly existentially quantified variables to universals. We illustrate that problem on an example. Consider
% the following formula:
% \begin{equation}\label{simpleex}
%  \texttt{\{:a :b :c\}=>\{\{\_:x :d :e\}=>\{:f :g :h\}\}.} 
% \end{equation}
% According to Cwm's interpretation this formula means in \nthree Core Logic:
% \[
%  \texttt{<a b c>}\rightarrow \texttt{<{}<}\exists \texttt{x. x d e>}\rightarrow \texttt{<f g h>{}>.} 
% \]
% But this is equivalent to 
% \[
%  \texttt{<a b c>}\rightarrow \texttt{<} \forall \texttt{x.} \texttt{ <x d e>}\rightarrow \texttt{<f g h>{}>.} 
% \]
% which is again a formula with nested universal quantification. Following Cwm's semantics this could be expressed in \nthree as:
% \begin{equation}\label{simpleall}
%  \texttt{\{:a :b :c\}=>\{\{?x :d :e\}=>\{:f :g :h\}\}.} 
% \end{equation}
% % For all universal variables occurring in \emph{simple formulas} the following holds:
% % \[
% %  \text{parent}_c=\text{parent}_e
% % \]
% % The interpretations according to Cwm and EYE thus both quantify all universal variables occurring in \emph{simple formulas} on top level. 
% % 
% Formula~\ref{simpleall} is not a \emph{simple formula} while Formula~\ref{simpleex} is.


\section{The direct semantics of N3}
%\section{A proof calculus for N3}
\label{nthree}
% Below, we define the semantics of such formulas. %\todo{here or somewhere else: why are we not using the elaboration semantics?}
% Opposed to the previous chapters, we will use a direct approach here, mainly because this makes the later proofs concerning the proof calculus easier to follow.  
% 
Below, we define a direct semantics for \nthree. We do this to make the later proofs concerning the proof calculus easier to follow. For our direct semantics we assume that universal variables are
always quantified on top level. Using the result from the previous section we can nevertheless guarantee that this semantics is compatible with both reasoners -- EYE and Cwm -- 
if we limit the input to \emph{simple formulas}. By this limitation proofs remain interchangeable.\footnote{As we already know from the discussion in Section~\ref{pro} the proof 
vocabulary adds an extra level to the formulas, proofs are therefore most likely not \emph{simple formulas}. But as also discussed in that section, %this slight difference in the indoes not cause 
Cwm and EYE interpret the implicitly universally quantified which only appear nested in a proof because of the proof vocabulary almost equally.
}

%By limiting our applications using proofs to \emph{simple formulas} in the following chapter we can despite this choice guarantee that our applications are reasoner independent.
%In the later chapters we limit our later considerations  and to \emph{simple formulas} 

%These definitions state how terms and expressions can be constructed and how these can be composed to formulas. We had three different kinds of formulas:
% As we are going to use this terminology below, we repeat here the three different kinds of formulas we had:
% \begin{enumerate}
%  \item \emph{atomic formulas} \texttt{t t t.} consisting of three terms,
%  \item \emph{implications} \texttt{e => e.} where \texttt{e} are expressions, and
%  \item \emph{conjunctions} \texttt{f f} consisting of two formulas.
% \end{enumerate}
%
% %\todo{cite \cite{semN3}}
% 
% %\todo{Plan: stay with my formalisation (improves version), then limit to nesting level 2 and say that we have existential rules. Maybe cite a nice paper aboout existential rules.
% %What do you think Ruben?}
% \restdesc descriptions are expressed in the Notation3 (\nthree) rule language~\cite{N3Logic,Notation3}.
% We will introduce the \nthree language and its logic,
% focusing on the aspects relevant to our purposes. Our formalization is based on the formalization we gave in a previous paper \cite{semN3} and the informal
% semantic descriptions given in the above mentioned sources.
% %F\todo{cite \cite{semN3}}
% %\todo{Stay with the new formalisation of use simple FOL in the first place?}
% %%%%%%%%%%%%%%%%%%%%%%%%%%%%%%%%%%%%%%%%%%%%%%%%%%%%%%%%%%%%%%%%%%%%%%%%%%%%%%%%%%%%%%%%%%%%%%%%%%%%%%%%%%%%
% % Syntax
% %%%%%%%%%%%%%%%%%%%%%%%%%%%%%%%%%%%%%%%%%%%%%%%%%%%%%%%%%%%%%%%%%%%%%%%%%%%%%%%%%%%%%%%%%%%%%%%%%%%%%%%%%%%%
% \nthree augments the RDF model with symbols for quantification, implication, and statements about formulas:
% 
% 
% \begin{definition}[Basic N3 vocabulary]
% An \emph{N3 alphabet~$A$} consists of the following disjoint classes of symbols:
% \begin{itemize}
% \item A set $U$ of \uri symbols.
% \item A set $V=V_E\mathbin{\dot{\cup}} V_U$  of (quantified) variables, with $V_E$ being the set of existential variables
% and $V_U$ the set of universal variables.
% \item A set $L$  of literals.
% \item A boolean literal \verb!false!
% \item Brackets \verb!{!, \verb!}!, \verb!(!, \verb!)!
% \item Logical implication $\verb!=>!$ 
% \item Period \verb!.!
% \end{itemize}
% \end{definition}
%
% 
% 
% We define the elements of $U$ as in the corresponding specification~\cite{iri}.
% \nthree allows to abbreviate URLs as prefixed names~\cite{turtle}.
% Literals are strings beginning and ending with quotation marks `\verb!"!';
% existentials start with `\verb!_:!', universals with `\verb!?!'.
% 
% \nthree does not distinguish between predicates and constants---%
% a single \uri symbol can stand for both at the same time---%
% so the first-order-concept of a~\emph{term}
% has a~slightly different counterpart in \nthree: an \emph{expression}.
% Since the definition of expressions (Definition~\ref{expression})
% is closely related to the concept of a~formula (Definition~\ref{formula}),
% the two following definitions should be considered together.
% 
% \begin{definition}[Expressions]\label{expression}
%   Let $A$ be an \nthree alphabet.
%   The set of \textit{expressions} $E \subset A^{*}$ is
%   defined as follows:
%   \begin{enumerate}
%     \item Each \uri is an expression.
%     \item Each variable is an expression.
%     \item Each literal is an expression.
%     \item \label{list} If $e_1,\ldots,e_n$ are expressions, $\verb!(!e_1 \ldots e_n\verb!)!$ is an expression. 
%     \item \label{false} \verb!false! is an expression.
%     \item $\verb!{ }!$ is an expression.
%     \item \label{fe} If $f\in F$ is a formula, then $\verb!{!f\verb!}!$ is an expression. 
%   \end{enumerate}
%   The expression defined by \ref{list} is called a \textit{list}.
%   We call the expressions defined by \ref{false}--\ref{fe}
%   \textit{formula expressions} and denote the set of all formula expressions by $\mbox{\textit{FE}}$.
% \end{definition}
%
%
% Note that point \ref{fe} of the definition above makes use of formulas, which are defined as follows:
% 
% \begin{definition}[\nthree Formulas]
%     \label{formula}
%     The set~$F$ of \textit{\nthree formulas} over alphabet~$A$ is recursively defined as follows:
%     \begin{enumerate}  
%       \item \label{1} If $e_1, e_2, e_3 \in E$, then the following is a formula, called \textit{atomic} formula: \[e_1~ e_2~ e_3.\] 
%       \item \label{2} If $t_1, t_2$ are formula expressions then the following is a formula, called \mbox{\textit{implication}}: \[t_{1} \verb!=>!~t_{2}.\] 
%       \item \label{n} If $f_1$ and $f_2$ are formulas, then the following is a formula, called \textit{conjunction}: \[f_1 f_2\] 
%     %  \item Nothing else is a formula.
%     \end{enumerate}
% \end{definition}
% 
% We will refer to a~formula without any variables as a~\textit{ground formula}.
% Analogously, we call expressions without any variables \textit{ground expressions}.
% We denote the corresponding sets by $F_g$ respectively $E_g$. 
% An formula or expression which does not contain universal variables is called \emph{universal free}. 
% The set of universal free formulas (possibly containing existentials) is denoted by $F_e$, the set of universal free expressions by $E_e$.
% 
% In the examples in the remainder of this paper, we will use the common \rdf shortcuts:
% 
% \begin{remark}[Syntactic variants]
% \begin{itemize}
% \item A formula consisting of two triple subformulas starting with the same element \verb!<d> <p> <e>. <d> <q> <f>.! can be abbreviated using a semicolon: \verb!<d> <p> <e>; <q> <f>.!\\ 
% Two triple formulas sharing the first two elements  \verb!<d> <p> <e>. <d> <p> <f>.! can be abbreviated using a comma: \verb!<d> <p> <e>, <f>.!
%   \item \verb![]! can be used as an expression and is a shortcut for a new existential variable. So \verb![] <p> <d>.! stands for \verb!_:x <p> <d>.!
%   %If $[~]$ occurs in a formula $f$ instead of an expression, each instance of $[~]$  can be translated by a new existential variable.
%  \item An expression of the form \verb![<p> <o>]! is a shortcut for a new existential variable \verb!_:x!,
%    which is subject to the statement \verb!_:x <p> <o>.!
%    So \verb! <s> <p> [<q> <o>].! stands for \verb!<s> <p> _:x. _:x <q> <o>.!
%  \item \verb!a! is a~shortcut for \verb!rdf:type!~\cite{RDF}.
%  \end{itemize}
% \end{remark}
% 

% 
% 
%%%%%%%%%%%%%%%%%%%%%%%%%%%%%%%%%%%%%%%%%%%%%%%%%%%%%%%%%%%%%%%%%%%%%%%%%%%%%%%%%%%%%%%%%%%%%%%%%%%%%%%%%%%%%%%%%%%%%%%%%%%%%%%%%%%%%%%%%
%Semantics
% %%%%%%%%%%%%%%%%%%%%%%%%%%%%%%%%%%%%%%%%%%%%%%%%%%%%%%%%%%%%%%%%%%%%%%%%%%%%%%%%%%%%%%%%%%%%%%%%%%%%%%%%%%%%%%%%%%%%%%%%%%%%%%%%%%%%%%%%%
% As in \rdf, s
% Simple N3 triples of the form \verb!:s :p :o! can be understood as a first order formula $p(s,o)$. We call \texttt{:p} the predicate, \texttt{:s} 
%  the subject and \texttt{:o} the object of a triple. More complicated constructs often contain variables:
% in Notation3 existential and universal variables %occurring in formulas 
% are implicitly quantified. 


%The definition allows us to 
 %\nthree allows more complex structures, formulas like
% % \begin{Verbatim}[fontsize=\normalsize] 
% %          {?x :p :o.} => {:s :p2 {{?x :p3 :o3}=>{?x :p4 :o4.}.}.}.
% % \end{Verbatim}
% \[
% \verb! {{?x :p :a.} => {?x :q :b.}.} => {{?x :r :c.} => {?x :s :d.}.}.! 
% \]
% where for example the predicate \verb!:p! occurs as a component of level three are valid in \nthree. 
% Such deeply nested structures require a careful treatment of scoping for variables occurring in them. 
% Note for example that the above formula should be interpreted as
% \[
% (\forall x_1: p(x_1,a)\rightarrow q(x_1,b))\rightarrow (\forall x_2: r(x_2,c) \rightarrow s(x_2,d))
% \]
% and \emph{not} as
% \[\forall x: ((p(x,a)\rightarrow q(x,b))\rightarrow ( r(x,c) \rightarrow s(x,d)))\]
% Due to this particularities and because deeply nested structures are no requirement for our framework, we limit the
% considerations of
% this paper to \emph{simple formulas}
% and refer the reader interested in more details to the corresponding publication \cite{semN3}.

We start our formalisation by introducing ground formulas for \nthree. These are similar to the same concept in \nthree 
Core Logic (Definition~\ref{free}) and will become relevant in the following sections. We make use of the concept of \emph{components} as specified in Definition~\ref{formulacomponents}.

\begin{definition}[Ground and universal-free formulas and terms]
Let $\mathcal{A}$ be an \nthree alphabet and let $E$ be the set of existential and $U$ the set of universal variables.
\begin{itemize}
\item We call an \nthree formula $f$ over $\mathcal{A}$ \emph{ground} 
iff $\operatorname{comp}^n(f)\cap E=\emptyset$ and $\operatorname{comp}^n(f)\cap U=\emptyset$ for all $n\in\mathbb{N}$. We denote the set of ground formulas by $\mathcal{F}_g$.
\item We call a term $t$ \emph{ground} iff $t$ is either a constant or a formula expression $\texttt{\{}f\texttt{\}}$ where $f\in \mathcal{F}_g$. We denote the set of all ground terms by $\mathcal{T}_g$.
\item We call an \nthree formula $f$ over $\mathcal{A}$ \emph{universal-free} 
iff $\operatorname{comp}^n(f)\cap U=\emptyset$ for all $n\in\mathbb{N}$. We denote the set of universal-free formulas by $\mathcal{F}_e$.
\item We call a term $t$ \emph{universal-free} iff $t$ is either a constant, an existential or a formula expression $\texttt{\{}f\texttt{\}}$ where $f\in \mathcal{F}_e$. We denote the set of all ground terms by $\mathcal{T}_e$.
\end{itemize}
\end{definition}

A \emph{ground} formula is thus a formula which does neither contain universal nor existential variables. 
Similarly, a universal-free formula is a formula which does not contain implicitly universally quantified variables. 


% Universal variables in \emph{simple formulas} can be understood as universally quantified on the top level of the formula.
% The formula 
% \[
%  \verb!{?x :p :o1.} => {?x :q :o2.}.!
% \]
% is interpreted as 
% \[
%  \forall x: p(x, o_1)\rightarrow q(x, o_2)
% \]
As a next step, we define \emph{substitutions} which are similar to \emph{grounding functions} as we introduced to define the semantics of \nthree Core Logic (Definition~\ref{gf}).
% A special detail of these substitutions is that we have two different ways to apply them: one only for the direct components and one for all direct and nested components of a formula.
% To better understand the reason for this distinction, we consider an example:
Substitutions can be used to replace universals or existentials by terms. We distinguish two ways how such substitutions can be applied: we either apply them to all nested or direct components 
of a formula or -- as a second option -- only to its direct components. 
We do that to reflect how universals and existentials are treated if variables with the same name occur on different levels. In Formula~\ref{mixedlevels} 
\begin{equation}\notag
 \texttt{\{?x :q :b.\} =>\{ :a :b \{?x :s :d.\}.\}.}
\end{equation}
the name \texttt{?x} stands twice for the same variable, the formula means: 
\[
 \forall \texttt{x. <x q b>}\rightarrow\texttt{< a b <x s d>{}>.}
\]
In contrast to that
%To understand the need for the latter way of application we come back to
% 
% As we have seen earlier (Section~\ref{existentials}),
% the scope of an existential variable is always the formula expression it occurs in as a direct component. 
in Formula~\ref{eq1} from earlier: 
% \[
%  \verb!{_:x :p :o1.} => {_:x :q :o2.}.!
% \]
\[
 \verb!_:x :says {_:x :knows :Albert.}.!
\]
the name \texttt{\_:x} stands for different variables.
%This formula contains the blank node \texttt{\_:x} -- an implicitly existentially quantified variable -- twice. The meaning of that formula was:
\[
 \exists \texttt{x1. x1 says <}\exists\texttt{x2. x2 knows Albert>} 
\]
This motivates the following definition:
% The existential quantification of blank nodes, in contrast to universal quantification, 
% only counts for the direct formula they occur in and not for their subordinated formulas.
% \[
% (\exists x_1: p(x_1, o_1))\rightarrow (\exists x_2: q(x_2, o_2))
% \]


% 
% As quantification is implicit, we need to clarify the concept of variables which are free and \emph{accessible}. 
% By accesible we mean that the variable is quantified in the respective formula.
% To understand the need of such a definition consider the following example:
% \[
%  \verb!{?x :p _:y.} => {?x :q _:y.}.!
% \]
% This has to be interpreted as 
% \[\forall x: (\exists y_1: p(x, y_1))\rightarrow (\exists y_2: q(x, y_2))\]
% While the scope of the universal variable is the whole implication, the scope of the existential is limited to antecedent and consequent of the rule. 
% With our definition from above, this means, that universal quantification can be applied to components of level two or one, existential 
% quantification only to components of level one:
% 
% %\todo{example: why}
% \begin{definition}[Accessible variables]
%  Let $f$ be a simple \nthree formula over an alphabet $A$. We define the set of \emph{accessible universals} $\operatorname{AC}_U(f)$ and the set of \emph{accessible
%  existentials} $\operatorname{AC}_E(f)$ as of $f$ follows:
%  \[\text{AC}_U(f):=(\text{comp}^1(f)\cup\text{comp}^2(f))\cap V_U 
% \text{\hspace{0.5cm} and \hspace{0.5cm}}
% \text{AC}_E(f) := 
% \text{comp}(f)\cap V_E
% \]
%  The set of \emph{accessible variables} of $f$ is defined as: $\text{AC}(f)=\text{AC}_U(f)\cup\text{AC}_E(f)$.
% \end{definition}

%To incorporate this fact we will make use the component's definition above as every  
%Furthermore, we need substitutions:

% \begin{definition}[Substitution]
% %
% Let %$\mathfrak{I}$ be an Interpretation of an \nthree alphabet $A$ and 
% $W \subset V$ a set of variables in an \nthree alphabet $A$ and $f$ an \nthree formula.
% \begin{itemize}
% \item A substitution  $\sigma$ of $W$ %in $\mathfrak{I}$ 
% is a map $\sigma:W\rightarrow E$ from $W$ into the set of expressions of $A$. %domain of $\mathfrak{I}$.
% \item We retrieve the \emph{application} $f\sigma$ of a substitution $\sigma$ on a formula $f$ by replacing every variable $x$ of the domain of $\sigma$ which occurs in $f$
% by $\sigma(x)$.
% %By $f\sigma$ we denote the \emph{application} of the substitution $\simga$ on the formula $f$
% \end{itemize}
% \end{definition}
% 
% We will denote substitutions as sets of pairs of the form $x/e$ which indicates that $\sigma: x \mapsto e$. We write $\sigma=\{x_1/e_1,\ldots , x_n/e_n\}$.
%We thus define two ways to apply a substitution:


\begin{definition}[Substitution]
%
Let $\mathcal{A}$ be an \nthree alphabet 
and $f\in \mathcal{F}$ an \nthree formula over~$\mathcal{A}$. 
\begin{itemize}
 \item A \emph{substitution} is a finite set of pairs of terms $\{v_1/t_1, \ldots, v_n/t_n\}$ where each $t_i$ is an term and each $v_i$ is either an existential 
 or a universal such that $v_i\neq t_i$ and 
 $v_i \neq v_j$,
 if $i\neq j$.  
 \item 
 For a formula $f$ and a substitution $\sigma=\{v_1/t_1, \ldots, v_n/t_n\}$, we obtain the \emph{component application} 
 of $\sigma$ to $f$, $f\sigma^c$, by simultaneously replacing each $v_i$ 
 which occurs as a \emph{direct component} in $f$ by the corresponding expression $t_i$. 
 \item 
 For a formula $f$ and a substitution $\sigma=\{v_1/t_1, \ldots, v_n/t_n\}$, we obtain the \emph{total application} of $\sigma$ to $f$, $f\sigma^t$, 
 by simultaneously replacing each $v_i$ 
 which occurs as a \emph{direct or nested component} in $f$ by the corresponding expression~$t_i$. 
 \end{itemize}
\end{definition}

As the definition states, component application of a substitution only changes the direct components of a formula. 
For a substitution $\mu=\{\verb!_:x!/ \verb!:Kurt!\}$ we obtain:
\begin{multline}
(\texttt{\_:x :says \{\_:x :knows :Albert.\}.})\mu^c  =\nonumber \\ (\texttt{ :Kurt :says \{\_:x :knows :Albert.\}.})\nonumber
\end{multline}
A total application of $\sigma=\{\verb!?x!/ \verb!:Kurt!\}$ in contrast, replaces each \emph{occurrence} of a variable in a formula: 
\begin{multline}
(\texttt{?x :says \{?x :knows :Albert.\}.})\sigma^t =\\ (\texttt{:Kurt :says \{:Kurt :knows :Albert.\}.})\nonumber
\end{multline}





%Given a substitution $\sigma$, by $\sigma\{x/e\}$ we denote 
%the substitution which is identical to $\sigma$ except for $x\mapsto e$.


% But even for simple formulas there is one property regarding the behavior of existential variables (blank nodes) which deserves special treatment: 
% the scope of an existential variable is always limited to the formula expression (i.e. the brackets \verb!{!~\verb!}!) it occurs in. The formula
% \[
%  \verb!_:x :says {_:x :knows :Albert.}.!
% \]
% 
% Is interpreted as
% \[
% %\exists x : \text{says}(x, \text{knows}(x, \text{Albert})) 
% %\textit{\quad or \quad}
% \exists x_1 : \text{says}(x_1, (\exists x_2: \text{knows}(x_2, \text{Albert})))\]
% 
% and \emph{not} as 
% \[
%  \exists x : \text{says}(x, \text{knows}(x, \text{Albert}))
% \]
% 
% We will assume in our formalization that existential variables 

% As the definition states, component application of a substitution only changes the direct components of a formula. For a substitution $\sigma=\{\verb!?x!/ \verb!:Kurt!\}$ we obtain:
% \[
% (\verb!?x :says {?x :knows :Albert.}.!)\sigma^c = (\verb!:Kurt :says {?x :knows :Albert.}.!)
% \]
% A total application, in contrast, replaces each \emph{occurrence} of a variable in a formula. with the same example as above, by applying $\sigma$ as a total substitution, we get:
% \[
% (\verb!?x :says {?x :knows :Albert.}.!)\sigma^t = (\verb!:Kurt :says {:Kurt :knows :Albert.}.!)
% \]

%For the following definition, we count the components of each level of a formula $f$ from left to right according to their position in $f$.
%This enables us to state how the different kinds of variables in a formula should be treated by an interpretation:




% \begin{definition}[
% Formula expressions with unaccessible 
% variables]\label{evilexpressions}
% Let $A$ be an \nthree alphabet. 
% For a formula $g\in F$ over $A$ we define the set of %ungroundable 
% formula expressions which contain for $g$ unaccessible universals %,  $\text{FE}_X(g)$ 
% as follows:
% \[ \text{FE}_U(g):=\{\underline{\texttt{\{}}f\underline{\texttt{\}}}\in \text{FE}| \forall v\in V_U: n_f(v)\neq 1 \text{ and if } n_f(v)>1 \text{ then } n_g(v)=0  \} \]
% By $\text{FE}_U:=\bigcup_{g\in F} \text{FE}_U(g)$ we denote the union of all formula expressions of that kind. %with unaccessible universals. 
% The set of formula expressions with unaccessible existentials is defined as:
% \[
%  \text{FE}_E=\{\underline{\texttt{\{}}f\underline{\texttt{\}}}\in \text{FE}|\exists n\in \mathbb{N}: \text{comp}^n(f)\cap U_E\neq \emptyset\}
% \]
% By $\text{FE}_X:=\text{FE}_U\cup \text{FE}_E$ we denote the set of all formula expressions containing unaccessible variables, 
% by $\text{FE}_X(g):=\text{FE}_U(g)\cup \text{FE}_E$ the corresponding set for the formula~$g$.
% \end{definition}
% 
% Note that existential and universal replacements are uniquely defined by the set of their substitutions.\\
% %Note, that both definitions can be understood as a set of substitutions which can be applied to 
% %Both replacements allow different substitutions for the same variable. While in existential replacement a substitution is always only applied 
% %to the \emph{components} of one level of a formula, the substituions used within universal replacement change each \emph{occurrence} of a variable in a (part-)formula. 
% This definition states that only existential variables occurring as components in the same (sub-)formula should be treated equally. 
% This is different for universal variables. 
% To understand how nesting level two is of importance, we come back to our previous example:
% \[
% f=(\verb!{{?x :p :a.} => {?x :q :b.}.} => {{?x :r :c.} => {?x :s :d.}.}.!) 
% \]
% Suppose we want to apply a universal replacement $\mu= (\sigma_{ij})_{i,j \in \mathbb{N}}$ for \verb!?x!:
% 
% As $n_f(\verb!?x!)=3>2$ we have to consider the subformulas
% \[f_{21}=(\verb!{?x :p :a.} => {?x :q :b.}.!)\] and \[f_{22}=(\verb!{{?x :r :c.} => {?x :s :d}.}.!)\] separately.
% 
% For $f_{21}$ we get $n_{f_{21}}(\verb!?x!)=2$. Therefore we apply the substitution $\sigma_{21}= \{\verb!?x!/e_{21}\}$.
% The same holds for $f_{22}$. We get $n_{f_{22}}(\verb!?x!)=2$. Hence we apply the substitution $\sigma_{22}= \{\verb!?x!/e_{22}\}$.
% As a result we get:
% \[
% \mu(f)=(\verb!{{!e_{21}\verb! :p :a.} => {!e_{21}\verb! :q :b.}.} => {{!e_{22} \verb! :r :c.} => {! e_{22}\verb! :s :d.}.}.!) 
% \]
% 
% 
% Both substitutions are applied via total applications such that, 
% if the respective variable occurred in sub-formulas, it would be handled the same way as it is as a~component of 
% $f_{21}$ and $f_{22}$. Note that $e_{21}$ and $e_{22}$ can be different. Our definition only ensures, that, 
% if the same variable occurs in direct sibling formulas, it is treated in the same way.

%For simple formulas 
We can now define interpretation and semantics:
%We embed our concepts in the definition of interpretation and semantics:

\begin{definition}[Interpretation]
An interpretation $\mathfrak{I}$ of
an \nthree alphabet $\mathcal{A}$ consists of:
\begin{enumerate}
\item A set $\mathcal{D}$ called the domain of $\mathfrak{I}$.
\item A function $\mathfrak{a}: 
\mathcal{T}\setminus (U\cup E) \rightarrow \mathcal{D}$ called the object function.
\item A function $\mathfrak{p}:
\mathcal{D} \rightarrow 2^{\mathcal{D} \times \mathcal{D}}$ called the predicate function.
\end{enumerate}
\end{definition}

Note that the above definition is very close to the definition of a structure in \nthree Core Logic (Definition~\ref{n3corestructure}) 
and thereby also to structures in \rdf (Section~\ref{rdfsemantics} and \cite{RDFSemantics}).
%Note that in contrast to the classical definition of \rdf-semantics \cite{RDFSemantics} 
The main difference is that our domain does not distinguish between properties (IP) 
and resources (IR). 
The definitions are nevertheless compatible, as we assume $\mathfrak{p}(p)=\emptyset\in 2^{\mathcal{D} \times \mathcal{D}}$
for all resources $p$ which are not properties (i.e. $p \in \text{IR}\setminus \text{IP}$ in the \rdf-sense). 
% By extending given \rdf ground interpretation functions to Notation3 interpretation functions, 
% the meaning of all valid \rdf triples can be kept in Notation3 Logic. %\\
% 
%The main necessary extension would be a function which assigns domain values to formula expressions. 
As for \nthree Core formulas, we define the structure of simple \nthree formulas only for ground terms and use a substitution (grounding function) instead of a classical valuation 
function which maps the variables directly into the domain of discourse. The reason for that is the same as discussed above: cited formulas should not be ``\emph{referentially transparent}''~\cite[p.7]{N3Logic}, ie 
an interpretation needs to treat the formula expressions \texttt{\{:Superman :can :fly\}} and \texttt{\{:ClarkKent :can :fly\}} differently even if \texttt{:Superman} and \texttt{:ClarkKent} refer to the same 
element of the domain of discourse. As a consequence, the representation level also needs to be taken into account when we assign the meaning to variables.
For our definition of \nthree's direct semantics for simple formulas, we now take the two different ways of applying such a substitution into account:





\begin{definition}[Direct semantics of \nthree\label{sem_n3}]
Let $\mathfrak{I}=(\mathcal{D},\mathfrak{a,p})$ be an interpretation of an \nthree alphabet $\mathcal{A}$, let $U$ be the set of universal variable of $\mathcal{A}$ and $E$ the set of existentials,
and let  $f$ be a \emph{simple} formula over $\mathcal{A}$. % which contains at least one variable, $f_1$ and $f_2$ be ground formulas and $c_1,c_2,p \in E$ be ground expressions.
Then the following holds:
\begin{enumerate}
 \item\label{quant1} If $f$ contains universal variables then $\mathfrak{I}\models f$ iff $\mathfrak{I}\models f\sigma^t$ 
 for every substitution $\sigma: U\rightarrow \mathcal{T}_e$. 
 \item \label{quant2} If $f$ is universal-free and $W=\text{comp}(f)\cap E \neq \emptyset$ then $\mathfrak{I}\models f$ 
 iff $\mathfrak{I}\models f\mu^c$ for some substitution $\mu: W\rightarrow \mathcal{T}_g$.
%\end{enumerate %\begin{enumerate}
  %\item  \label{quant} If $\text{AC}(f)\neq \emptyset$ then
  %$\mathfrak{I}\models f$ iff for all substitutions 
  %$\sigma: \text{AC}_U(f)\rightarrow E_g$ %\cup\text{FE}_E$ 
  %there 
  %exist a substitution $\mu: \text{AC}_E(f) \rightarrow E_g$ %\cup \text{FE}_E$ 
  %such that:
  %$\mathfrak{I}\models f\sigma \mu$
  \item If $f$ is universal free and $\text{comp}(f)\cap E=\emptyset$:
  \begin{enumerate}
   \item If $f$ is an atomic formula $t_1\, t_2\, t_3$, then  $\mathfrak{I} \models t_1\, t_2\, t_3$. 
  iff $(\mathfrak{a}(t_1),\mathfrak{a}(t_3))\in\mathfrak{p}(\mathfrak{a}(t_2))$.
  \item If $f$ is a conjunction $f_1f_2$, then  $\mathfrak{I}\models f_1 f_2$ iff $\mathfrak{I}\models f_1$ and $\mathfrak{I}\models f_2$.\label{conj}
  \item If $f$ is an implication\label{implication} then %of the form 
  \begin{itemize}
  %\item %$\verb!{! f_1 \verb!}! \verb!=>! \verb!{! f_2 \verb!}!$, then 
  \item $\mathfrak{I} \models \verb!{! f_1 \verb!}! \verb!=>! \verb!{! f_2 \verb!}!$ iff $\mathfrak{I} \models f_2$ if $\mathfrak{I} \models f_1$.
  %
  \item \label{fal2} %$\verb!{! f_1 \verb!}! \,\verb!=>!\, \verb!false!$. then 
  $\mathfrak{I} \models \verb!{! f_1 \verb!}! \,\verb!=>!\, \verb!false!$. iff $\mathfrak{I} \not\models f_1$.
   \item $\mathfrak{I} \models \verb!{ }! \verb!=>! \verb!{! f_2 \verb!}!$. iff $\mathfrak{I} \models f_2$.
   \item $\mathfrak{I} \models \texttt{false => }e$ and $\mathfrak{I}\models e\texttt{ => \{~\}}.$, for every expression $e$.
   \end{itemize}
 \end{enumerate}
\end{enumerate}
  
\end{definition}

In point~\ref{quant1} in the definition we first ground all uniersal variables. This is the same as assuming all universals to always be quantified on top level. Our semantics is thus following 
EYE's interpretation but is also correct for all \emph{simple formulas} if we follow the semantics as implemented in Cwm.
By grounding the universals first and then treating existentials afterwards (point~\ref{quant2}) we furthermore make sure 
that in case of conflicts the universal quantifier is outside of the existential as it was stated in Quote~I from Section~\ref{quantsec}.
% In N3 the statement
% \[
%  \verb!?x :loves _:y.!
% \]
% has to be interpreted as $\forall x \exists y: \text{loves}(x,y)
% \text{ and \emph{not} as }
% \exists y \forall x : \text{loves}(x,y)$.


%\noindent
With the above we can now define models:

\begin{definition}[Model]
Let $\Phi$ be a set of \nthree formulas. We call an interpretation $\mathfrak{I}=(\mathcal{D},\mathfrak{a,p})$ a \textit{model} of $\Phi$ iff $\mathfrak{I}\models f$ for every formula $f\in \Phi$.
\end{definition}
%
As in first order logic, we can define the notion of logical implication:

\begin{definition}[Logical implication]\label{log_impl}
Let $\Phi$ be a set of \nthree formulas  and $\phi$ a formula over the same \nthree alphabet $A$. We say that $\Phi$ (logical) implies 
$\phi$ ($\Phi \models \phi$) iff every
model $\mathfrak{I}\models \Phi$ is also a model of $\phi$.
\end{definition}

% To conclude this section and to make a connection to other formalisms we summarize:
% \begin{itemize}
%  \item Simple triples in N3 of the form \verb!:s :p :o.! can be understood as a first order formula $p(s,o)$ with $p$ being a predicate, $s$ and $p$ being 
%  constants. We call $s$ the subject of the formula, $p$ the predicate and $o$ the object.
%  \item N3 supports implicit quantification. 
%   For \emph{simple} formulas the scope of a universal variable is always the formula itself, 
% \end{itemize}


%\todo{elaborate}

%Note that if we restrict Notation3 Logic to nesting level
%The presented subset of Notation3 Logic corresponds to 



\section{A proof calculus for N3}\label{cal}
%\subsection{Anatomy of a~Hypermedia API Composition Proof}
\label{sec:ProofAnatomy}
Having the direct semantics of (simple) \nthree formulas defined we can now come to the actual topic of this chapter: proofs. When new knowledge is derived,
a proof is a list of the different concrete steps 
 applied which led to this new knowledge. This proof can be used by others to understand and either verify or decline the derivations. A proof calculus is now a set of all 
possible conceptual steps allowed to be used in a proof.\footnote{More about proof calculi can be found in most introductions to FOL, for example \cite{ebbinghaus,enderton,mendelson}.}
The proof calculus we discuss below is based on the steps which can be described by the \nthree proof vocabulary which we further describe in the next section.
% 
% introduce a proof calculus for \nthree. This calculus contains 
% all steps which are covered in the proof vocabulary created in the context of the Semantic Web Application Platform (SWAP) \cite{SWAP}. We discuss the vocabulary itself in the following section.
% % 
% The \nthree proof vocabulary created in the context of the Semantic Web Application Platform (SWAP) \cite{SWAP} 
% enables us to formalize proofs in a~machine-readable way.
% This subsection gives a short introduction into the terminology used and the resulting proofs,
% focusing on the aspects relevant to our purposes.
% 
% A proof is a conjunction of \nthree formulas describing 
% inference steps
% a reasoner has performed to come to a certain conclusion, so called \textit{proof steps}. 

The \nthree proof calculus contains four different kinds of proof steps. We write them as deduction rules, using ``$\vdash$''.

\begin{definition}[Proof steps]\label{proofsteps}
Let $\mathcal{F}$ be the set of simple formulas over an \nthree alphabet $\mathcal{A}$, $U$ the set of universals and $E$ the set of existentials in $\mathcal{A}$,  $\Gamma\subset \mathcal{F}$ a set of formulas and
$f,f_1,f_2, g\in \mathcal{F}$. 
A \textit{proof step} is one of the following inference rules:
\begin{enumerate}
 \item \emph{Axiom:} If $f \in \Gamma$ then $\Gamma \vdash f$.%Parsing: The step of reading formulas out of a source file. If we see each soucre file as a formula, this step can be understood as 
 \item \emph{Conjunction elimination:} If $\Gamma \vdash f_1f_2$ then $\Gamma \vdash f_1$ and $\Gamma \vdash f_2$.
 \item \emph{Conjunction introduction:} Let $\Gamma\vdash f_1$ and $\Gamma \vdash f_2$ and 
 let \[\sigma: U\rightarrow U\setminus (\bigcup_{n\in \mathbb{N}_{>0}}\text{comp}^n(f_1) )  %\text{comp}^1(f_1)\cup\text{comp}^2(f_1)) 
 \text{ and } %$\mu$ be a substitution with
 \mu:E\rightarrow E\setminus \text{comp}(f_1)\] be substitutions. Let $f_2'= f_2\sigma^t\mu^c$ 
 %
 %$\rho_1, \ldots, \rho_m$ be existential %and $\mu_1,\ldots, \mu_n$ universal 
 %renamings such that 
 %$\Gamma \vdash f_1f_2'$ for \[
 %$f_2'= \rho_1\circ \ldots \circ \rho_m(f_2)$ with % being a renamed version of $f_2$ such that
 %$\text{comp}(f_1)\cap \text{comp}(f_2') \cap V_E = \emptyset$ %and $\text{comp}^2(f_1)\cap \text{comp}^2(f_2') \cap V_U = \emptyset$
 %
%  is a renamed version of $f_2$ such that \[\{\text{comp}(f_1)\cup \text{comp}^2(f_1)\}\cap \{\text{comp}(f_2')\cup \text{comp}^2(f_2')\}\cap V=\emptyset,\] 
%  then $\Gamma \vdash f_1f_2'$.
then \[\Gamma \vdash f_1f_2'\]
 \item \emph{Generalized modus ponens:} If $\Gamma \vdash \verb!{!f_1\verb!}=>{!f_2\verb!}.!$ and $\Gamma \vdash g$ %\mu_{1}\circ\ldots\circ\mu_{n}(f_1)$ then 
 and there exists a substitution $\sigma:\text{comp}(f_1)\cap U \rightarrow \mathcal{T}_e$ such that 
 \[
  (\verb!{!f_1\verb!}=>{!f_2\verb!}.!)\sigma^t= (\verb!{!f_1'\verb!}=>{!f_2'\verb!}.!)\text{ and }
 f'_1=g\] 
 %universal replacements $\mu_{1}, \ldots, \mu_{n}$  %for the variables in  $\verb!{!f_1\verb!}=>{!f_2\verb!}!.$ %$\Gamma \vdash \mu_{1}\circ\ldots\circ\mu_{n}(f_2)$.
%such that \[\mu_{1}\circ\ldots\circ\mu_{n}(\verb!{!f_1\verb!}=>{!f_2\verb!}.)!=  (\verb!{!f_1'\verb!}=>{!f_2'\verb!}.!) \text{ and } f_1'=g\] 
then $\Gamma \vdash f_2'$.
 \end{enumerate}
\end{definition}

When we define proof steps in order to explain derivations done in a logic, the most important requirement these steps need to fulfil is that they need to be correct: Only 
if we can be sure that the application of a proof step on valid formulas leads again to valid formulas we can trust the proofs which have been produced applying this calculus.
%, that is that  
We therefore state the following theorem:

\begin{theorem}[Correctness of proof calculus]\label{correctness}
Let $\Phi$ be a set of \nthree formulas  and $\phi$ a formula over the same \nthree alphabet $\mathcal{A}$. Let $U$ be the set of universals and $E$ the set of existentials in $\mathcal{A}$. 
Then the following holds:
\[ \text{If } \Phi \vdash \phi \text{ then } \Phi \models \phi.\]
\end{theorem}



\begin{proof}

We prove that every proof step is correct.
%\begin{addmargin}[3cm]{2cm}
%Inhalt
%\end{addmargin} 
\begin{enumerate}
%\leftskip1.5em %this is a quick fix, if you know a better solution, please apply it here
 \item \emph{Axiom:} For the axiom step the claim is trivial, as it corresponds to Definition~\ref{log_impl}.
 \item \emph{Conjunction elimination:}
 Let $\Phi\models f_1 f_2 $ and let  $\mathfrak{I}=(\mathcal{D},\mathfrak{a,p})$ be a model for $\Phi$ and $f_1 f_2$. 
 \begin{itemize}
 \item If $f_1 f_2$ is universal-free and $\text{comp}(f_1f_2)\cap E=\emptyset$, the claim follows immediately 
 from Definition  \ref{sem_n3}.\ref{conj}. 
 
 \item If $f_1f_2$ universal-free and $\text{comp}(f_1f_2)\neq \emptyset$ let $\mu:\text{comp}(f_1f_2)\rightarrow \mathcal{T}_g$ be a substitution such that
 $\mathfrak{I}\models (f_1f_2)\mu^c$. Then $\mathfrak{I}\models (f_1\mu^c) (f_2\mu^c)$, thus  $\mathfrak{I}\models (f_1\mu^c)$ and 
 $\mathfrak{I}\models (f_2\mu^c)$.
 \item If $f_1f_2$ are not universal-free, then $\mathfrak{I}\models (f_1f_2)\sigma^t$ for all substitutions $\sigma:U\rightarrow \mathcal{T}_e$. 
 
 The claim follows by the same 
 argument as above.
 \end{itemize}
%  If $f_1 f_2$ is not ground let $\mu_1,\ldots ,\mu_n$ be universal replacements and $\rho_1,\ldots \rho_m$ existential replacements for all variables 
%  occurring in $f_1 f_2$ as defined in 
%  Definition~\ref{sem_n3}.\ref{quant} such that:
%  \[
%   \mathfrak{I}\models \rho_1\circ \ldots \circ \rho_m \circ \mu_1 \circ \ldots \circ \mu_n (f_1 f_2)
%  \]
% It follows immediately by Definition \ref{repl} and \ref{sem_n3}.\ref{conj}:
%   \[
%   \mathfrak{I} \models \rho_1\circ \ldots \circ \rho_m \circ \mu_1 \circ \ldots \circ \mu_n (f_1 )
%  \] 
% To prove the claim for $f_2$ we take into account, that according to Definition \ref{repl} the only 
% substitution of an existential or universal replacement, which changes 
% variables in both subformulas $f_1$ and $f_2$ from $f_1f_2$ is $\sigma_{11}$.
% We construct new replacements $\nu'$ for each replacement $\nu = (\sigma_{ij})_{i,j \in \mathbb{N}}$ from above as follows:\\
% Let $k_1:=0$ and for each level $i$ of $f_1$ let $k_i := |\text{comp}^{(i-1)}(f_1)\cap \text{FE}|$ be the number of formula expressions. % of $f_1$. 
% We define
% $\nu' := (\sigma_{i (j+k_i)})_{i,j \in \mathbb{N}}.$ Then the following holds:
%   \[
%   \mathfrak{I} \models \rho_1'\circ \ldots \circ \rho_m' \circ \mu_1' \circ \ldots \circ \mu_n' (f_2 )
%  \]
 \item \emph{Conjunction introduction:} 
 Let $\Phi\models f_1$, $\Phi\models f_2$ and let  $\mathfrak{I}=(\mathcal{D},\mathfrak{a,p})$ be a model for $\Phi$, $f_1$ and $f_2$. 
 %As $f_2'$ is just a renamed version of $f_2$, $\Phi\models f_2'$. The claim follows by Definition \ref{sem_n3}.\ref{conj}.
%An existential renaming does not change the meaning of a formula.
As the renaming substitutions $\sigma$ and $\mu$ do not change the meaning of a formula, for  $f_2'= f_2\sigma^t\mu^c$ the following holds:
%an existential renaming does not change the meaning of a formula, 
$\mathfrak{I}\models f_2'$. 
%If $f_1$ or $f_2'$ is universal free 
It immediately follows that $\mathfrak{I}\models f_1f_2'$. %If $f_1f_2'$ both contain universals
%for $f_2'= \rho_1\circ \ldots \circ \rho_m(f_2)$, $\rho_1,\ldots,\rho_m
% Let $\rho^1_1, \ldots, \rho^1_{m_1},\rho^2_1, \ldots, \rho^2_{m_2}$ be existential replacements and $\mu^1_1,\ldots \mu^1_{n_1}, \mu^2_1,\ldots \mu^2_{n_2}$ universal 
% replacements as in Definition \ref{sem_n3}.\ref{quant} such that:
%  \[
%   \mathfrak{I}\models \rho^1_1\circ \ldots \circ \rho^1_{m_1} \circ \mu^1_1 \circ \ldots \circ \mu^1_{n_1} (f_1)
%  \]
% \text{ and }
%  \[
%   \mathfrak{I}\models \rho^2_1\circ \ldots \circ \rho^2_{m_2} \circ \mu^2_1 \circ \ldots \circ \mu^2_{n_2} (f_2')
%  \]
% We construct new replacements for each variable $v$ %\in \text{dom}(\nu^1)\cup \text{dom}(\nu^2)$ 
% which occurs in $f_1$ or $f_2'$:
% 
% %If $v$ only occurs in one of the two formulas, 
% % If $v$ only occurs in $f_1$ we keep the replacement and rename and renumber it 
% % to 
% % $\rho'_l$ resp. $\mu'_l$. 
% If $v$ only occurs in one of the formulas there is only one replacement $\nu^1$ or $\nu^2$ for $v$, we add an additional arbitrary 
% replacement $\nu^2$ respectively $\nu^1$ with $\text{dom}(\nu^i)= \{v\}$ to our set of replacements.
% %If there is no replacement $\nu^1$ or $\nu^2$ for $v$ we add an arbitrary 
% %If the variable only occurs in $f_2'$ 
% %we add a new substitution 
% %Now, we always have two replacements for the same variable. We take these the two replacements
% Now, there are always two replacements for each variable $v$. We combine those two replacements
% $\nu^1=(\sigma^1_{ij})_{i,j \in \mathbb{N}}$ and $\nu^2=(\sigma^2_{ij})_{i,j \in \mathbb{N}}$ %for $v$ 
% and % the same variable, 
% construct a new replacement $\nu'=(\sigma'_{ij})_{i,j \in \mathbb{N}}$: 
% Let $k_1:=0$ and for each level $i$ of $f_1$ let $k_i := |\text{comp}^{(i-1)}(f_1)\cap \text{FE}|$ be the number of formula expressions. We define the substitutions 
% $\sigma_{ij}'$ :
% \[
% \sigma_{ij}':= \begin{cases}
%                \sigma^1_{ij} & \text{ if } j \leq k_i \\
%                \sigma^2_{i (j+k_i)} & \text{ else.}
%               \end{cases}
% \]
% The new replacements can be combined and ordered, such that
%  \[
%   \mathfrak{I}\models \rho'_1\circ \ldots \circ \rho'_{m} \circ \mu'_1 \circ \ldots \circ \mu'_{n} (f_1f_2')
%  \]
% As this procedure can be done with every set of replacements \linebreak $\rho^1_1, \ldots, \rho^1_{m_1},\rho^2_1, \ldots, \rho^2_{m_2},\mu^1_1,\ldots \mu^1_{n_1}, \mu^2_1,\ldots \mu^2_{n_2}$, 
% the claim follows by Definition \ref{sem_n3}.\ref{quant} and \ref{sem_n3}.\ref{conj}.
\item \emph{Generalized modus ponens:} the claim follows directly from Definitions \ref{sem_n3}.\ref{quant1} %, \ref{sem_n3}.\ref{quant2}   
and~\ref{sem_n3}.\ref{fal2}.
\end{enumerate}
\end{proof}



% Additional to the above mentioned steps proof steps themseves, %, the reasoner is able to apply existential renaming. 
% the substitutions done during the reasoning process %as well as those applied in generalized modus ponens 
% can be 
% described by the vocabulary\footnote{The vocabulary's rdf definition can be found at: \url{http://www.w3.org/2000/10/swap/reason\#}.}.
%Additional to the replacements done during the application of the generalized modus ponens, the proofs shown within this paper will also include existential renaming. 
%Although this is not expressed as a proof step cannot be expressed as a proof step, the vocabulary provides the opportunity to describe the substitutions made in such a pre-step.
%The reasoning vocabulary\footnote{The vocabulary's definition can be found at: \url{http://www.w3.org/2000/10/swap/reason\#}.}
%enables us to name the aforementioned proof steps. 
%their outcome, and the resources used to come to a conclusion. 
%\todo{Remark: proof calculus is not complete since we do not have any way to come from existential to universal}
The proof calculus is thus correct. % Based on our experiences on proof calculi in first order logic (for example \cite{ebbinghaus,enderton,mendelson})
The natural next step here would be to also prove the completeness of the proof calculus: unfortunately, the calculus is not complete.
% 
% -- there are for example no proof steps
% dealing with existential 
% variables --
% and we can therefore not provide such a proof. 
To see this consider the following rule containing a blank node in its antecedent:
\[
 \texttt{\{\_:x :knows :Kurt.\}=>\{:Kurt :is :known\}.}
\]
Before they apply this rule, the reasoners Cwm and EYE internally translate it to:
\[
 \texttt{\{?x :knows :Kurt.\}=>\{:Kurt :is :known\}.}
\]
According to all definitions of \nthree's semantics we have given so far, this translation is correct. The step (or combination of steps depending on the calculus) 
performed here is not part of the proof calculus described by the SWAP
proof vocabulary. Therefore, it is impossible for reasoners applying this step to communicate this application to different parties. To also be complete or to at least reflect the actual 
reasoning steps performed by the reasoners, the SWAP vocabulary would need to be extended by proof steps which act on existential variables.

Before we introduce the proof vocabulary, we make an observation about the generalised modus ponens which will become important in the following chapter:
% 
% We will examine the generalized modus ponens in more detail,
% as this is the proof step where implication rules, 
% such as \restdesc descriptions, are applied.

\begin{lemma}\label{lemma:Reasoning}
Let $\mathcal{A}$ be an \nthree alphabet, $U$ the set of universal variables, %$\Gamma\subset F$ a set of formulas, 
$f\in \mathcal{F}_g$ a ground formula and $\verb!{!f_1\verb!}=>{!f_2\verb!}!\in \mathcal{F}$ an implication formula %of nesting level 2 
where all 
universal variables which occur in $f_2$  
also occur in $f_1$. If the generalized modus ponens is applicable to $f$ and $\verb!{!f_1\verb!}=>{!f_2\verb!}!$ then
the resulting formula does not contain universal variables.
\end{lemma}

\begin{proof}
 %Let  $\mu_{1}, \ldots, \mu_{n}$ be 
 %universal replacements 
 Let $\sigma:U\rightarrow \mathcal{T}_e$ be a substitution %such that 
 such that  $(\verb!{!f_1\verb!}=>{!f_2\verb!}.!)\sigma^t= \verb!{!f\verb!}=>{!f_2'\verb!}.!$ 
As $f$ is a ground formula, $\text{range}(\sigma|_{U\cap(\bigcup_{i\geq 1}\text{comp}^i(f_1))}) \subset E_g$. 
With the condition that every universal variable of $f_2$ 
is also in $f_1$, that is
\[((\bigcup_{i\leq 1}\text{comp}^i(f_2))\cap U )\subset((\bigcup_{i\leq 1}\text{comp}^i(f_1))\cap U )\]
the claim follows. 
%
 %As the implication
 %has nesting level 2, by Definition \ref{repl}, from each replacement only the first substitution $\sigma^{\mu_i}_{11}=: \sigma_{\mu_i}$ 
 %gets totally applied to the whole implication 
 %and we obtain
 %there are, by definition \ref{repl}, some substitutions $\sigma_1\ldots\sigma_n$ such that 
 %\[\mu_{1}\circ\ldots\circ\mu_{n}(\verb!{!f_1\verb!}=>{!f_2\verb!}.!)=(\verb!{!f_1\verb!}=>{!f_2\verb!}.!)\sigma_{\mu_1}^t\ldots\sigma_{\mu_n}^t.\]
 %Therefore all 
 %universal variables which are replaced in $f_1$ are also replaced by the same substitution in $f_2$.
\end{proof}



\section{The proof vocabulary}\label{sec:proofvoc}
%As explained earlier, the proof steps we introduced in the previous section all have been defined in the context of the Semantic Web Application platform SWAP. 
%The proof steps introduced in the previous section can all be expressed in \nthree.
After having introduced the concrete proof steps which can be described by it in the last section, we now introduce the SWAP proof vocabulary.\footnote{The vocabulary's RDF definition can be found at: \url{http://www.w3.org/2000/10/swap/reason\#}.}
This proof vocabulary makes it possible 
to express proofs directly in \nthree and thereby make them accessible for Semantic Web technology. Semantic Web reasoners can process proofs, exchange them among each others 
or -- for example -- check them for correctness. By expressing proofs in \nthree, we thus make them part of the Semantic Web.

Below, we define the \nthree names of the steps listed in Definition~\ref{proofsteps}. From a technical point of view, we can only be aware of \emph{axioms} 
if we previously parse them. In the proof vocabulary, the axiom step is therefore named after this action.
% In the \nthree proof vocabulary
% we will discuss next,
% this step is therefore named after this action.

\begin{definition}[Proof vocabulary]\label{proofvoc}
Let $\mathcal{A}$ be an \nthree alphabet and $\mathfrak{I}=(\mathcal{D}, \mathfrak{a}, \mathfrak{p})$ be an 
interpretation of its formulas. Let $C$ be the set of constants in $\mathcal{A}$. Let $\verb!x!, \verb!y!,  \verb!y!_1, \ldots, \verb!y!_n \in C$ be \nthree representations of 
proof steps and $\verb!z!_1,\verb!z!_2,\verb!z!_3\in C$.
\begin{enumerate}
 \item 
 Proof step types:
 \begin{itemize}
\item $\mathfrak{I}\models \verb!x a r:Proof.!$ iff \verb!x! is the proof step which leads to the proven result.
\item $\mathfrak{I}\models \verb!x a r:Parsing.!$ iff \verb!x! is a parsed axiom. %This step involves parsing the resources.
\item $\mathfrak{I}\models \verb!x a r:Conjunction.!$ iff \verb!x! is a conjunction introduction.
\item $\mathfrak{I}\models \verb!x a r:Inference.!$ iff \verb!x! is a generalized modus ponens.
\item $\mathfrak{I}\models \verb!x a r:Extraction.!$ iff \verb!x! is a conjunction elimination.
\end{itemize}
 \item 
 Proof predicates:
\begin{itemize}
\item $\mathfrak{I}\models \verb!x r:gives {!f\verb!}!$. iff $f\in F$ is the formula obtained by applying \verb!x!.
\item $\mathfrak{I}\models \verb!x r:source u!$. iff \verb!x! is a parsed axiom and $\verb!u!\in U$ is the \uri of the %parsing's 
parsed axiom's
source. 
\item $\mathfrak{I}\models \verb!x r:component y!$. iff \verb!x! is a conjunction introduction and $\verb!y!$ is a proof step which gives one of its components.
\item $\mathfrak{I}\models \verb!x r:rule y.!$ iff \verb!x! is a generalized modus ponens and $\verb!y!$ is the proof step which leads to the applied implication.
\item $\mathfrak{I}\models \verb!x r:evidence (y!_1,\ldots,\verb!y!_n\verb!)!$. iff \verb!x! is a generalized modus ponens and $\verb!y!_1,\ldots, \verb!y!_n$ 
are the proof steps which lead to the formulas whose conjunction was used for the unification with the antecedent of the implication.
\item $\mathfrak{I}\models \verb!x r:because y!$. iff \verb!x! is a conjunction elimination and $\verb!y!$ is the proof step which yields the to-be-eliminated conjunction.
\end{itemize}
\item
Substitutions:
\begin{itemize}
 \item $\mathfrak{I}\models \verb!x r:binding z!_1.$ iff \verb!x! includes a substitution $\verb!z!_1$.
 \item $\mathfrak{I}\models \verb!z!_1 \verb! r:variable z!_2.$ iff $\verb!z!_1$ is a substitution whose domain is $\{\verb!z!_2\}$.
 \item $\mathfrak{I}\models \verb!z!_1 \verb! r:boundTo z!_3.$ iff $\verb!z!_1$ is a substitution whose range is $\{\verb!z!_3\}$.
\end{itemize}

\end{enumerate}
\end{definition}


\begin{lstlisting}[
  float=t,
  caption={Example proof: Formula~\ref{out} can be derived by applying Rule~\ref{uni} on Formla~\ref{ref}. },
  label=lst:proofexample]
§\textcolor{gray}{PREFIX : <http://example.org/ex\#>}§
§\textcolor{gray}{PREFIX r: <http://www.w3.org/2000/10/swap/reason\#>}§

[] a r:Proof, r:Conjunction;
  r:component <#lemma1>;
  r:gives {
    :Albert :knows :Kurt.
  }.

<#lemma1> a r:Inference;
  r:gives {
    :Albert :knows :Kurt.
  };
  r:evidence (
    <#lemma2>
  );
  r:rule <#lemma3>.

<#lemma2> a r:Extraction;
  r:gives {
    :Kurt :knows :Albert.
  };
  r:because [ a r:Parsing; r:source <Formula_2.1>].

<#lemma3> a r:Extraction;
  r:gives {
    {:Kurt :knows ?x_0_1} => {?x_0_1 :knows :Kurt}.
  };
  r:because [ a r:Parsing; r:source <Formula_2.7>].
\end{lstlisting}

An \nthree proof is now a conjunction of all the proof steps leading to the proven formula. To illustrate that, we discuss an example proof.
We come back to the formulas we used in Chapter~\ref{unilog} of this thesis. From Formula~\ref{ref}
\[
 \verb! :Kurt :knows :Albert.!
\]
and Rule~\ref{uni}
\begin{multline}\notag
 \texttt{\{:Kurt :knows ?x.\} => }%} \\ \texttt{
 \texttt{\{?x :knows :Kurt.\}.}%\nonumber
\end{multline}
we could derive Formula~\ref{out}.
\begin{equation}\notag
\texttt{:Albert :knows :Kurt.}
\end{equation}
The proof for that derivation is displayed in Listing~\ref{lst:proofexample}. To produce this proof we used Formula~\ref{uni} as a goal. The proof is produced by EYE.
 We go through the proof from the bottom to the top: 
 
\texttt{<\#lemma3>} in line 25 is an \texttt{r:Extraction} which results in the formula
\texttt{\{:Kurt :knows ?x\_0\_1\} => \{?x\_0\_1 :knows :Kurt\}.} as indicated by the predicate \texttt{r:gives}.
This formula was picked (\texttt{r:because}) as one conjunct from another formula produced by another proof step which here is denoted by an anonymous blank node. 
This proof step is an \texttt{r:Parsing}  
%This anonymous formula is an \ % \texttt{r:source} of that \texttt{r:Extraction} was the formula we get by
having as source  (\texttt{r:source})  the source file \texttt{<Formula\_2.7>} which  contains Formula~\ref{uni}.

The description of \texttt{<\#lemma2>} in line 19 is very similar to the above. We again have an \texttt{r:Extraction}, 
this times made on a parsed formula contained in the file \texttt{<Formula\_2.1>}. This step results in the formula \texttt{:Kurt :knows :Albert}.

\texttt{<\#lemma1>} in line 10 now makes use of the two formulas: The lemma is an inference step (\texttt{r:Inference}) which applies the rule (\texttt{r:rule})
produced by \texttt{<\#lemma3>} on the formula produced by \texttt{<\#lemma1>} (\texttt{r:evidence}). This results in the formula \texttt{:Albert :knows :Kurt.}

This last formula is also the result of out proof as indicated in line 4. The last proof step we apply (\texttt{r:Proof}) is a \texttt{r:Conjunction} with only one component, the 
result of \texttt{<\#lemma1>}. We get \texttt{:Albert :knows :Kurt.}



% To produce a proof for an API composition problem, the reasoner needs to be aware of all formulas at its disposal (in our case $H \cup R \cup B$) and of the goal 
% which it is expected to prove.
% The latter is given to the reasoner as the consequence of a~\textit{filter rule}
% $\verb!{!f\verb!} => {!g\verb!}.!$%, the \textit{filter rule}.
% This triggers the reasoner to prove an instance of~$f$ and in case of success,
% return each provable ground instance of $g$ if possible,
% or a provable instance containing existentials otherwise.

The proof above has been automatically produced and in our example some of the proof steps have not really been necessary: 
we do for example not need the conjunction of only one component indicated in line 4. The proof did on the other hand not make use of the different predicates describing a substitution. 
It renamed the variable \texttt{?x} from our input Formula~\ref{uni} to \texttt{?x\_0\_1} (line 27) and then unified this variable \texttt{?x\_0\_1} with \texttt{:Albert} when executing 
\texttt{<\#lemma1>} (line 10) without letting us know. 

Such and other simplifications are often made by \nthree reasoners producing proofs.
% 
% % Even the proof step \texttt{r:Conjunction} is often applied 
% % 
% % To keep the proof readable and easy to produce, such adjustments are made by many reasoners -- the application of a \texttt{r:Conjunction} step is for example also often not listed 
% 
% 
% For brevity, not all reasoners display every proof step in a proof:
% especially conjunction elimination and introduction are often omitted.
However, to the best of our knowledge,
all reasoners' proofs contain all applications of $\verb!r:Inference!$ leading to a goal~$g$,
which allows us to measure a~proof's length by counting applications of the generalized modus ponens.

% \begin{definition}[Proof length]
% Let $p$ be an \nthree formula representing a proof, ie a conjunction of formulas using the proof vocabulary defined in Definition~\ref{proofvoc}. 
% Then the \emph{length} of the proof is defined as the number of different resources $\texttt{x}$ for which the following triple is stated:
%  \texttt{x a r:Inference.}
% \end{definition}


%\newpage
\section{Conclusion}
In this chapter we introduced and explained the \nthree proof vocabulary: This vocabulary makes it possible to express the derivations performed by a reasoner in a Semantic Web format. 
By using \nthree to express proofs, we make sure that proofs produced by one reasoner can be used as input by another reasoner. Such a reasoner
can then verify the proofs or use them for further reasoning. 

To allow this exchangeability of proofs, these proofs do not only need to share the same syntactic format, they also need to be interpreted equally by all reasoners. 
From our consideration in the earlier chapter, we know that this can be problematic: especially the interpretation of implicitly quantified variables differs between reasoner.
To make the exchange of proofs possible despite the discrepancies in the interpretations of \nthree formulas we 
introduced the concept of \emph{simple formulas}: By excluding deeply nested universals,  we can guarantee for universals to  always quantified on top level. For these formulas 
we then defined a direct semantics which assumes to level quantification for universals. With this direct semantics we proved that the calculus behind the proof vocabulary is correct.
It is therefore save to exchange proofs only containing \emph{simple formulas} among different reasoners.

With this knowledge we now go to the next chapter: we know that proofs \nthree reasoners can produce proofs, exchange and verify them and use them for further applications. 
In the next chapter, we want to take a look at possibilities for such \emph{further applications}. We consider a concrete example: If we describe possible API operation in \nthree we can 
use proofs to automatically combine these APIs and thereby use proofs as a kind of plan. We discuss this idea in detail in the following chapter and also explain how this  idea of 
using proofs for planning can be applied in other scenarios.
%and exchange proofs among different parties.
