
\subsection{\restdesc Descriptions}\label{rd}

\restdesc descriptions are designed to explain how a hypermedia API can be used to perform a specific action. 
Such a process of using an API consists of different steps:
given all needed information, the client sends an \http request to the hypermedia API on a server.
The API interprets the request and,
if possible, reacts by fulfilling the indicated task
or retrieving the information requested. The server sends a~response and thereby creates a new situation. As mentioned in \cref{api}, this process 
is called an API operation.
Hypermedia APIs are commonly able to perform different kinds of operations.

To describe such an operation in \nthree
%As this process makes use of \http, a %we start with a closer look to this protocol and how it can be expressed in \nthree. we start by giving a short overview of the \http predicates used in this paper, how this protocol can be expressed in \nthree
%For the description of hypermedia API operations an \nthree 
a formalization of \http
is needed. 
We use the \rdf vocabulary as defined in the corresponding W3C Working Draft \cite{httprdf}.
In order to facilitate the understanding of the following sections, we give a short overview of the \http predicates used in this paper.

\begin{definition}[\http predicates\label{httppr}]
Let $A$ be an \nthree alphabet,
$M$ be a set of \http method names,
$\{ \texttt{"GET"},\texttt{"POST"},\texttt{"PUT"},\linebreak[1]
\texttt{"DELETE"},\texttt{"HEAD"},\texttt{"PATCH"}\}\subset M \subset L$,
$\verb!u!\in U$ an \http message,
$\verb!v!\in U\cup L$
and $\mathfrak{I}=(\mathcal{D},\mathfrak{a,p})$ an interpretation of $A$.
Then:


\begin{itemize}
 \item  $\mathfrak{I}\models$ \verb!u http:headers v.! iff \texttt{v} is an \http header of \verb!u!.
 \item $\mathfrak{I}\models$ \verb!u http:body v.! iff \verb!v! is the \http body of \verb!u!. 
  \item $\mathfrak{I}\models$ \verb!u http:methodName v.! iff $\verb!u!$ is a request and 
  $\verb!v!\in M$ its method name.
  \item $\mathfrak{I}\models$ \verb!u http:requestURI v.! iff $\verb!u!$ is a request and $\verb!v!$ is its \URL.
  \item $\mathfrak{I}\models$ \verb!u http:resp v.! iff $\verb!u!$ is a request and $\verb!v!$ is its responding \http message.
\end{itemize}
\end{definition}

% \begin{definition}[\http Request]
%  
% \end{definition}

This vocabulary can be used to describe \http-requests.  Such a request must always have a method name and a request \uri.
%With the usual prefix definitions for \texttt{owl}\footnote{\texttt{@prefix owl: <http://www.w3.org/2002/07/owl\#>} }, 
%\texttt{rdf}\footnote{@prefix rdf: <http://www.w3.org/1999/02/22-rdf-syntax-ns\#> .} and \texttt{xsd}\footnote{@prefix xsd: <http://www.w3.org/2001/XMLSchema\#> .},
Using owl-vocabulary, such requests can be defined by the following class:
%If we use the above 
%\begin{lstlisting}[
%  float=t,
%  caption={OWL description of the class \texttt{http:Request}},
%  label=request]
%§\textcolor{gray}{@prefix rdf: <http://www.w3.org/1999/02/22-rdf-syntax-ns\#> .}§
%§\textcolor{gray}{@prefix owl: <http://www.w3.org/2002/07/owl\#> }§
%§\textcolor{gray}{@prefix xsd: <http://www.w3.org/2001/XMLSchema\#> .}§
%§\textcolor{gray}{@prefix http: <http://www.w3.org/2011/http\#>.}§
\begin{verbatim}
http:Request rdfs:subClassOf http:Message, 
    [
    rdf:type     owl:Restriction ;
    owl:onProperty http:methodName ;
    owl:qualifiedCardinality "1"^^xsd:nonNegativeInteger
    ],
    [ 
    rdf:type owl:Restriction ;
    owl:onProperty http:requestURI ;
    owl:qualifiedCardinality "1"^^xsd:nonNegativeInteger
    ] .
\end{verbatim}
    %\end{lstlisting}
These two properties also have to be specified in an \http request description:
%This enables us to define an \http request description:

\begin{definition}[\http request description]
  \label{def:HttpRequestDescription}
 Let $A$ be an \nthree alphabet which contains a set $H$ of \http predicates including those defined in Definition~\ref{httppr}. 
  \begin{itemize}
  % \item An \http request    
\item An \emph{\http request description} is a conjunction $f=f_1 f_2 \ldots f_n\in F$ of atomic formulas with the following properties:
\begin{itemize}
 \item All atomic formulas $f_i$ share the same existential variable $\verb!_:x!\in V_E$ as a subject.
 \item The predicate of each formula $f_i$ is an \http-predicate, i.e. $f_i$ has the form $\verb!_:x :h!_i \verb!:o!_i\verb!.!$ with $\verb!:h!_i\in H$.
 \item The conjunction $f$ contains one atomic formula $f_i$ with the predicate\linebreak \verb! http:methodName!
 %\item The conjunction $f$ contains 
 and one formula $f_j$ with the predicate\linebreak 
 \verb! http:requestURI!.
 \item The object of every atomic formula $f_i=\verb!_:x :h!_i \verb!:o!_i\verb!.!$ with $\verb!h!_i \neq \verb!http:resp!$ is either a universal variable, 
 a URI or a literal, $\verb!:o!_i\in V_U\cup U\cup L$.
\end{itemize}
 %let  
 %$\verb!_:x! \in V_E$ be an existential variable 
 %and $f_1, f_2,\ldots, f_n\in F$, $n \in \mathbb{N}$, $n\geq 2$, be \nthree formulas over $A$.

%  \item We call an \nthree 
%  formula $f=f_1 f_2 \ldots f_n\in F$ 
%  an \textit{\http request description} if:
%  
%  \begin{itemize}
% \item for all $i\leq n$: $f_i=\verb!_:x! \verb! p!_i \verb! b!_i.$ with $\verb!p!_i\in H$, $\verb!b!_i\in V_U\cup U \cup L$.
% \item for one $k\leq n$:  $f_k=\verb!_:x! \verb! http:methodName! \verb! b!_k$. 
% \item for one $j\leq n$:  $f_j=\verb!_:x! \verb! http:requestURI! \verb! b!_j$.  
% \end{itemize}
\item  We call an \http request description $f=f_1\ldots f_n$ \textit{sufficiently specified},
if  $\texttt{:o}_i\in E_g$ for every triple $f_i=\texttt{\_:x :h}_i \texttt{:h}_i$ with $\texttt{:p}_i\neq \texttt{http:resp}$. % i.e. 
%$\texttt{:o}_i\in E_g$
  %if it contains, with the exception of \verb!_:x!, only ground expressions.
\end{itemize}

\end{definition}


The definition reflects %the \emph{minimal} requirements 
the syntactical requirements
to an \http request description, %, such a description 
it should contain  the URL and the method name of the described request and it can contain additional information which can be
described using the \http-predicates.
%it can always contain more formulas.
If the object of these formulas are instantiated, i.e. sufficiently specified, they can be sent to a server and, 
if they contain all necessary information, executed by an API which will return the \http response.
\restdesc descriptions enable us to specify
the intended functionality of a~hypermedia API's operations:

\begin{definition}[\restdesc description]
Let $A$ be an \nthree alphabet containing the predicates defined in Definition \ref{httppr}, $F$ the set of formulas over $A$.
A \restdesc description $f\in F$ of a hypermedia API operation is a \emph{simple} \nthree formula %an \nthree implication %with nesting level $n(f)= 2$, 
%with $\text{comp}^n(f)=\emptyset$ for all $n>2$
of the form:
\[	  \verb!{ precondition } => { http-request  postcondition }.!\]
where \verb!precondition!, \verb!http-request! and \verb!postcondition! are \nthree formulas over $A$ with the following properties:
\begin{enumerate}
 \item \emph{\texttt{precondition}} describes the resources needed to execute the operation and does not contain %the expression \texttt{false} and none 
 any existential variable. %And $comp^n(\texttt{precondition}) = \emptyset$ for all $n>1$.
 \item \emph{\texttt{http-request}} is an \http request description which describes a request which can be used to obtain the desired 
 result of executing the operation. % and 
 It contains no triple having the same subject as the \verb!http-request!.
All universal variables which occur in \verb!http-request!
 do also occur in \verb!precondition!. 
 
 \item \emph{\texttt{postcondition}} describes one or more results obtained by the execution of the operation. All universal variables contained in \verb!postcondition! 
also occur in \verb!precondition!. 
\end{enumerate}


\end{definition}
By making sure that the subject of any triple in the postcondition is different than the subject of the request, we make both syntactical distinguishable. 
%By prohibiting components of a higher level than~2,
%we especially exclude nested rules.
Note that a RESTdesc description is an existential rule as defined in the Datalog$^{\pm}$ framework \cite{datalogpm}:
our restriction on universal variables,
that all universals in the head of the rule should also occur in the body,
%exclude all rules with new 
is very similar to Datalog \cite{datalog}
%rules which introduce new universal variables are excluded and
%and
%But as 
and our rules 
 allow (and expect) new existentials in the consequence. 
%  The reason for this will become clear in the following sections, but the basic idea here is
%  to express knowledge and situations which cannot be obtained by reasoning 
%  using existential variables: we are aware of the existence of facts or instances, but we have to act
%  to get further details.
 %As we will further explain in the following sections
%We will see in the coming sections, especially in section
%we cannot guarantee that the models of ground formulas (facts)
%and \restdesc rules are finite,
%as is the case with Datalog. 
%\todo{make the connection to datalog stronger}
%\todo{!}

The reasons for the restrictions will become more clear in \cref{sec:Reasoning},
where we show that for every ground instance of the precondition,
the \http request is sufficiently specified and the postcondition will not contain any universal variables.
From an operational point of view, the remaining existential variables in the postcondition
are those which are expected to be grounded through the execution of the \http request.

\begin{lstlisting}[
  float=t,
  caption={\restdesc description of the action ``obtaining a~thumbnail'' \emph{(desc\_thumbnail.n3)}},
  label=lst:Thumbnail]
§\textcolor{gray}{@prefix dbpedia: <http://dbpedia.org/resource/>.}§
§\textcolor{gray}{@prefix dbpedia-owl: <http://dbpedia.org/ontology/>.}§
§\textcolor{gray}{@prefix ex: <http://example.org/image\#>.}§
§\textcolor{gray}{@prefix http: <http://www.w3.org/2011/http\#>.}§

{ ?image ex:smallThumbnail ?thumbnail. }
=>
{
  _:request http:methodName "GET";
            http:requestURI ?thumbnail;
            http:resp [ http:body ?thumbnail ].

  ?image dbpedia-owl:thumbnail ?thumbnail.
  ?thumbnail a dbpedia:Image;
             dbpedia-owl:height 80.0.
}.
\end{lstlisting}

%To illustrate our definition on a~concrete hypermedia API example,
\Cref{lst:Thumbnail} shows a~description
that explains the \verb!smallThumbnail! relation in a~hypermedia API.
The \emph{precondition} demands the existence of
a~\verb!smallThumbnail! hyperlink
between an \verb!?image! resource
and a~\verb!?thumbnail! resource.
The \http request
is a~\verb!GET! request to the \URL of \verb!?thumbnail!.
The response to this request will be a~representation of \verb!?thumbnail!.
These characteristics of this representation are detailed in the remainder of the \emph{postcondition}.
This states that the original \verb!?image!
will be in a~\verb!thumbnail! relationship
(the meaning of which is defined by the \dbpedia ontology~\cite{DBpedia})
with \verb!?thumbnail!.
Furthermore, \verb!?thumbnail! will be an \verb!Image!
and have a~\verb!height! of \verb!80.0!.

There are two different ways to interpret this description:
First the declarative, static way as defined in \cref{nthree},
which could be phrased as
\enquote{the existence of the \texttt{smallThumbnail} relationship
implies the existence of a~\texttt{GET} request
which leads to an~80px-high thumbnail of this image.}

The second interpretation is the operational, dynamic way.
In this case, a~software agent has a~description of the world,
against which the description is \emph{instantiated},
i.e., the rule is applied.

Thus, given a~concrete set of triples, such as:
\begin{Verbatim}
    </photos/37> ex:smallThumbnail </photos/37/thumb>.
\end{Verbatim}
the description in \cref{lst:Thumbnail} would be instantiated to: 
\begin{Verbatim}
    _:request http:methodName "GET";
              http:requestURI </photos/37/thumb>;
              http:resp [ http:body </photos/37/thumb> ].

    </photos/37> dbpedia-owl:thumbnail </photos/37/thumb>.
    </photos/37/thumb> a dbpedia:Image;
                       dbpedia-owl:height 80.0.
\end{Verbatim}
Thereby, the description has been instantiated into a~concrete \http request
that can be executed by the agent.
Note how this instantiation directly results in RDF triples,
which can be interpreted by any RDF-compatible client.
The request has been sufficiently specified
as defined in Definition~\ref{def:HttpRequestDescription}.
In addition, the instantiated postcondition
explains the properties realized by this concrete request.
Here, an \http \verb!GET! request to \url{/photos/37/thumb}
will result in a~thumbnail of the image \url{/photos/37}
that will have a~height of 80~pixels.
This dynamic interpretation is helpful to agents
that want to understand the impact
of performing a~certain action on resources they have at their disposition.

\begin{lstlisting}[
  float=b,
  caption={\restdesc description of the action ``uploading an image'' \emph{(desc\_images.n3)}},
  label=lst:Images]
§\textcolor{gray}{@prefix dbpedia: <http://dbpedia.org/resource/>.}§
§\textcolor{gray}{@prefix dbpedia-owl: <http://dbpedia.org/ontology/>.}§
§\textcolor{gray}{@prefix ex: <http://example.org/image\#>.}§
§\textcolor{gray}{@prefix http: <http://www.w3.org/2011/http\#>.}§

{ ?image a dbpedia:Image. }
=>
{
  _:request http:methodName "POST";
            http:requestURI "/images/";
            http:body ?image;
            http:resp [ http:body ?image ].
  ?image ex:comments _:comments;
         ex:smallThumbnail _:thumb.
}
\end{lstlisting}

\restdesc descriptions are not limited to \verb!GET! requests.
They can also describe state-changing operations,
for instance, those realized through the \verb!POST! method.
\Cref{lst:Images} shows a~description for an image upload action.
The postconditions contain existential variables that are not referenced
(\verb!_:comments! and \verb!_:thumb!),
which might appear strange at first sight.
However, these triples are important to an agent
as they convey an \emph{expectation} of what happens when an image is uploaded.
Concretely, any uploaded image will receive a~\verb!comments! link
and a~\verb!smallThumbnail! link.
Even though the exact values will only be known at runtime
when the actual \verb!POST! request is executed,
at design-time, we are able to determine that there will be several links.
The meaning of those links is in turn expressed by other descriptions,
such as the one in \cref{lst:Thumbnail} discussed~above.
