\section{Conclusion}
\label{sec:Conclusion}
In this article,
we explained a~novel solution to automated composition and execution of hypermedia APIs.
A~crucial part in generating a~composition
is the ability to determine whether it will satisfy a~given goal
without any undesired effects.
This has led us to the approach of a~pragmatic proof,
wherein hypermedia API operations are incorporated as inference rules.
We distinguish between a~pre-execution proof and a~post-execution proof,
where the former has the additional assumption that all hypermedia API operations will succeed,
hence the ``pragmatic'' label of the method.

We selected an RDF-based method and logic for this task,
in order to bridge between existing Web technologies
and concepts from logic programming.
A~benefit of proof-based composition is that it does not require new algorithms and tools,
but can be applied with existing Semantic Web reasoners.
Those reasoners can easily incorporate external sources of knowledge
such as ontologies or business rules.
Furthermore, the performance of composition generation improves
with the evolution of those reasoners.
Also, the fact that a~third-party tool is used allows independent validation of the composition.

Our approach is a~special use case for proofs,
which have traditionally been regarded as a~part of trust on the Semantic Web.
While pre-proofs partly contribute to this,
they also have the added functionality of generating a~composition during that process.
It will be interesting to explore other opportunities
to exploit the power of proof creation
and the mechanisms behind it.
This application can serve as an example of how to apply such ideas.

In the past, we have already employed the method
in the domain of sensor APIs~\cite{verborgh_ssn_2012},
yet we want to extend the approach to other domains such as multimedia analysis and transcoding~\cite{verborgh_mtap_2013,vanlancker_mtap_2013}.
In the longterm, we aim at offering the composition method described in this article
as a~hypermedia API itself,
so it can be used for dynamic mash-up and composition~generation.

Another interesting path is to explore the limits of the used logic.
For instance, it would currently be impossible to express the deletion of resources,
even though this is a~common operation on the Web
and even has a~designated HTTP method \verb!DELETE!.
We are currently experimenting with capturing explicitly described states
inside RESTdesc descriptions to account for these situations.

%\todo{comment: we could mention that we are also monitoring the results of the ongoing research regarding Datalog$^\pm$ with great interest. 
%Depending on the result we would like to put certain restrictions on our RESTdesc rules to ensure decidability.}

A~crucial part of the proof-based method is that the interaction remains driven by hypermedia.
In contrast to traditional approaches,
where a~plan determines the full interaction,
the composition here serves as a~guideline to complete the interaction.
Until the moment machines are able to autonomously interpret
the meaning of following a~hyperlink---%
like we humans can---%
guiding them through a~hypermedia application with descriptions and proofs
can be the~pragmatic~alternative.

