\section{Conclusion}
\label{sec:Conclusion}
In this chapter we investigated how proofs produced by \nthree reasoners can be used in practice. 
% \nthree proofs are special in the sense that they can themselves be 
% expressed in the logic and serve as input for further reasoning which makes it possible to exchange, test or reuse them amongst different reasoning systems. The practial 
% applications we discussed used proofs as plans: 
The applications we discussed used proofs as plans: We describe possible API operations or possible sensor queries by means of existential rules where
the existentials 
represent the knowledge gained by performing the operation of executing the query. In the pre- and postconditions, this knowledge is set into context. We express 
under which circumstances we can execute operations and queries and what their outcome means for our set-up. By that, we connect the rules to the context knowledge 
such that plans produced with them adapt to the present situation. In combination with a goal, we use descriptions, background knowledge and rules to generate 
proofs. 
As every description which is relevant for the goal appears in such a proof, we can understand the proof as a plan and use the information about specific 
query parameters or 
call methods as instructions to execute these plans. By constantly updating such plans we can always take all necessary information into account and adapt in case 
the situation changes.

The use cases we described here were only two possible applications of \nthree proofs: The fact that \nthree reasoners produce proofs which are themselves written in \nthree 
provides many opportunities: Proofs can be used to generate explanations together with the nurse assignments we discussed in Section~\ref{orca} 
(the author of this thesis was involved in the development of the so-called \emph{Why-Me?-button} which exactly does that and was submitted for patent application 
\cite{patent_15202196-0-1955}) or they can help to find the causes of constraint violations as discussed in Section~\ref{usecase2} (this is further explained in \cite{ben}).
There are many set-ups in in which \nthree proofs can be beneficial.

Coming back to our initial discussion in Chapter \ref{unilog} we can conclude that if we realise the \emph{Proof layer} of the Semantic Web stack by using \nthree,
we have a clear and strong connection to the highest layer of the stack, the layer of \emph{Applications}. 
As this is the most important layer which ultimately needs to be supported 
by all levels if we want to realise the vision of the Semantic Web, using \nthreelogic can bring us one step closer to fully realising the Semantic Web.