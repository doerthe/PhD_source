\section{Conclusion}
In order to answer the question whether \nthreelogic can solve the same practical problems as SPARQL and OWL DL reasoning with a comparable performance we considered two use cases:
A semantic nurse call system and a system for \rdf validation.  

The nurse call system has been previously implemented using OWL DL reasoning to draw conclusions from the ontology involved
and SPARQL querying to select a nurse based on the reasoning 
result. 
We implemented a system which combined ontology reasoning and querying using only one framework: \nthreelogic. To draw conclusions from 
the knowledge expressed in the ontology, we applied OWL RL reasoning. 
% While this was enough for our use case, such rules 
% making sense of the concepts used in OWL ontologies could be further extended if needed. 
The decision tree
for assigning nurses to a call depending on the current context has been implemented by directly using \nthree rules. 
We compared the execution times of both implementations and saw that the solution using \nthree performed faster which was caused by two reasons:
Firstly, we only took the \owl concepts which were relevant for our use case into account and not all inference steps included in the profile OWL DL. 
This optimisation could be made because \nthreelogic is very flexible: using rules we can choose which OWL concepts we want to support. 
Since \nthree also supports existential rules and is in most reasoners enriched with various built-ins, we can go beyond OWL RL if needed 
(see also Section~\ref{ontres} below).
The second reason for the good performance was that two different tasks -- querying and ontology reasoning -- could be performed by one single framework and
not by two 
different ones as in the original implementation. Here, \nthree completes the task of a \emph{unifying logic} and connected the building blocks \emph{Querying} and 
\emph{Ontologies/Taxonomies}.

For the second use case  we considered -- \rdf data validation -- SPARQL is normally seen as the method of choice~\cite[p. 143]{hartmann2016}. By implementing 
rules for the constraint types supported by RDFUnit -- a very popular engine to test constraints on \rdf datasets -- 
we showed that this typical use case for querying 
can also be supported by \nthreelogic. The comparison of our implementation with RDFUnit showed that both implementations are comparable in terms of accuracy and 
execution times. For small datasets our implementation was faster, for datasets containing more than 100,000 triples RDFUnit performed better.
The real advantage of using \nthree reasoning instead of SPARQL querying was in this case again the fact that our system can be easily extended to perform more complex 
reasoning.

Both use cases we considered were originally approached using OWL DL reasoning and/or SPARQL querying and we could retrieve 
comparable -- and in some aspects even better -- results by applying 
\nthreelogic instead. This indicates that \nthreelogic -- if equipped with the right set of built-in predicates -- can connect the layers \emph{Querying}
and \emph{Ontologies/Taxonomies} for the Semantic Web stack just as we expect it from the \emph{Unifying Logic}.

% Getting back to our initial question which was whether or not \nthree can fulfil the role of the \emph{Unifying Logic} of the Semantic Web, we have so far covered three of the four main 
% requirements as discussed in Section~\ref{req}:  We discussed the semantics of \nthree, its compatibility to \rdf and its ability to connect the logical building blocks. Next, we 
% focus on the layer of proofs. This is the topic of the following two chapters.