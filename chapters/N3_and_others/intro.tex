In the previous chapters we discussed \nthreelogic, its relation to \rdf and the possibilities to fix its formal semantics. % which is not always clear but can -- once the community comes to an agreement -- be formally specified. 
A clearly defined semantics and compatibility with \rdf were only two requirements on a \emph{Unifying Logic} we identified in Section~\ref{req}.
% After this examination of the logic as a system in itself we now 
% look at the context it was designed for, the Semantic Web. We discussed in Section \ref{req} that 
A \emph{Unifying Logic} for the Semantic Web should furthermore
connect the logical building blocks, \ie the blocks of \emph{Querying}, \emph{Ontologies/Taxonomies} and \emph{Rules}, and it should support the layer of \emph{Proofs}. 
The last point -- proofs -- will be discussed in the following chapter. Here, we focus on \nthree's relation to the logical blocks:
Being a rule logic itself \nthreelogic covers the \emph{Rules} block -- we can perform rule-based reasoning with \nthree.
For \emph{Querying} and \emph{Ontologies/Taxonomies} we need to investigate further: To what extend does \nthreelogic support these two blocks? 
Can it solve the same practical problems as the standards SPARQL and OWL-DL listed in the Semantic Web stack? If yes, is the performance comparable?

%to step back and have a look at the bigger picture: Which tasks can we perform by applying Notation3 Logic?
%Or, coming back to our discussion
%  come back 
% to our initial question
% at this logic of \nthreelogic as a possible candidate to become the unifying logic of the Semantic Web:
% Can we use \nthree to perform the tasks of querying and ontology reasoning in the same use cases SPARQL and OWL are typically used for? 
% How is the performance of the \nthree-based solution in these cases?
In order to answer these questions, we consider two practical use cases which both can be solved by applying a combination of ontology reasoning and querying:\begin{enumerate}
\item a semantic nurse-call system, and
\item a system to perform \rdf validation
\end{enumerate}
Both use cases have been previously implemented by either using OWL-DL or \rdf{}S reasoning and SPARQL querying and we aim to solve them by applying \nthree reasoning instead.
Below, we describe the use cases in detail and discuss how \nthree can 
be used to perform the same tasks as the existing implementations. Both sections finish with a comparison between the approaches. We then conclude our findings.
