
 \chapter{Problems in Cwm}\label{bugsincwm}
 \begin{lstlisting}[
  float=t,
  caption={Output of Cwm when provided with Formula~\ref{cwmbug1}. },
  label=cwmbug1int]
§\textcolor{gray}{@prefix : <http://example.org/ex\#>}§
§\textcolor{gray}{@prefix c: <\#>}§
   
@forAll c:x .
:a :b {
  :c :d {:e :f {c:x :g :h.}.}.
}.
{c:x :p :o.} => {c:x :q :o.}.
\end{lstlisting}
In this section we explain where the Cwm's interpretations of \nthree formulas as specified in this paper differ
from the actual implementation. %and the \wwwc team submission of \nthree~\cite{Notation3}. 
We start with a simple example. Consider the formula:  
%
\begin{multline}\label{cwmbug1} 
\texttt{\{?x :p :o\}=>\{?x :q :o\}.}\\
\texttt{:a :b \{:c :d \{:e :f \{?x :g :h\}\}\}.}
\end{multline}
According to the formalisation provided in this paper, the occurrences of the variable \texttt{?x} in the implication  are universally quantified on its
$\text{parent}_c$ formula which is the overall formula.
This quantification then also counts for the third occurrence of \texttt{?x} which is more deeply nested. In core logic this formula can be represented as:
\begin{multline}\tag{\ref{cwmbug1}'}\label{cwmbugint}
 \forall \texttt{x. <x p o>}\rightarrow\texttt{<x q o>.}\\
 \texttt{<a b <c d <e f <x g h>{}>{}>.}
\end{multline}
In this case this is also the result Cwm gives. In Listing~\ref{cwmbug1int} we display the reasoning result of Cwm\footnote{Version: v 1.197 2007/12/13 15:38:39 syosi} when provided with Formula~\ref{cwmbug1}. The output is produced by using the option 
\texttt{-{}-think} which makes the reasoner derive the deductive closure of its input dataset. During this process Cwm also translates all implicit universal quantification into its explicit counterpart.
But if we now change the order of the conjunction to
\begin{multline}\label{cwmbug2}
\texttt{:a :b \{:c :d \{:e :f \{?x :g :h\}\}\}.}\\
\texttt{\{?x :p :o\}=>\{?x :q :o\}.}
\end{multline}
%
we get a different result.
%\rv{Okay, but this is a bug, right? In that case, I'd remove the appendix.}
According to the formalisation this formula has the same meaning as the previous one namely Interpretation~\ref{cwmbugint}. 
In Listing~\ref{cwmbug2int} we display the reasoning output of Cwm for this formula. We clearly see that in this interpretation an extra universal quantifier for the
deeply nested occurrence of \texttt{?x} is added (line 7). For Cwm the interpretation of the conjunction depends on the order of its conjuncts. 
 Cwm does its translation from implicit to explicit quantification while parsing. 
Whenever a new universal variable is encountered which is not quantified yet a universal quantifier is added on its $\text{parent}_c$ level. 
This mechanism does not take the later encounter of universal variables on higher 
levels into account. Such a mechanism would have a negative impact on the performance of Cwm. Similar examples can be easily constructed.
\begin{lstlisting}[
  float=t,
  caption={Output of Cwm when provided with Formula~\ref{cwmbug2}. },
  label=cwmbug2int]
§\textcolor{gray}{@prefix : <http://example.org/ex\#>}§
§\textcolor{gray}{@prefix c: <\#>}§
   
@forAll c:x .
:a :b {
  :c :d {
    @forAll c:x. 
    :e :f {c:x :g :h.}.
  }.
}.
{c:x :p :o.} => {c:x :q :o.}.
\end{lstlisting}
%
% \todo{Mention somewhere that there is no difference between operators and predicates?}
% \section{Reasoning Output}\label{ap1}
% In this appendix we give evidence for the different interpretations of the formulas displayed in Sections \ref{ovw} and \ref{n3} by the reasoners EYE\footnote{Version: v16.1027.1037 beta}
% and cwm\footnote{Version: v 1.197 2007/12/13 15:38:39 syosi}. 
% We limit our explanations to the cases which are not covered by our previous publication~\cite{arndt_ruleml_2015}. 
% This is in particular the scoping of existential quantification in EYE which changed since then.
% 
% In Section \ref{ovw} we talk about the meaning of formula \ref{eq1} 
% \[
% \verb!_:x :says {_:x :knows :Albert.}.!
%  \]
% 
% which is for EYE:
% \[\exists x : \text{says}(x, \text{knows}(x, \text{Albert})) \tag{\ref{eq1}a} \]
% \textit{``There exists someone who says about himself that he knows Albert.''}
% 
% 
% and for cwm:
% %               
% \[
% \exists x_1 : \text{says}(x_1, (\exists x_2: \text{knows}(x_2, \text{Albert})))\tag{\ref{eq1}b} \]
% \textit{``There exists someone who says that there exists someone (possibly someone else) who knows Albert.''}
% 
% Both reasoners, EYE and cwm, offer the option to output the deductive closure of the formulas they were given. 
% The reasoning output of cwm for Formula \ref{eq1} is displayed in Listing \ref{cwm1}. 
% We see that both existentials are expressed by different square brackets which always stand for new variables, thus, cwm supports interpretation \ref{eq1}b.
% \begin{lstlisting}[
%   float=t,
%   caption={Reasoning result of cwm for Formula \ref{eq1} },
%   label=cwm1]
% §\textcolor{gray}{@prefix : <http://example.org/ex\#>.}§
% 
% [ :says { [ :knows :Albert ]. } ].
% \end{lstlisting}
% 
% 
% To test how EYE interprets that formula, we need to add an additional rule:
% \begin{multline}\label{eq111}
%  \texttt{\{?x :says \{?x :knows :Albert.\}\} =>} \\ \texttt{ \{:s :p :o.\}.}
% \end{multline}
% This rule is interpreted as: 
% \[
%  \forall x: \text{says}(x, \text{knows}(x,Albert))\rightarrow p(s,o)
% \]
% This means that the rule should only lead to the result 
% \[\verb!:s :p :o.!\] 
% if interpretation \ref{eq1}a is implemented, but not in case of interpretation \ref{eq1}b. 
% Listing \ref{eye1} displays the output of asking EYE for the deductive closure of Formula \ref{eq1} and \ref{eq111}, and indeed, the triple appears in the output.
% Thus, EYE supports interpretation \ref{eq1}a.
% %Running the reasoners with that option lead to the results displayed in Listings \ref{cwm1} and \ref{eye1}.
% 
% 
% 
% 
% \begin{lstlisting}[
%   float=t,
%   caption={Reasoning result of eye for Formula \ref{eq1} and Formula \ref{eq111} },
%   label=eye1]
% §\textcolor{gray}{PREFIX : <http://example.org/ex\#>}§
% 
% _:e_x_1 :says 
%            {_:e_x_1 :knows :Albert}.
% :s :p :o.
% 
% {?U_0 :says {?U_0 :knows :Albert}} 
%  => {:s :p :o}.
% \end{lstlisting}
% 
% % If we give thi\ref{eq1} is given to the reasoners cwm and EYE, we get the results shown in Listings \ref{cwm1} and \ref{eye1} (Lines 1-4) respectively.
% % Cwm again uses the \verb![]!-notation, 
% % the occurrences of ``\verb!_:x!'' are translated into two different new blank nodes; cwm chooses~\ref{zw}.
% 
% % If we now take a closer look into the output of EYE (Listing~\ref{eye1}, Lines 1--4) we see that there, both 
% % occurrences of \verb!_:x! get translated into the same variable \verb!_:e_x_1!. As the EYE reasoner gives different names to variables it interprets differently, this output 
% % is already an indication that EYE prefers the first interpretation (\ref{zwei}). 
% % To better illustrate that this is actually the case we ran the EYE reasoner again with the additional formula 
% % 
% % Together with EYE's interpretation of universally quantified variables as explained in the next subsection, this should lead to the result \[\verb!:s :p :o.!\] if EYE actually 
% % treats both variables as the same. The result of the reasoning with Formulas \ref{eq1} and \ref{eq111} is shown in Listing \ref{eye1}, Lines 1-5. 
% % We see that EYE indeed chooses the 
% % second interpretation \ref{zw}\footnote{Note that this is different than it used to be when our previous paper \cite{arndt_ruleml_2015} was written,
% % the scoping of EYE recently changed.}.
% % 
% % The reasons for this difference are manifold and have for example been influenced by the handling of blank nodes in TriG \cite{TriG}. In the \wwwc team submission, the topic is
% % handled by an informal text.
% % 
% % 
% % there is for example the addition of named graphs (TriG \cite{TriG}) to \rdf and the fact that identically labeled 
% % blank nodes in different named graphs are understood as the same instances. On the other hand there is the \wwwc team submission \cite{Notation3} whose informal text
% % seems to favor interpretation \ref{zw}.
% 
% % For both interpretations there are good reasons. The \wwwc team submission \cite{Notation3} 
% % states that 
% % in case of nested formulas blank nodes are  ``\textit{used to only identify blank node in the formula it occurs directly in}'', which supports cwm's point of view. 
% % On the other hand the inclusion of TriG \cite{TriG} into \rdf makes is also desirable to be in line with that specification and there it is states that 
% % ``\textit{Blank\-Nodes sharing the same label in differently labeled graph statements are considered to be the same BlankNode}'', which corresponds 
% % with the position EYE is taking.
% 
% 
% 
% 
% %This example is particularly interesting, because interpretation \ref{zwei} is the interpretation the EYE reasoner chooses while \ref{zw} is how cwm understands the formula. 
% 
% 
% 
% 
% 
% 
% 
% 
% %Consider the following implication:
% %If we take a look into implicitly existentially quantified variables, we find differences in the interpretations. %As \nthreelogic allows nesting for cited formulas and rules
% %Implicit existential quantification gets ambiguous 
% %Whenever there is some kind of nesting involved, in particular when it is used in cited formulas and rules, formulas can be ambiguous. 
% %To start with a less controversial example, consider the following rule:
% %Another case where the scope of implicit quantification is not intuitively clear is the case of existential variables occurring in rules. The formula
% \begin{multline}\label{exrule}
%  \texttt{ \{\_:x :knows :Kurt\} =>}\\ \texttt{ \{\_:x :knows :Albert\}.}
% \end{multline}
% This formula could be understood as:
% %\small
% \begin{multline}
%  (\exists x_1:\text{knows}(x_1, \text{Albert}))\rightarrow \\ (\exists x_2:\text{knows}(x_2, \text{Kurt}))\tag{\ref{exrule}a}\label{exra}
% \end{multline}
% \textit{``If there exists someone who knows Albert, then there also exists someone (probably different) who knows Kurt.''}
% \normalsize
% \begin{center}or\end{center}
% \[
%  \exists x: (\text{knows}(x, \text{Albert}))\rightarrow (\text{knows}(x, \text{Kurt}))\tag{\ref{exrule}b}\label{exrb}
% \]
% \textit{``There exists someone and if this someone knows Albert, then he also knows Kurt.''}
% 
% By giving this formula to an \nthree reasoner, 
% 
% To understand how this formula is interpreted by the reasoners cwm and EYE\footnote{\todo{add versions}}, we invoked them with Formula \ref{exrule} and the additional fact:
% \begin{equation}\label{help}
% \verb! :Kurt :knows :Albert.!
% \end{equation}
% % We added the last formula because in case of interpretation~\ref{exra}, this formula would cause the production of a new triple 
% % \begin{equation}
% %  \verb! [ :knows :Kurt.]!
% % \end{equation}
% %
% Both reasoners offer the option
% to display all conclusions they can draw from a given set of formulas, the output of this process is presented 
% in Listings~\ref{cwm2} and~\ref{eye2}. 
% To understand the reasoning result of cwm (Listing \ref{cwm1}), the reader has to know that in \nthree the symbol ``\texttt{[ ]}'' represents a new ``fresh'' blank node.
% We see in Lines~5--6 in both listings, that the variables are actually understood as being different, the derivations of the triples in Line~4 furthermore show 
% that both reasoners also put the quantifiers at the same position as in Interpretation~\ref{exrb}.
% 
% 
% \begin{lstlisting}[
%   float=t,
%   caption={Reasoning result of cwm for Formulas \ref{exrule} and \ref{help}.},
%   label=cwm2]
% §\textcolor{gray}{ @prefix : <http://example.org/ex\#> .}§
% 
% :Albert :knows :Kurt .
% [ :knows :Albert ].   
% { [ :knows :Kurt ].} => 
%           { [ :knows :Albert ].}.
% \end{lstlisting}
% 
% 
% \begin{lstlisting}[
%   float=t,
%   caption={Reasoning result of EYE for Formula \ref{exrule} and \ref{help}. },
%   label=eye2]
% §\textcolor{gray}{PREFIX : <http://example.org/ex\#>}§
% 
% :Albert :knows :Kurt.
% _:sk_0 :knows :Albert.
% {?U_1 :knows :Kurt} => 
%             {_:sk_0 :knows :Albert}.
% \end{lstlisting}
% 
% 
% 
% Both formulas we examined in this subsection use a \texttt{\{~\}}-construction to refer formulas. %so one might wonder why EYE makes a difference between the two constructions.
% Cwm does not make a difference between the types of occurrences and follows the \wwwc team submission:
% 
% %That this is the case can be shown if we additionally run the reasoner with the rule
% \MyQuote{
% ``When formulae are nested, \_: blank nodes syntax \emph{[is]} used 
% to only identify blank node in the formula it occurs directly in. 
% It is an arbitrary temporary name for a symbol which is existentially quantified within the current 
% formula (not the whole file). They can only be used within a single formula, and not within nested formulae.''
% }
% 
% But what exactly does EYE do? %If we take a closer look into the specification of EYE, we can read on there Web-page\footnote{\url{http://eulersharp.sourceforge.net/README}}:
% As stated on the EYE website\footnote{\url{http://eulersharp.sourceforge.net/README}}, EYE differentiates between different literals. In an Implication if the form $L_1\verb!=>!L_2$,
% both $L_1$ and $L_2$ are understood as different literals, every statement which does not occur in a rule is understood as being part of the remaining literal of formulas. 
% The scope of an existential is then this literal. So in the above examples, the difference is that the existentials in Formula~\ref{exrule} occurred in premise and conclusion
% of a rules---and thereby in different literals---while in Formula~\ref{eq1} both variables were part of the same literal, the remaining literal.
% 

\chapter{Attributes and Methods for Evaluation}
For the evaluation in Section~\ref{cases} we defined several attributes. In this section, we display the definition of these attributes
and explain how they have been used for the categorisation
of different cases. 
\section{Critical Built-in Constructs}\label{cbi}
\begin{figure*}
\begin{minipage}{\textwidth}
\begin{center}\scriptsize
\begin{tabular}{llll}
\hline
\multicolumn{2}{l}{production rule}& rules for $b$ & rules for $bi$ \\
  \hline
%Syntax: &&\\
%&&\\
\texttt{s ::=}&\texttt{f}& $\texttt{s}.b\leftarrow \emptyset$& $\texttt{s}.\bi \leftarrow \texttt{f}.\bi$\\
       &&\\
\texttt{f ::=}&  $ \texttt{t}_1 \texttt{t}_2 \texttt{t}_3.$
                  &   $\texttt{f}.\biu \leftarrow f_1( \texttt{t}_2.m_c,  (\texttt{t}_1.u  \cup \texttt{t}_3.u),$%, \bigcup_{i=1}^3 \texttt{t}_i.\biu)$
                  %(\texttt{t}_1.\biu \cup \texttt{t}_2.\biu   \cup \texttt{t}_3.\biu))$
       &   $ \texttt{f}.\bi \leftarrow \max(\texttt{t}_1.\bi, \texttt{t}_2.\bi, \texttt{t}_3.\bi)$                                                       \\
       &&\hspace{2cm}$(\texttt{t}_1.\biu \cup \texttt{t}_2.\biu   \cup \texttt{t}_3.\biu))$&\\
    &  $\texttt{e}_1 \texttt{=>}  \texttt{e}_2.$ & $\texttt{f}.\biu \leftarrow \texttt{e}_1.\biu \cup  \texttt{e}_2.\biu$& $\texttt{f}.\bi \leftarrow \max(\texttt{e}_1.\bi,  \texttt{e}_2.\bi)$\\
%    &  \texttt{@forAll :u}     & universal quantification\\
%    &  \texttt{@forSome :u}     & existential quantification\\
    & $ \texttt{f}_1 \texttt{f}_2$ &                $\texttt{f}.\biu \leftarrow \texttt{f}_1.\biu\cup\texttt{f}_2.\biu$ &                $\texttt{f}.\bi \leftarrow \max(\texttt{f}_1.\bi,\texttt{f}_2.\bi)$ \\
&&\\
\texttt{t ::=}& \texttt{uv} 
&                $\texttt{t}.\biu \leftarrow \emptyset $&                $\texttt{t}.\bi \leftarrow 0$ \\
            & \texttt{ex}&               $\texttt{t}.\biu \leftarrow \emptyset$ &               $\texttt{t}.\bi \leftarrow 0$\\
      & \texttt{c}&               $\texttt{t}.\biu \leftarrow \emptyset$ &               $\texttt{t}.\bi \leftarrow 0$\\
 %     & \texttt{l} &                literals\\
      & \texttt{e}&                $\texttt{t}.\biu \leftarrow\texttt{e}.\biu $ &                $\texttt{t}.\bi \leftarrow\texttt{e}.\bi $ \\
      & \texttt{(k)}& $\texttt{t}.\biu \leftarrow\texttt{k}.\biu$& $\texttt{t}.\bi \leftarrow\texttt{k}.\bi$\\
      & \texttt{()}& $\texttt{t}.\biu \leftarrow \emptyset$& $\texttt{t}.\bi \leftarrow 0$\\
      &&\\
\texttt{k ::=}& \texttt{t}& $\texttt{k}.\biu \leftarrow\texttt{t}.\biu$& $\texttt{k}.\bi \leftarrow\texttt{t}.\bi$ \\
&$\texttt{t k}_1$& $\texttt{k}.\biu \leftarrow \texttt{t}.\biu \cup \texttt{k}_1.\biu$ & $\texttt{k}.\bi \leftarrow \max(\texttt{t}.\bi, \texttt{k}_1.\bi)$ \\
&&\\
\texttt{e ::=}&\texttt{\{f\}}&                $\texttt{e}.\biu \leftarrow \emptyset$ &                $\texttt{e}.\bi \leftarrow f_2 (\texttt{f}.\bi ,(\texttt{f}.\biu \setminus \texttt{e}.s))$\\
       &\texttt{\{\}} &  $\texttt{e}.\biu \leftarrow \emptyset$ &  $\texttt{e}.\bi \leftarrow 0$\\
       &\texttt{false}      &                $\texttt{e}.\biu \leftarrow \emptyset$ &                $\texttt{e}.\bi \leftarrow 0$\\
  \hline
\end{tabular}\normalsize
\caption{Attribute rules for the synthesized attributes $\biu$ (middle) and $\bi$ (right), and their corresponding production rules (left) 
from the \nthree grammar (Figure~\ref{N3S}).
The attributes test whether a formula contains built-ins whose subject or object has universals which are not quantified on the parent or any higher level of the built-in.
\label{eval}}
\end{center}
\end{minipage}
\end{figure*}
To identify the built-in constructs in our datasets which are causing conflicting interpretations 
we extend the attribute grammar by the two synthesized attributes $b$ and $bi$. We display the rules for these attributes in Figure~\ref{eval}.

Attribute $b$ is used to pass the set of universal variables occurring in the subject or object position of a built-in predicate upwards in the syntax tree to the 
$\text{parent}_c$ formula of that built-in. Here, it is most interesting how the attribute behaves for the production rule of a simple triple, 
\ie the rule $\texttt{f := t}_1 \texttt{t}_2 \texttt{t}_3$. In that case we make use of the attributes $m_c$ and $u$ which have been defined in Section~\ref{unicwm} 
and apply function $f_1: \mathcal{T}\times 2^{U}\times 2^{U}\rightarrow 2^{U}$  which 
is defined as follows: 
%
%To collect these variables we make use of attribute $u$ defined in Section~\ref{unicwm}.
%For $f_1: \mathcal{T}\times 2^{U}\times 2^{U}\rightarrow 2^{U}$ we have the following definition:
\[
 f_1(t, s_1, s_2) :=\begin{cases} s_2 & \texttt{if } t \text{ is no built-in symbol} \\ s_1 & \text{else} \end{cases} 
\]
The first argument of the function is the actual value of the predicate which can either be a built-in or not. In the case it is not a built-in, 
the function simply passes the information collected by the attribute so far upwards. If the predicate is a built-in, the values of the attribute $u$ for
subject and object get 
collected, \ie all universal variables which occur in these two positions. For the other production rules the attribute value is either the empty set 
-- in case we have a rule resulting in a terminal node of the tree -- or it gets passed upwards through the tree. The only exception is the production rule
\texttt{e ::= \{f\}} for which the attribute value of \texttt{e} is the empty set since \texttt{f} is a $\text{parent}_c$ formula. For this formula the attribute 
value $\texttt{e}.s$ with $s$ defined as in Section~\ref{unicwm} gives information which variables are already quantified on a higher level. This is used 
in the definition of attribute $bi$.

Attribute $bi$ carries the information whether or not a built-in predicate has been found whose subject or object contains one or more universal variable which Cwm's 
interpretation quantifies either on the same level the built-in function occurs or any lower level. If this is the case, the value of the attribute is 1, else it is 0.
If we look at Figure~\ref{eval} we see that the rules for this attribute mostly simply pass information upwards with one exception: For the production rule 
\texttt{e ::= \{f\}} we apply the function $f_2:  \mathbb{N}\times 2^{U\times U}\rightarrow \mathbb{N}$ which is defined as follows:
% 
% % which test for every built-in whether its subject or its object contains a universal 
% % variable which -- according to Cwm's interpretation -- is not 
% % quantified on the $\text{parent}_c$ level of the built in or any higher level. 
% % These new synthesized attributes 
% % make use of the attributes $u$ and $s$ to get the scoping information. % their full definition can be found in~\ref{cbi}.
% which are used to determine whether a formula contains built-ins whose subject or object contains universal variables which according to Cwm's 
% interpretation
% are quantified either on the same level 
% the built-in occurs (if the built-in does not occur on top-level) or on a deeper level. Our attributes depend on the attributes $s$ and $u$ defined in Section~\ref{unicwm}. 
% We display the attribute rules in Figure~\ref{eval}. The attribute $\bi$ collects the information whether a built-in with nested universals has been found in a subformula (in this case its value is 1 otherwise 0).
% The attribute $\biu$ collects all universal variables which are found in the subject or object of a built-in function and passes them upwards to the next formula expression.
% 
% The function $f_1:  \mathbb{N}\times 2^{U\times U}\rightarrow \mathbb{N}$ is defined as follows:
\[
 f_2 (n, s) := \begin{cases} 0 &\text{if } n=0 \text{ and } s=\emptyset\\ 1 & \text{else}\end{cases} 
\]
To better understand this definition, we take a closer look to the arguments the function is used with in the attribute definition: The first argument is simply used 
to pass the information that a problematic built-in construct has been found upwards in the syntax tree. So, if its value is 1, the application of function 
$f_2$ should also result in 1 (second case). As the second argument, we have the difference of $\texttt{f}.b$ -- the variables we collected using attribute $b$ -- and $\texttt{e}.s$
-- the variables which are quantified on the $\text{parent}_c$ level of the formula.  If this difference is not empty, this means that we found a variable 
which occurs in the subject or object position of a built-in (collected by $b$) which Cwm quantifies either on the same level as the built-in or on a lower level. 
These are the constructs which are problematic in our consideration. In these cases function $f_2$ will result in 1, else its value is 0.

% This function thus tests in our application whether either an occurrence of a built-in together with a nested variable has already been found, in that case we have $n=1$ -- % 
% the information about that finding is just passed upwards -- %
% or whether the universal
% variables we have found so far in connection with a built-in are not scoped on a higher level. For the latter we compare the value of the attribute $\biu$ 
% which collects all universal variables found
% in the subject or predicate of a built-in with the set of all universal variables scoped on a higher level of the formula expression, collected by the attribute $s$.
% 
% For $f_2: \mathcal{T}\times 2^{U\times U}\times 2^{U\times U}\rightarrow 2^{U\times U}$ we have the following definition:
% \[
%  f_2(t, s_1, s_2) :=\begin{cases} s_2 & \texttt{if } t \text{ is no built-in symbol} \\ s_1\cup s_2 & \text{else} \end{cases} 
% \]
% Remember that the attribute $m_c$ is used to generate the translation of the \nthree formula to core logic according to Cwm. In the cases where we have a built-in as predicate
% the values of that attribute is just the name of the built-in. The function now acts differently depending on whether a built-in is found here or not. 
% In the latter case it only passes the nested variables 
% occurring with built-ins further upwards. In the former case, all variables which occurred in predicate or object position of the built-in are also passed upwards. 
% When the calculated set then reaches
% function $f_1$ it checks whether these variables are already scoped in the formula expression the built-in occurs in.


\section{Nested Universals}\label{nested} 
The reasons for conflicting interpretations of a formula 
can be overlapping. 
Especially for proofs we want to know whether conflicts are only 
caused by the proof vocabulary which normally uses graphs -- 
these cases are rather harmless -- or if either a built-in construct in combination with a formula expression containing a universal variable which is quantified within this expression
or some other occurrence of a universal variable quantified on level nested inside the result of a proof step is causing the disagreement of interpretations. 

For built-ins we use the attributes defined in \ref{cbi}.
%
To find deeply nested universals we introduce the concept of depth: we measure the depth of a universal variable as one plus
the number of formula expressions its quantifier is enclosed in when translating the \nthree formula 
to its core logic translation according
to Cwm. If a variable does not occur in a formula at all, it has depth 0. As an example consider Formula~\ref{fff} from Section \ref{unicwm}:
\begin{multline}\tag{\ref{fff}}
\texttt{\{\{?x :q ?y.\} => \{?x :r :c.\}.\} =>}
\texttt{\{?x :p :a.\}.}
\end{multline}
Here, variable \texttt{?z} has depth 0 -- it does not occur in the formula --  variables \texttt{?y} and \texttt{?x} have depths 2 and 1, respectively. 

In order to determine the depth of the deepest nested variable in a formula we define two attributes: the inherited attribute $c$ and the synthesized attribute $d$. The first attribute $c$
simply goes downwards in the syntax tree and counts every formula expression it encounters on the way down.  
To understand attribute $d$ we first take a closer look to the formula involved,
$f_3: 2^{U}\times\mathbb{N}\times\mathbb{N}\rightarrow \mathbb{N}$ is defined as follows:
\[
 f_3(s, n, m) := \begin{cases}
                  m &\text{if } s=\emptyset\\
                  \max(n,m) &\text{else}
                     \end{cases}
\]
Attribute $d$ now takes this formula to check for every formula expression and for the top formula whether
according to Cwm's translation there is at least one quantifier for universal variables  at that level (remember that the set of universals quantified at each level is captured by
attribute $q$). If so, it takes the depth of the current formula which is captured by attribute $c$ 
and compares it with the values for $d$ already found in the depending nodes. From these two values it takes the maximum. 
If there are no universal quantifiers at a node, the attribute only passes the maximum value for $d$ of the depending nodes upwards. The value $\texttt{s}.d$ is now the depth of the deepest 
nested universal in the formula.

\begin{figure}
%\begin{minipage}{0.95\textwidth}
\begin{center}%
\small
\begin{tabular}{llll}
\hline
\multicolumn{2}{l}{production rule} & rules for $d$ & rules for $c$\\
  \hline
%Syntax: &&\\
%&&\\
\texttt{s::=}&\texttt{f}& $\texttt{s}.d \leftarrow f_3(\texttt{f}.q, 1, \texttt{f}.d)$& $ \texttt{f}.c \leftarrow 1$\\
       &&\\
\texttt{f::=} &$\texttt{t}_1 \texttt{t}_2 \texttt{t}_3.$&   $ \texttt{f}.d \leftarrow \max_i \texttt{t}_i.d$ 
                                                             & $\texttt{t}_i.c \leftarrow \texttt{f}.c$\\
    &$\texttt{e}_1 \texttt{=>}  \texttt{e}_2.$& $\texttt{f}.d \leftarrow \max_i \texttt{e}_i.d$ & $\texttt{e}_i.c \leftarrow \texttt{f}.c$ \\
%    &  \texttt{@forAll :u}     & universal quantification\\
%    &  \texttt{@forSome :u}     & existential quantification\\
    &$\texttt{f}_1 \texttt{f}_2$ &                $\texttt{f}.d \leftarrow \max_i \texttt{f}_i.d$  &     $\texttt{f}_i.c \leftarrow \texttt{f}.c$\\
&&\\
\texttt{t::=}&\texttt{uv} %\hspace{1cm}
&                $\texttt{t}.d \leftarrow 0$ %\hspace{2cm} 
&              \\
            &\texttt{ex} &               $\texttt{t}.d \leftarrow 0$&              \\
      &\texttt{c} &               $\texttt{t}.d \leftarrow 0$&               \\
 %     & \texttt{l} &                literals\\
      &\texttt{e} &                $\texttt{t}.d \leftarrow\texttt{e}.d $ &   $\texttt{e}.c \leftarrow\texttt{t}.c$        \\
      &\texttt{(k)}& $\texttt{t}.d \leftarrow\texttt{k}.d$& $\texttt{k}.c \leftarrow\texttt{t}.c$\\
      &\texttt{()}& $\texttt{t}.d \leftarrow 0$& \\
      &&\\
\texttt{k::=}&\texttt{t}     & $\texttt{k}.d \leftarrow\texttt{t}.d$ & $\texttt{t}.c \leftarrow\texttt{k}.c$\\
&$\texttt{t k}_1$              & $\texttt{k}.d \leftarrow \max(\texttt{t}.d, \texttt{k}_1.d)$ & $\texttt{t}.c, \texttt{k}.c \leftarrow \texttt{k}.c$\\
&&\\
\texttt{e::=}&\texttt{\{f\}} &  $\texttt{e}.d \leftarrow f_3 (\texttt{f}.q ,\texttt{f}.c, \texttt{f}.d )$&   $\texttt{f}.c \leftarrow \texttt{e}.c+1$           \\
       &\texttt{\{\}}        &  $\texttt{e}.d \leftarrow 0$ & \\
       &\texttt{false}       &  $\texttt{e}.d \leftarrow 0$&   \\
  \hline
\end{tabular}
\normalsize
\caption{Attribute rules for the synthesized attribute $d$ (middle) and the inherited $c$ (right), and their corresponding production rules (left) 
from the \nthree grammar (Figure~\ref{N3S}).
The attributes keep track of the deepest nested universal in a formula.
\label{neval}}
\end{center}
%\end{minipage}
\end{figure}

Because of the nature of the proof we know that if the maximum depth of the universal quantifiers is two or lower, then the only reason for conflicting interpretations 
is the use of the 
predicate \texttt{r:gives} which has a graph as object. If the maximum depth of universal variables is bigger than two, 
we know that the conflict 
 found in the proof is more serious and caused by a deeper nesting in one of the formulas occurring in the object of \texttt{r:gives}.
 

% \begin{equation}\label{eq11}
% \verb!:Kurt :says {_:x :knows :Albert.}.!
%  \end{equation}
% 
% can without further explanation either stand for 
% \[
%  \exists x: \text{says}(\text{Kurt}, \text{knows}(x, \text{Albert}))\tag{\ref{eq11}a} \label{eq11a}
% \]
% \textit{
% `` There exists someone of whom Kurt says that he knows Albert.''
% }
% \begin{center}or\end{center}
% \[
% \text{says}(\text{Kurt}, (\exists x: \text{knows}(x, \text{Albert})))\tag{\ref{eq11}b} \label{eq11b}
% \]
% \textit{``Kurt says that there exists someone who knows Albert.''}

% \MyQuote{
% ``When formulae are nested, \_: blank nodes syntax \emph{[is]} used 
% to only identify blank node in the formula it occurs directly in. 
% It is an arbitrary temporary name for a symbol which is existentially quantified within the current 
% formula (not the whole file). They can only be used within a single formula, and not within nested formulae.''
% }








% 
% 
% Conclusion: For cwm and the team submission, everything in curly brackets is a new formula and every existentially quantified formula is on quantified in that particular formula.
% 
% EYE distinguishes between implications and cited formulas. In implications the 
% 
% \begin{lstlisting}[
%   float=t,
%   caption={Reasoning result of cwm for Formula \ref{eq22} },
%   label=cwm3]
% §\textcolor{gray}{@prefix   : <http://example.org/ex\#>.}§
% §\textcolor{gray}{@prefix ex: <\#> .}§
% 
% {@forAll ex:x.
%    {ex:x :p :a.} => {ex:x :q :b.}.}     
% => 
% {@forAll ex:x.
%   {ex:x :r :c.}=>{ex:x :s :d.}.}.
% \end{lstlisting}
% 
% \begin{lstlisting}[
%   float=t,
%   caption={Reasoning result of eye for Formula \ref{eq22} and Formula \ref{eq2222} },
%   label=eye3]
% §\textcolor{gray}{PREFIX : <http://example.org/ex\#>}§
% 
% {{?U_0 :p :a} => {?U_0 :q :b}} 
% => 
% {{?U_0 :r :c} => {?U_0 :s :d}}.
% 
% {:s :r :c} => {:s :s :d}.
% {:s :p :a} => {:s :q :b}.
% \end{lstlisting}
% 
% We give these formulas to both reasoners to learn how the formula is interpreted and get the results as shown in Listings~\ref{cwm3}, and \ref{eye3}. As the initial
% result of EYE was inconclusive, we added the following formula:
% \begin{equation}\label{eq2222}
%  \texttt{\{:s :p :a.\} => \{:s :q :b.\}.}
% \end{equation}
% 
% 
% 
% 
% 
% If the reasoner follows interpretation~\ref{ei} that formula would derive the rule
% \[
%  \verb!{:s :r :c} => {:s :s :d}.!
% \]
% And indeed we see in the output of the EYE reasoner (Listing~\ref{eye3}) that the rule was produced, EYE follows Interpretation~\ref{ei}. 
% The result of cwm (Listing~\ref{cwm3}) contains the \nthree construct \verb!@forAll! although that is not always the case (see Section~\ref{remarkExplicitQuantification}), 
% here it can be understood as its first order counter part. We see that
% cwm follows Interpretation~\ref{ci}.
% 
% In EYE, the quantification of universal variables is always put on top level, but how exactly is the rule for cwm? Here the \wwwc team submission gives us an indication:
% 
% \MyQuote{
%  ``Apart from the set of statements, a formula also has a set of URIs of symbols which are universally quantified, 
%  and a set of URIs of symbols which are existentially quanitified. Variables are then in general symbols which have been quantified. 
%  There is a also a shorthand syntax ?x which is the same as :x 
%  except that it implies that x is universally quantified not in the formula but in its parent formula.''}
% 
% So, if we find a new universal variable which is not quantified yet, we need to set a quantifier on its parent level. 
% Having this in mind, we assume that adding a simple dummy rule 
% like 
% \begin{equation}\label{dummy}
%  \texttt{\{?x ?x ?x.\}=>\{?x ?x ?x.\}.}
% \end{equation}
% would cause cwm to understand Formula~\ref{eq22} the same way EYE does. And this is indeed the behavior we can observe if we provide cwm with Formulas~\ref{dummy}, \ref{eq2222} and 
% \ref{eq22} as displayed in Listing~\ref{cwm33}. In this case the existence of another rule using the same quantifiers (here Rule~\ref{dummy}) changes the meaning of our initially
% examined rule (Rule~\ref{eq22}). \todo{In the actual implementation of cwm it 
% makes a difference whether I state my dummy rule before or after the actual rule. I assume that this is simply a bug in cwm. Or should I mention that?}
% 
% \begin{lstlisting}[
%   float=t,
%   caption={Reasoning result of cwm for Formulas \ref{dummy},\ref{eq2222} and 
% \ref{eq22} },
%   label=cwm33]
% §\textcolor{gray}{@prefix   : <http://example.org/ex\#>.}§
% §\textcolor{gray}{@prefix ex: <\#> .}§
% 
% @forAll ex:x .
%     
% {:s :p :a.} => {:s :q :b.}.
% {:s :r :c.} => {:s :s :d.}.
%     
% {ex:x ex:x ex:x.}=>{ex:x ex:x ex:x.}.
% 
% {{ex:x :p :a.} => {ex:x :q :b.}.}     
% => 
% {{ex:x :r :c.} => {ex:x :s :d.}.}.
% \end{lstlisting}
% 
% \todo{I think this section is already long, should I add an example of subordination as the one below?}
% \begin{equation}\label{nest}
%  \verb!{?x :p :o.} => {?x :pp {?x :ppp :ooo.}.}.!
% \end{equation}