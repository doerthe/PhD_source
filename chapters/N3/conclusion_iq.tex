\section{Conclusion}
In this chapter we discussed the concept of implicit quantification. \nthree assumes blank nodes to be quantified on their \emph{direct formula} and universals to be 
quantified on their \emph{parent formula}. While we could clearly identify the direct formula as the formula in brackets \texttt{\{ \}} surrounding a blank node,
the identification of the \emph{parent} was more difficult. Our tests showed that the reasoners EYE and Cwm understand this concept differently and that neither
the \wwwc team submission nor the journal paper introducing \nthree clarify this term. 
%We compared \nthree's implicit quantification with \rdf and other logical formats which have a similar construct.

In order to agree on \emph{one} interpretation we need to be able to express the difference and to understand its impact for practical use cases. We therefore 
define a core logic for \nthree which supports the same features as the original logic but only supports explicit quantification. Having such a logic at hand, we can map the syntax of 
\nthree to its different interpretations and compare the results. This will be the topic of the next chapter.
% 
% 
% In this chapter we have seen that the semantics of \nthree is not clear when it comes to implicit quantification: According to the \wwwc team submission, 
% implicitly universally quantified variables are quantified on their \emph{parent formula}. Unfortunately, it is not further defined what this \emph{parent formula} is and 
% our tests have shown that the reasoners Cwm and EYE understand this concept differently. 