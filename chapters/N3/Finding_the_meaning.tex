\section{Implicit quantification in practice}
%The examples of implicit quantification given in Section~\ref{n3examples} were rather  
The example formulas we discussed in Section~\ref{n3examples} were rather simple and understanding their intended meaning was not 
 difficult (maybe with the exception of cited formulas which can lead to discussions \cite{TriGsemantics}). For the only case where 
two different interpretations were plausible -- Formula~\ref{both} which had universals and existentials occurring together --
the \wwwc team submission~\cite{Notation3} contains a clear statement which interpretation needs to be chosen: \\
% This is different when it comes to implicit quantification. Just like \rdf, \nthree allows the 
% The examples of \nthree formulas given so far were rather easy and their interpretation was rather straightforward
% (maybe with the exception of cited formulas which can lead to discussions \cite{TriGsemantics}).
% This is different when it comes to implicit quantification. Just like \rdf, \nthree allows the usage of implicitly existentially quantified variables, called \emph{blank nodes}.
% Blank nodes either start with ``\verb!_:!'' or are expressed using square brackets ``\texttt{[~]}''. 

\MyQuote{ \label{beide} ``If both universal and existential quantification are specified for the same formula, 
then the scope of the universal quantification is outside the scope of the existentials''.}

Unfortunately, not all cases are that clear.
% 
% 
% in Section~\ref{n3examples}, we gave a few examples for \nthree formulas and their intended meanings. Since \nthree aims to extend \rdf, a \wwwc standard with well-defined 
% semantics, the meaning of simple  
%
% In the cases illustrated in Section~\ref{n3examples}, the interpretation of the implicitly quantified formulas was rather easy:
% the variables are existentially and universally quantified at the top of the formula; 
% if they co-occur in the same formula, the universal quantification dominates the existential. 
When implicitly quantified variables occur in deeply nested formulas, 
 their intended meaning is not always obvious and the interpretations of such formulas sometimes differ between reasoning engines. 
%
%In the next subsections we give examples for such ambiguous formulas.
In this section we want to better understand these differences. 
With this goal, we perform several tests on the reasoners Cwm \cite{cwm} and EYE~\cite{eyepaper} and compare their results. 
Cwm and EYE were chosen because they cover most constructs specified in \nthree{}'s \wwwc team submission. 
In contrast to for example FuXi~\cite{fuxi}, they both support rather complex constructs like nested rules. 
% and because they differ in their way to handle 
% implicit universal quantification.
% Further details about how differences can be detected can be found in our previous paper \cite{arndt_ruleml_2015}. % and in \ref{ap1}.



% 
% Before clarifying syntax and semantics of \nthree in a more formal way than presented in Section~\ref{n3examples}, 
% we test how implicit quantification of \nthreelogic is understood in practice. We  take a look to the official sources of \nthree, namely the \wwwc team submission~\cite{Notation3}
% and the journal paper about \nthree~\cite{N3Logic}, and test, how the reasoners Cwm~\cite{cwm} and EYE~\cite{eye} understand them. 
% We choose these two reasoners because of their coverage: while for example FuXi~\cite{fuxi} only supports \nthree datalog -- a subset of \nthree which does not support nested rules --
% the reasoners Cwm and EYE cover a big part of the specifications
% As in \rdf, atomic formulas are triples consisting of subject, predicate and object. They can be intuitively understood as 
% first order formulas of the form \linebreak $\text{predicate}(\text{subject}, \text{object})$. It is also easy to get an idea of the meaning of conjunctions or implications 
% if no variables are involved. Including implicit quantification is more difficult.
% Definition \ref{voc} distinguishes between two kinds of variables: universal and existential variables. 
% As the names indicate, these variables are meant to be implicitly quantified. But how do we have to understand this ``implicit'' quantification?
% Some cases are quite simple. 
% If we have the formulas
% \[
% \verb! _:x :knows :Kurt.! \text{ and } \verb! ?x :knows :Kurt.!
% \]
% It is rather straight forward to understand them as ``someone knows Kurt.'' and ``everyone knows Kurt.'' In first order logic:
% %We can understand that as the fact that Kurt knows someone or something, written in first order logic:
% \[
% \exists x: \text{knows}(x, \text{Kurt}) \text{ and } \forall x: \text{knows}(x, \text{Kurt}).
% \]
%Similarly simple constructions with universals can be understood:
%\[
%\verb! ?x :knows :Kurt.!
%\]
%Means that everyone knows Kurt, in first order:
%\[
%\forall x: \text{knows}(x, \text{Kurt}).
%\]
% But the above grammar also enables us to construct more complicate statements. Does the construct 
% \begin{equation}\label{both} \verb! ?x :loves _:y.!\end{equation} mean 
% ``everybody loves someone'' or ``there is someone
% who is loved by everyone'', in first order formulas: 
% %\[\forall x \exists y : \text{loves}(x,y) \tag{1}\quad \text{ vs. } \exists y \forall x : \text{loves}(x,y)\]
% \stepcounter{equation}
% \begin{equation} % oder auch align
% \forall x \exists y : \text{loves}(x,y)\quad  \text{ vs. }\quad \exists y \forall x : \text{loves}(x,y) \tag{\ref{both}a, b}
% \end{equation}
% 
% 
% In this case we know the answer, the team submission \cite{Notation3} clearly chooses (\ref{both}a): 
% 
% \MyQuote{ \label{beide} ``If both universal and existential quantification are specified for the same formula, 
% then the scope of the universal quantification is outside the scope of the existentials''.}
% %
% And also the reasoners we tested, in particular EYE and cwm, have implemented the first interpretation (\ref{both}a).
% %\\
% 
% Such clarity is lacking when it comes to nested formulas or co-occurring formula expressions which contain variables. 
% We will treat this in the following sections, first for existential variables, then for universals.
%This is the topic of the following sections.

\subsection{Existentials}
To test how both cwm and EYE understand existential quantification, we confronted them with some examples.
Both reasoners offer the option to output all knowledge they are aware of, this includes all derived formulas and rules as well as the
input. In most cases, different variables sharing the same name are renamed to be distinguishable. %standardized apart\footnote{Cwm does standardization apart for existentials, 
%using square brackets ``\texttt{[ ]}'' as introduced in Remark \ref{rem}, EYE employs Prolog's standardization apart.}. %, in EYE this process includes standardization apart\footnote{As EYE is written in Prolog, 
%Prolog's standardization apart mechanism is used see e.g. CITATION}.   
Therefore we can use the derived output of such a reasoning process with a simple rule as input as indication of
how the formula is interpreted. As a first example %for our tests we took 
we invoked both reasoners with
a formula with nested existentials:
\begin{equation}\label{eq1}
\verb!_:x :says {_:x :knows :Albert.}.!
 \end{equation}
Is there someone who says about himself that he knows Albert, or does this someone just state that someone exists who knows Albert?
In first order logic
\[\exists x : \text{says}(x, \text{knows}(x, \text{Albert})) \tag{\ref{eq1}a}\]
\hspace{6cm} or
\[\exists x_1 : \text{says}(x_1, (\exists x_2: \text{knows}(x_2, \text{Albert})))\tag{\ref{eq1}b} \label{zwei}\]
%
Listing \ref{third} shows the output of EYE given formula (\ref{eq1}) as only input, Listing \ref{cwm1} the output of cwm. 
We clearly see\footnote{To see this evidence 
for cwm, recall that every new bracket ``\texttt{[}$\ldots$\texttt{]}'' corresponds with a \emph{new} existential variable, see also Remark \ref{rem} or \cite{turtle} for further information. }
that both reasoners favor 
option (\ref{zwei}).  
\begin{lstlisting}[
  float=t,
  caption={Reasoning result of EYE for formula (\ref{eq1})},
  label=third]
§\textcolor{gray}{@prefix : <http://example.org/test\#>.}§

_:x_1 :says {_:x_2 :knows :Albert}..

\end{lstlisting}

\begin{lstlisting}[
  float=t,
  caption={Reasoning result of cwm for formula (\ref{eq1}) },
  label=cwm1]
§\textcolor{gray}{@prefix : <http://example.org/test\#>.}§

[ :says { [ :knows :Albert ]. } ].
\end{lstlisting}

%\FloatBarrier


We observe similar behavior using the same existential quantifier in two co-occurring graphs.
In an example formula such as 

\begin{equation} 
\verb!{ _:x :knows :Albert.} => { _:x :knows :Kurt.}.! 
\end{equation}
%So, for EYE, the scope of existential quantification is always only the direct formula expression the existential occurs in but 
%not it nested dependencies.
The two \verb!_:x! are interpreted as different variables by both reasoners. In first order logic this would be:
\[
 (\exists x_1: \text{knows}(x_1, \text{Albert}))\rightarrow (\exists x_2: \text{knows}(x_2, \text{Kurt}))
\]
%
%
%
This interpretation is also in line with the official team submission \cite{Notation3}:

\MyQuote{
``When formulae are nested, \_: blank nodes syntax \emph{[is]} used 
to only identify blank node in the formula it occurs directly in. 
It is an arbitrary temporary name for a symbol which is existentially quantified within the current 
formula (not the whole file). They can only be used within a single formula, and not within nested formulae.''
}
%
This means, the scope of an existential quantifier is always only the formula-expression ``\verb!{!$\ldots$\verb!}!'' it occurs in, but not its nested dependency.
%We added these two examples to be able to compare the interpretation of existential quantification with universal quantification.
%Although this section might not seem surprising, we included the above 
%We keep these examples for the scoping of existentials in the paper, to be able to emphasize the different scopes of existential and universal quantification.



\subsection{Universals}
When it comes to the definition of the scope, universal quantifiers are more complicated. To illustrate that, we consider the following example:
%Another example is the following formula:
%\\ %s and this is also how it is implemented in the reasoners we are considering in this paper, cwm and EYE. 
%It gets even more difficult when it comes to nested rules. As the critical example for this paper, we consider the formula 
\small
\begin{equation}\verb! {{?x :p :a.} => {?x :q :b.}.} => {{?x :r :c.} => {?x :s :d.}.}.! \label{eq2}\normalsize \end{equation}
\normalsize
Are all \verb!?x! the same? If not, which ones do we have to understand as equal? Two options seem to be most probable:
\[
(\forall x_1: p(x_1,a)\rightarrow q(x_1,b))\rightarrow (\forall x_2: r(x_2,c) \rightarrow s(x_2,d))\tag{\ref{eq2}a}
\]
\hspace{6cm}or
\[\forall x: ((p(x,a)\rightarrow q(x,b))\rightarrow ( r(x,c) \rightarrow s(x,d)))\tag{\ref{eq2}b}\]

As above, we gave formula (\ref{eq2}) as input for both reasoners, cwm and EYE. 

Lines 1-9 of Listing \ref{forth} show the result of EYE which seems to imply\footnote{Where applicable, 
EYE employs the ``standardization apart'' mechanism of its programming language Prolog.} that EYE supports the second interpretation (\ref{eq2}b), 
but as it does not differ from the input, 
we ran another test to verify that and added the formula 
\begin{equation}\label{eq4} \verb!{:e :p :a.} => {:e :q :b.}.! \end{equation}
to the reasoning input in order to see whether the reasoner outputs %(\ref{eq2}b) the reasoner's the result should contain the derived formula  
\begin{equation}\verb!{:e :r :c.} => {:e :s :d.}.!\end{equation}
as it would be the case with interpretation (\ref{eq2}b) but not with interpretation (\ref{eq2}a). % this formula cannot be derived. 
The reasoning output of EYE shown in Listing \ref{forth} (all lines)
verifies
that EYE interprets all variables with the same name which occur in one single implication equally regardless of how deeply nested they occur. %variables with the same name in all nested formulas  equally within one single implication.

In contrast to this, Listing \ref{cwm2} shows the result cwm gives.  Here, the keyword\linebreak %\footnote{ For further explanation of the 
%keyword \texttt{@forAll} see section \ref{expl}.} 
``\verb!@forAll!''
can be understood as its first order counterpart ``$\forall$'' (see Section \ref{expl}). %Here we see a clear difference between that cwm's interpretation of the input differs from EYE. 
Cwm understands 
formula (\ref{eq2}) as stated in interpretation (\ref{eq2}a). Here we see a clear difference between the two reasoners.


%\FloatBarrier


After examining universals in co-ordinated expressions such as in the above implication, 
we are also interested in how those variables are handled in subordinated formula expressions, similar to those in formula (\ref{eq1}). 
We consider the following formula: 
\begin{equation}\label{nest}
 \verb!{?x :p :o.} => {?x :pp {?x :ppp :ooo.}.}.!
\end{equation}
To learn how the reasoners interpret this formula, we give the simple formula \begin{equation}\label{spo}\verb!:s :p :o.! \end{equation} as additional input. 
Listings \ref{eye4} and \ref{cwm4} show the reasoning results of EYE respectively cwm. We clearly see that the two reasoners agree in their interpretation 
and that this interpretation of formula (\ref{nest}) differs from the interpretation of the existential counterpart formula (\ref{eq1}). 





%Thus, our formalization has to carefully distinguish the interpretation of nested universals and existentials 
%This particular difference
%This difference has to be respected in the formalization in section \ref{formal}.\\
Having considered the contrary behavior of the reasoners in the interpretation of formula (\ref{eq2}), the obvious question is: 
how is this interpretation meant to be according to the official sources? The team submission \cite{Notation3} states the following:
%
%To be sure that this
%result has the expected meaning, we did two additional tests. For the first one we added the formula
%\begin{equation}\label{eq3}
% \verb!{:c :p :o} => {:c :p :o}.!
%\end{equation}
%for the second the formula
%\begin{equation}
% \verb!{?x :p :o} => {?x :p :o}.!
%\end{equation}
%Listing \ref{forth} shows the result.









%\begin{lstlisting}[
%  float=t,
%  caption={Example of nested implications containing a universal variable \emph{(example2.n3)}},
%  label=first]

%§\textcolor{gray}{@prefix : <http://example.org/test\#>.}§

%{
%  {?x :p :o.} => {?x :p :o.}.
%} 
%=> 
%{
%  {?x :pp :oo.} => {?x :pp :oo.}.
%}.
%\end{lstlisting}

\begin{lstlisting}[
  float=t,
  caption={Output of EYE for formula (\ref{eq2}) and formula (\ref{eq4})  },
  label=forth]  
§\textcolor{gray}{@prefix : <http://example.org/test\#>.}§

{
  {?U0 :p :a.} => {?U0 :q :b.}.
}
=> 
{
  {?U0 :r :c.} => {?U0 :s :d.}.
}.

{:e :p :a.} => {:e :q :b.}.

{:e :r :c.} => {:e :s :d.}.

\end{lstlisting}

\begin{lstlisting}[
  float=t,
  caption={Output of cwm for formula (\ref{eq2})  },
  label=cwm2]  
§\textcolor{gray}{@prefix : <http://example.org/test\#>.}§
§\textcolor{gray}{   @prefix ex: <\#> .}§

{
  @forAll ex:x. {ex:x :p :a.} => {ex:x :q :b.}.
}     
=> 
{ 
  @forAll ex:x. {ex:x :r :c.} => {ex:x :s :d.}.
}.
\end{lstlisting}



\begin{lstlisting}[
  float=t,
  caption={Output of EYE for formulas (\ref{nest}) and (\ref{spo}) },
  label=eye4]  
§\textcolor{gray}{@prefix : <http://example.org/test\#>.}§

:s :p :o.
:s :pp {:s :ppp :ooo}.
{?U0 :p :o} => {?U0 :pp {?U0 :ppp :ooo}}.

\end{lstlisting}

\begin{lstlisting}[
  float=t,
  caption={Output of cwm for formulas (\ref{nest}) and (\ref{spo}) },
  label=cwm4]  
§\textcolor{gray}{@prefix : <http://example.org/test\#>.}§
§\textcolor{gray}{   @prefix ex: <\#> .}§

@forAll ex:x .   
:s :p :o;
   :pp {:s :ppp :ooo .}.
{ ex:x :p :o .} => {ex:x :pp {ex:x :ppp :ooo.}.}.
\end{lstlisting}



\MyQuote{
 ``Apart from the set of statements, a formula also has a set of URIs of symbols which are universally quantified, 
 and a set of URIs of symbols which are existentially quanitified. Variables are then in general symbols which have been quantified. 
 There is a also a shorthand syntax ?x which is the same as :x 
 except that it implies that x is universally quantified not in the formula but in its parent formula.''}
 %
This quote strengthens the position of cwm but also makes the formalization and implementation of Notation3 challenging, 
%The scope of a variable is its parent formula, but 
especially considering it together with our observation on equations (\ref{nest}) and (\ref{spo}): %, this gets more complicated, 
%as the scoping also includes nested formulas.
%What if this parent formula
%depends of another formula containing the same variable?
If a universal variable occurs in a deeply
nested formula, the scope of this particular variable can either be its direct parent, the parent of any predecessor containing a variable with the same name
or even the direct parent of a predecessor's sibling containing the same variable on highest level. %Take for example the formula
Consider for example the formula
%\small
\begin{multline}\label{twovars}
 \texttt{\{?x :p :o.\}=> \{%} \\ \texttt{
 \{\{?x :p2 ?y.\} => \{?x :p3 ?y.\}.\}}\\ \texttt{=>\{\{?x :p4 ?y.\} => \{?x :p5 ?y.\}.\}.\}.}
\end{multline}
%\normalsize
Which, according to (III), has to be interpreted as the first order formula 
\[\forall x: p(x,o)\rightarrow ((\forall y_1: p_2(x, y_1) \rightarrow p_3(x, y_1))\rightarrow(\forall y_2: p_4(x, y_2)\rightarrow p_5(x, y_2))) \]
Note that in this example, there are two different scopes for \verb!?y!, but only one for \verb!?x!. One can easily think of more  complicated cases. 
%From here on we invite the interested reader to invent more complicate examples and to test them in both reasoners.
%There is another aspect which is interesting 
%in this context. It is not mentioned how nested formulas are expected to  be treated.
%If universal quantification really behaved as existential quantification with the one and only exception that 
%the quantifier is on the parent level of the formula, the behavior of the reasoners with respect to formula (\ref{nest}) should be similar to 
%their handling of formula (\ref{eq1}). 
%As both reasoners agree in their interpretation of the formulas, we understand this rather as a vagueness 
%in the formulation and see once again the need to formalize Notation3 logic.

%\clearpage

%\FloatBarrier 

