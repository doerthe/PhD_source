\section{Quantification in N3 Logic}\label{quantsec}
%The examples of implicit quantification given in Section~\ref{n3examples} were rather  
The example formulas we discussed in Section~\ref{n3examples} were rather simple and understanding their intended meaning was not 
 difficult (maybe with the exception of cited formulas which can lead to discussions \cite{TriGsemantics}). For the only case where 
two different interpretations were plausible -- Formula~\ref{both} which had universals and existentials occurring together --
the \wwwc team submission~\cite{Notation3} contains a clear statement which interpretation needs to be chosen: \\
% This is different when it comes to implicit quantification. Just like \rdf, \nthree allows the 
% The examples of \nthree formulas given so far were rather easy and their interpretation was rather straightforward
% (maybe with the exception of cited formulas which can lead to discussions \cite{TriGsemantics}).
% This is different when it comes to implicit quantification. Just like \rdf, \nthree allows the usage of implicitly existentially quantified variables, called \emph{blank nodes}.
% Blank nodes either start with ``\verb!_:!'' or are expressed using square brackets ``\texttt{[~]}''. 

\MyQuote{  ``If both universal and existential quantification are specified for the same formula, 
then the scope of the universal quantification is outside the scope of the existentials''.}

Unfortunately, not all cases are that clear.
% 
% 
% in Section~\ref{n3examples}, we gave a few examples for \nthree formulas and their intended meanings. Since \nthree aims to extend \rdf, a \wwwc standard with well-defined 
% semantics, the meaning of simple  
%
% In the cases illustrated in Section~\ref{n3examples}, the interpretation of the implicitly quantified formulas was rather easy:
% the variables are existentially and universally quantified at the top of the formula; 
% if they co-occur in the same formula, the universal quantification dominates the existential. 
When implicitly quantified variables occur in deeply nested formulas, 
 their intended meaning is not always obvious and the interpretations of such formulas sometimes differ between reasoning engines. 
%
%In the next subsections we give examples for such ambiguous formulas.
In this section we want to better understand these differences. 
With this goal, we perform several tests on the reasoners Cwm \cite{cwm} and EYE~\cite{eyepaper} and compare their results. 
Cwm and EYE were chosen because they cover most constructs specified in \nthree{}'s \wwwc team submission. 
In contrast to for example FuXi~\cite{fuxi}, they both support rather complex constructs like nested rules. 
% and because they differ in their way to handle 
% implicit universal quantification.
% Further details about how differences can be detected can be found in our previous paper \cite{arndt_ruleml_2015}. % and in \ref{ap1}.



% 
% Before clarifying syntax and semantics of \nthree in a more formal way than presented in Section~\ref{n3examples}, 
% we test how implicit quantification of \nthreelogic is understood in practice. We  take a look to the official sources of \nthree, namely the \wwwc team submission~\cite{Notation3}
% and the journal paper about \nthree~\cite{N3Logic}, and test, how the reasoners Cwm~\cite{cwm} and EYE~\cite{eye} understand them. 
% We choose these two reasoners because of their coverage: while for example FuXi~\cite{fuxi} only supports \nthree datalog -- a subset of \nthree which does not support nested rules --
% the reasoners Cwm and EYE cover a big part of the specifications
% As in \rdf, atomic formulas are triples consisting of subject, predicate and object. They can be intuitively understood as 
% first order formulas of the form \linebreak $\text{predicate}(\text{subject}, \text{object})$. It is also easy to get an idea of the meaning of conjunctions or implications 
% if no variables are involved. Including implicit quantification is more difficult.
% Definition \ref{voc} distinguishes between two kinds of variables: universal and existential variables. 
% As the names indicate, these variables are meant to be implicitly quantified. But how do we have to understand this ``implicit'' quantification?
% Some cases are quite simple. 
% If we have the formulas
% \[
% \verb! _:x :knows :Kurt.! \text{ and } \verb! ?x :knows :Kurt.!
% \]
% It is rather straight forward to understand them as ``someone knows Kurt.'' and ``everyone knows Kurt.'' In first order logic:
% %We can understand that as the fact that Kurt knows someone or something, written in first order logic:
% \[
% \exists x: \text{knows}(x, \text{Kurt}) \text{ and } \forall x: \text{knows}(x, \text{Kurt}).
% \]
%Similarly simple constructions with universals can be understood:
%\[
%\verb! ?x :knows :Kurt.!
%\]
%Means that everyone knows Kurt, in first order:
%\[
%\forall x: \text{knows}(x, \text{Kurt}).
%\]
% But the above grammar also enables us to construct more complicate statements. Does the construct 
% \begin{equation}\label{both} \verb! ?x :loves _:y.!\end{equation} mean 
% ``everybody loves someone'' or ``there is someone
% who is loved by everyone'', in first order formulas: 
% %\[\forall x \exists y : \text{loves}(x,y) \tag{1}\quad \text{ vs. } \exists y \forall x : \text{loves}(x,y)\]
% \stepcounter{equation}
% \begin{equation} % oder auch align
% \forall x \exists y : \text{loves}(x,y)\quad  \text{ vs. }\quad \exists y \forall x : \text{loves}(x,y) \tag{\ref{both}a, b}
% \end{equation}
% 
% 
% In this case we know the answer, the team submission \cite{Notation3} clearly chooses (\ref{both}a): 
% 
% \MyQuote{ \label{beide} ``If both universal and existential quantification are specified for the same formula, 
% then the scope of the universal quantification is outside the scope of the existentials''.}
% %
% And also the reasoners we tested, in particular EYE and cwm, have implemented the first interpretation (\ref{both}a).
% %\\
% 
% Such clarity is lacking when it comes to nested formulas or co-occurring formula expressions which contain variables. 
% We will treat this in the following sections, first for existential variables, then for universals.
%This is the topic of the following sections.

\subsection{Existentials}\label{existentials}
We start our considerations by taking a closer look at implicit existential quantification in nested formulas, 
\ie formulas where the blank node occurs within curly brackets \{~\}.  Such constructs are typically used to cite formulas or
to state rules. We take the following example for the latter:
%An example for such a formula is the following:
% 
% To test how both cwm and EYE understand existential quantification, we confronted them with some examples.
% Both reasoners offer the option to output all knowledge they are aware of, this includes all derived formulas and rules as well as the
% input. In most cases, different variables sharing the same name are renamed to be distinguishable. %standardized apart\footnote{Cwm does standardization apart for existentials, 
% %using square brackets ``\texttt{[ ]}'' as introduced in Remark \ref{rem}, EYE employs Prolog's standardization apart.}. %, in EYE this process includes standardization apart\footnote{As EYE is written in Prolog, 
% %Prolog's standardization apart mechanism is used see e.g. CITATION}.   
% Therefore we can use the derived output of such a reasoning process with a simple rule as input as indication of
% how the formula is interpreted. As a first example %for our tests we took 
% we invoked both reasoners with
% a formula with nested existentials:
\begin{equation}\label{eq1}
\verb!_:x :says {_:x :knows :Albert.}.!
 \end{equation}
%  \begin{equation}\label{eqrule}
% \verb!{_:x :knows :Albert.}=>{:Albert :knows _:x.}.!
%  \end{equation}
This triple is interesting because it contains the blank node \texttt{\_:x} at two positions: as a subject of the triple and inside a nested formula. We already know 
that a blank node represents an implicitly existentially quantified variable, but what we do not know yet is: what is the exact position of its quantifier 
and what is the scope of the variable? Depending on the answer, the two occurrences of \texttt{\_:x} could either refer to the same instance of the domain of discourse, 
then the formula means
%The formula could either mean
%
\[\exists x : \text{says}(x, \text{knows}(x, \text{Albert})) \tag{\ref{eq1}a} \label{zwei}\]
\begin{center}\textit{``There exists someone who says about himself that he knows Albert.''}
  \end{center}
  or 
%   it can refer to two (possibly) different individuals. Here, we have two options:
    the \texttt{\_:x} is always quantified in the direct formula it occurs in (\ie the brackets \texttt{\{~\}}), then the two existentials can refer to different objects 
  and we get
%  
% \[
% \exists x_1 : \exists x_2: \text{says}(x_1, ( \text{knows}(x_2, \text{Albert})))\tag{\ref{eq1}b} \label{zw1}\]
% \begin{center}\textit{``There exists some $x_1$ and some $x_2$ and  $x_1$ says about $x_2$ that he knows Albert.''}\end{center}
\[
\exists x_1 : \text{says}(x_1, (\exists x_2: \text{knows}(x_2, \text{Albert})))\tag{\ref{eq1}b} \label{zw}\]
\begin{center}\textit{``There exists someone who says that there exists someone (possibly someone else) who knows Albert.''}\end{center}
\begin{lstlisting}[
  float=t,
  caption={Reasoning result of cwm for Formula \ref{eq1}. The two occurrences of the variable \texttt{\_:x} get translated to two blank nodes in square bracket notation. },
  label=cwm1]
§\textcolor{gray}{@prefix : <http://example.org/ex\#>.}§

[ :says { [ :knows :Albert ]. } ].
\end{lstlisting}

\begin{lstlisting}[
  float=t,
  caption={Reasoning result of EYE for Formula \ref{eq1}. The two occurrences of the variable \texttt{\_:x} get translated to two different blank nodes.},
  label=third]
§\textcolor{gray}{PREFIX : <http://example.org/ex\#>}§

_:t_0 :says {_:t_1 :knows :Albert}.
\end{lstlisting}
As explained earlier in Section~\ref{prn3}, \nthree reasoners can be invoked to output the deductive closure of their input files. If we provide 
Cwm and EYE with Formula~\ref{eq1}  and let them derive the deductive closure we get the result displayed in Listings~\ref{cwm1} and \ref{third}, respectively.\footnote{
Unless indicated otherwise we use in this thesis the following versions of the reasoners:
Cwm v 1.197 2007/12/13 15:38:39 syosi and EYE v18.0312.0936}
We see, that Cwm uses the bracket notation \texttt{[~]}  for blank nodes where, as explained in Section~\ref{vars}, every new bracket refers to a fresh new blank node; 
EYE gives them different names (\texttt{\_:t\_1} and \texttt{\_:t\_2}). This behaviour shows, that both reasoners understand the two occurrences of \texttt{\_:x} as different 
blank nodes which results in Interpretation~\ref{zw}.%
\footnote{
Strictly speaking, this example still does not make sure that the reasoners really assume Interpretation~\ref{zw} since both blank nodes could be quantified on top level:
$\exists x_1 : \exists x_2: \text{says}(x_1, ( \text{knows}(x_2, \text{Albert})))$.
To exclude this possibility the reasoners can be tested with Triple~\ref{simple} and the rule \texttt{\{\_:x :knows :Albert.\} => \{:Albert :is :known.\}.}: only if 
the quantifier for the blank node is in the antecedent of the rule, this input results in \texttt{:Albert :is :known.} This is the case for both reasoners.
}
%but we do not know from that example, whether this separation really results in Interpretation~\ref{zw}. It could still be that the reasoners understand the quantifier

% Strictly speaking, this example still does not make sure that the reasoners really assume Interpretation~\ref{zw} since the quantifiers for the blank nodes in the formulas are 
% not written down and the formula could still have the following meaning:
% \[
% \exists x_1 : \exists x_2: \text{says}(x_1, ( \text{knows}(x_2, \text{Albert})))\tag{\ref{eq1}c} \label{zw1}\]
% \begin{center}\textit{``There exists some $x_1$ and some $x_2$ and  $x_1$ says about $x_2$ that he knows Albert.''}\end{center}
% To exclude this possibility, we do one extra test. Depending on the position of the existential quantifier the rule 
% % Is there someone who says about himself that he knows Albert, or does this someone just state that someone exists who knows Albert?
% % In first order logic
% % \[\exists x : \text{says}(x, \text{knows}(x, \text{Albert})) \tag{\ref{eq1}a}\]
% % \hspace{6cm} or
% % \[\exists x_1 : \text{says}(x_1, (\exists x_2: \text{knows}(x_2, \text{Albert})))\tag{\ref{eq1}b} \label{zwei}\]
% % %
% % Listing \ref{third} shows the output of EYE given formula (\ref{eq1}) as only input, Listing \ref{cwm1} the output of cwm. 
% % We clearly see\footnote{To see this evidence 
% % for cwm, recall that every new bracket ``\texttt{[}$\ldots$\texttt{]}'' corresponds with a \emph{new} existential variable, see also Remark \ref{rem} or \cite{turtle} for further information. }
% % that both reasoners favor 
% % option (\ref{zwei}).  
% 
% 
% 
% %\footenotetext
% 
% %\FloatBarrier
% % 
% % 
% % We observe similar behavior using the same existential quantifier in two co-occurring graphs.
% % In an example formula such as 
% %
% \begin{equation} \label{ruru}
% \verb!{_:x :knows :Albert.} => {:Albert :is :known.}.! 
% \end{equation}
% either translates to
% \[
% (\exists x: \text{knows}(x,\text{Albert}))\rightarrow \text{is}(\text{Albert}, \text{known})\tag{\ref{ruru}a}
% \]
% in which case the formula is equivalent to 
% \[
% \forall x:( \text{knows}(x,\text{Albert}))\rightarrow \text{is}(\text{Albert}, \text{known})\tag{\ref{ruru}a'}
% \]
% or it translates to
% \[
% \exists x:( \text{knows}(x,\text{Albert}))\rightarrow \text{is}(\text{Albert}, \text{known})\tag{\ref{ruru}b}
% \]
% These two options can be tested against each other by giving Rule~\ref{ruru} to the reasoners together with Triple~\ref{simple},
% $\verb! :Kurt :knows :Albert.!
% $, if the reasoners derive 
% \begin{equation}
%  \verb!:Albert :is :known.!
% \end{equation}
% we can assume that they favour 
% 
% 
% %So, for EYE, the scope of existential quantification is always only the direct formula expression the existential occurs in but 
% %not it nested dependencies.
% The two \verb!_:x! are interpreted as different variables by both reasoners. In first order logic this would be:
% \[
%  (\exists x_1: \text{knows}(x_1, \text{Albert}))\rightarrow (\exists x_2: \text{knows}(x_2, \text{Kurt}))
% \]
%
%
%
This example shows what is meant by the following quote from the \wwwc team submission \cite{Notation3}:

\MyQuote{
``When formulae are nested, \_: blank nodes syntax \emph{[is]} used 
to only identify blank node in the formula it occurs directly in. 
It is an arbitrary temporary name for a symbol which is existentially quantified within the current 
formula (not the whole file). They can only be used within a single formula, and not within nested formulae.''
}
%
%This means, the scope of an existential quantifier is always only the formula-expression ``\verb!{!$\ldots$\verb!}!'' it occurs in, but not its nested dependency.
%We added these two examples to be able to compare the interpretation of existential quantification with universal quantification.
%Although this section might not seem surprising, we included the above 
%We keep these examples for the scoping of existentials in the paper, to be able to emphasize the different scopes of existential and universal quantification.
In other words: existential variables are always quantified in the direct formula -- marked by curly brackets \texttt{\{~\}} -- they occur in. 
This quantifier does always only count for that formula and not for the nested formulas depending on it.


\subsection{Universals}\label{universals}
The case of implicit existential quantification in \nthree was still clear and the reasoners we tested did not disagree on the meaning of implicitly existentially 
quantified variables. This is different for universal quantification. A typical example for a formula containing universal quantification was given before in Formula~\ref{uni}:
\begin{multline}\tag{\ref{uni}}
 \texttt{\{:Kurt :knows ?x.\} => }%} \\ \texttt{
 \texttt{\{?x :knows :Kurt.\}.}%\nonumber
\end{multline}
It would not be very useful to handle the quantification in this equally as existential quantification. If universal quantifiers where also always quantified on their direct 
formula we would get the interpretation:
\[\tag{\ref{uni}a}
(\forall x_1: \text{knows}(\text{Kurt}, x_1))\rightarrow (\forall x_2: \text{knows}(x_2, \text{Kurt}))
\]
\begin{center}\textit{``If everyone knows Kurt then Kurt knows everyone''}
  \end{center}
Here, we would never be able to write any rules which reason on \rdf triples since this rule only gets invoked for a universal statement -- ``\textit{Everyone knows Kurt.}'' 
--  but it is not possible to 
make universal statements in plain \rdf. Therefore, the interpretation we already discussed earlier in Section~\ref{vars} is correct in this case:
\[\tag{\ref{uni}b}
\forall x:(\text{knows}(\text{Kurt}, x))\rightarrow (\text{knows}(x, \text{Kurt}))
\]
\begin{center}\textit{``Everyone Kurt knows also knows Kurt.''}
  \end{center}
The quantifier for both occurrences of \texttt{?x} is in front of the whole formula. But what happens if the universal variable occurs in a deeper nested formula? 
% We test 
% that on the following example:
% 
% 
%
Here,
the \wwwc team submission states the following:\\

\MyQuote{
 ``There is a also a shorthand syntax ?x which [...] implies that x is universally quantified not in the formula but in its \textbf{parent formula}.''}
We learn that universal variables are quantified on their \emph{parent formula}. Unfortunately neither the \wwwc team submission nor the journal paper about \nthreelogic 
provide us with a definition of that concept. We therefore perform some tests to better understand which formula is the parent.
 
Consider the following formula:
%When it comes to the definition of the scope, universal quantifiers are more complicated. To illustrate that, we consider the following example:
%Another example is the following formula:
%\\ %s and this is also how it is implemented in the reasoners we are considering in this paper, cwm and EYE. 
%It gets even more difficult when it comes to nested rules. As the critical example for this paper, we consider the formula 
\begin{multline}
\texttt{\{\{?x :p :a.\} => \{?x :q :b.\}.\} =>}\\
\texttt{\{\{?x :r :c.\} => \{?x :s :d.\}.\}.} \label{eq2} \end{multline}
Are all \verb!?x! the same? If not, which ones do we have to understand as equal and where are they quantified? 
Of the several options to interpret this formula, two seem to be most likely:
\[\forall x: ((p(x,a)\rightarrow q(x,b))\rightarrow ( r(x,c) \rightarrow s(x,d)))\tag{\ref{eq2}a}\label{ei}\]
\begin{center}or\end{center}
\vspace{-\baselineskip}
\begin{multline}
(\forall x_1: p(x_1,a)\rightarrow q(x_1,b))\rightarrow (\forall x_2: r(x_2,c) \rightarrow s(x_2,d))\tag{\ref{eq2}b}\label{ci}
\end{multline}
%But which interpretation is correct?
% \[
% (\forall x_1: p(x_1,a)\rightarrow q(x_1,b))\rightarrow (\forall x_2: r(x_2,c) \rightarrow s(x_2,d))\tag{\ref{eq2}a}
% \]
% \hspace{6cm}or
% \[\forall x: ((p(x,a)\rightarrow q(x,b))\rightarrow ( r(x,c) \rightarrow s(x,d)))\tag{\ref{eq2}b}\]
%In interpretation \ref{ci}, 
Interpretation~\ref{ei} understands the top formula as the parent of all other formulas, the quantifier for all occurrences of \texttt{?x} is on that formula.
For Interpretation~\ref{ci} the parent of the first two occurrences of \texttt{?x} is the formula 
\begin{equation}
\texttt{\{?x :p :a.\} => \{?x :q :b.\}.}\label{sub1}
\end{equation}
and the parent of the third and fourth occurrence of \texttt{?x} is the formula
\begin{equation}\texttt{\{?x :r :c.\} => \{?x :s :d.\}.}\label{sub2}\end{equation}
In this interpretation each occurrence of curly brackets \texttt{\{ \}} marks a level. The direct formula is what 
we encounter inside the same brackets a variable occurs in, the parent formula is then the next formula 
outside of these brackets. This explains, why the two sub formulas above carry universal quantifiers.

If a reasoner follows Interpretation~\ref{ei}, Formula~\ref{eq2} applied on the following rule
\begin{equation}\label{ss1}
 \texttt{\{:e :p :a.\} => \{:e :q :b.\}.}
\end{equation}
results in
\begin{equation}\texttt{\{:e :r :c.\} => \{:e :s :d.\}.}\label{ss2}\end{equation}
While this result cannot be derived by following Interpretation~\ref{ci}. %However, the latter yields Formula~\ref{sub2} when provided with Formula~\ref{sub1}.


If we test the reasoners Cwm and EYE with Formulas~\ref{eq2} and \ref{ss1} as input we get the results displayed in Listing~\ref{cwm2}\footnote{The 
predicate \texttt{log:implies} present in the original output is replaced here by implication arrow \texttt{=>}. The two symbols have the same meaning.} and \ref{forth}, respectively.
\begin{lstlisting}[
  float=t,
  caption={Output of cwm for for Formulas~\ref{eq2} and \ref{ss1}. The reasoner does not derive Formula~\ref{ss2}. },
  label=cwm2]  
§\textcolor{gray}{@prefix : <http://example.org/ex\#>.}§
§\textcolor{gray}{   @prefix unive: <\#> .}§
{:e :p :a.}  => {:e :q :b .}.

{@forAll unive:x . {unive:x :p :a .}=>{unive:x :q :b.}.} 
=> 
{@forAll unive:x . {unive:x :r :c .}=>{unive:x :s :d .}.}.
\end{lstlisting}
\begin{lstlisting}[
  float=t,
  caption={Output of EYE for Formulas~\ref{eq2} and \ref{ss1}. The reasoner derives Formula~\ref{ss2}.  },
  label=forth]  
§\textcolor{gray}{PREFIX : <http://example.org/ex\#>}§

{{?U_0 :p :a}=>{?U_0 :q :b}}=>{{?U_0 :r :c}=>{?U_0 :s :d}}.
{:e :p :a} => {:e :q :b}.
{:e :r :c} => {:e :s :d}.
\end{lstlisting}
% The universals in the antecedent and consequent of the main implication reside in two different scopes.
% %This reflects Cwm's interpretation. % as two subformulas:
% Here, it is important that Formula~\ref{eq2} is composed of the subformulas:
% \begin{equation}
% \texttt{\{?x :p :a.\} => \{?x :q :b.\}.}\label{sub1}
% \end{equation}
% \begin{center}  
% and 
% \end{center}
% \begin{equation}\texttt{\{?x :r :c.\} => \{?x :s :d.\}.}\label{sub2}\end{equation} %are direct sub-formulas of formula \ref{eq22}. 
% Syntactically this division is marked by the use of curly brackets \texttt{\{\}}.
% Formula \ref{sub1} has two direct subformulas (again marked by brackets), namely:
% \[\texttt{\{?x :p :a.\}} \text{\hspace{0.5cm} and \hspace{0.5cm}} \texttt{\{?x :q :b.\}}\tag \label{\ref{sub1}a,b}\] 
% while
% \[\texttt{\{?x :r :c.\}} 
% \text{\hspace{0.5cm} and \hspace{0.5cm}} \texttt{\{?x :s :d.\}}
% \tag \label{\ref{sub2}a,b}\]are subformulas of Formula \ref{sub2}. 
%While it is rather easy to understand, how implicitly quantified universals are handled in EYE---they are all universally quantified on top level---it is
%worth to take a closer look into the quantification performed by cwm. 
% Cwm now applies the \wwwc team submission:
% 
% 
% \MyQuote{
%  ``There is a also a shorthand syntax ?x which [...] implies that x is universally quantified not in the formula but in its parent formula.''}
% %
We clearly see that Formula~\ref{ss2} is derived by EYE but not by Cwm. 
We furthermore see that the output of Cwm contains the key word \texttt{@forAll} at the beginning of the antecedent and of the consequent of the main rule in Lines 5 and 7. 
Here, this keyword has the same meaning as the first order symbol $\forall$. Cwm follows Interpretation~\ref{ci}.  We see that 
Cwm and EYE understand the concept of a \emph{parent} differently. 
For the remainder of this thesis we write $\textit{parent}_c$ when we refer to the parent formula as implemented in Cwm and $\textit{parent}_e$ 
for the concept of a parent realised in EYE.

While the concept of $\text{parent}_e$ is rather simple -- all variables have the same parent formula -- we need to perform more tests to better understand the concept $\text{parent}_c$: What happens 
if variables occur on different levels of a formula? Take as an example:
\begin{equation}\label{mixedlevels}
 \texttt{\{?x :q :b.\} =>\{ :a :b \{?x :s :d.\}.\}.}
\end{equation}
According to the explanation above, the parent of the second occurrence of \texttt{?x} seems to be the formula
\texttt{ :a :b \{?x :s :d.\}.}, but is that also the formula which carries the quantifier for the variable? Listing~\ref{mio} shows the output Cwm produces for Formula~\ref{mixedlevels}. 
We clearly see that there is only one universal quantifier (\texttt{@forAll}, Line~4)  for both occurrences of \texttt{?x}. 
As a general rule we can derive from that behaviour that if the parent formula of a universal variable is already in scope of a universal quantifier
this scoping also counts for its subformula. %By performing similar tests with deeper nested universals, we furthermore know that this 
\begin{lstlisting}[
  float=t,
  caption={Output of Cwm for Formula~\ref{mixedlevels} in which the variable \texttt{?x} occurs on different levels. The output formula only contains one universal quantifier 
  for both occurences. },
  label=mio]  
§\textcolor{gray}{@prefix : <http://example.org/ex\#>.}§
§\textcolor{gray}{   @prefix unive: <\#> .}§

@forAll unive:x. 
 {unive:x :q :b.} =>{:a :b {unive:x :s :d.}.}.
\end{lstlisting}
%This special case is not covered in the \wwwc team submission and the journal paper about \nthree which both only contain the explanation given in Quote III.

%The situation described above seems to be an exception from the 
%The previous examples reveal several problems: Firstly, the idea of the quantification on the \emph{parent formula} 

The previous examples reveal a general problem: 
We needed to explain how the \wwwc team submission is \emph{interpreted} by the reasoners EYE and Cwm.  %the \wwwc team submission which does not contain more information than Quote~III above.
There are many different possibilities how the term \emph{parent formula} could be understood 
and specific details like the interpretation of formulas with variables nested on different levels are not even mentioned in the \wwwc team submission. 
%differently and 
%and
Only
specific testing made us come to our conclusion that the above are EYE's and Cwm's interpretations. To be sure that the reasoners behave as intended we furthermore
contacted their creators.\footnote{We got clarification about Cwm's reasoning by asking on the public mailing list \url{https://lists.w3.org/Archives/Public/public-cwm-talk/}. 
We furthermore stood in close contact with Tim Berners-Lee with whom one of the supervisors of this thesis, Ruben Verborgh, collaborates.
We thoroughly discussed the working of EYE with with Jos De Roo.
}
% of the formula.
Most users are not even aware of the differences in interpretations and the lack of a formalism renders it difficult to discuss or even just express them. %---%
To come to a~clear and unambiguous definition of the logic, 
\nthree needs to be formalised and there needs to be a formalism to express the differences of existing interpretations. 
%This way, user can at least know how the
% The term \emph{parent formula} of a formula $f$ is understood as the next formula $g$ either occurring in curly brackets $\{g\}$ or being the top formula 
% for which $\texttt{\{}f\texttt{\}}$ is a direct component. % of the formula itself or -- in case it is a conjunction -- of one of its conjuncts.
% In our example, Formula~\ref{sub1} is the parent formula of Formulas~\ref{sub1}a and b, and Formula~\ref{sub2} is the parent formula of Formulas~\ref{sub2}a and b.
% This explains the scoping in interpretation \ref{ci}. We refer to this understanding of the concept  parent by using the subscript c, we write $\textit{parent}_c$.
% 
% This is the interpretation EYE applies. For the remainder of this paper we refer to this concept of the term by adding the subscript e, ie $\textit{parent}_e$ 
% refers 
% to the parent formula understood as the top formula.
% 
% 
% 
% 
% As above, we gave formula (\ref{eq2}) as input for both reasoners, cwm and EYE. 
% 
% Lines 1-9 of Listing \ref{forth} show the result of EYE which seems to imply\footnote{Where applicable, 
% EYE employs the ``standardization apart'' mechanism of its programming language Prolog.} that EYE supports the second interpretation (\ref{eq2}b), 
% but as it does not differ from the input, 
% we ran another test to verify that and added the formula 
% \begin{equation}\label{eq4} \verb!{:e :p :a.} => {:e :q :b.}.! \end{equation}
% to the reasoning input in order to see whether the reasoner outputs %(\ref{eq2}b) the reasoner's the result should contain the derived formula  
% \begin{equation}\verb!{:e :r :c.} => {:e :s :d.}.!\end{equation}
% as it would be the case with interpretation (\ref{eq2}b) but not with interpretation (\ref{eq2}a). % this formula cannot be derived. 
% The reasoning output of EYE shown in Listing \ref{forth} (all lines)
% verifies
% that EYE interprets all variables with the same name which occur in one single implication equally regardless of how deeply nested they occur. %variables with the same name in all nested formulas  equally within one single implication.
% 
% In contrast to this, Listing \ref{cwm2} shows the result cwm gives.  Here, the keyword\linebreak %\footnote{ For further explanation of the 
% %keyword \texttt{@forAll} see section \ref{expl}.} 
% ``\verb!@forAll!''
% can be understood as its first order counterpart ``$\forall$'' (see Section \ref{expl}). %Here we see a clear difference between that cwm's interpretation of the input differs from EYE. 
% Cwm understands 
% formula (\ref{eq2}) as stated in interpretation (\ref{eq2}a). Here we see a clear difference between the two reasoners.
% 
% 
% %\FloatBarrier
% 
% 
% After examining universals in co-ordinated expressions such as in the above implication, 
% we are also interested in how those variables are handled in subordinated formula expressions, similar to those in formula (\ref{eq1}). 
% We consider the following formula: 
% \begin{equation}\label{nest}
%  \verb!{?x :p :o.} => {?x :pp {?x :ppp :ooo.}.}.!
% \end{equation}
% To learn how the reasoners interpret this formula, we give the simple formula \begin{equation}\label{spo}\verb!:s :p :o.! \end{equation} as additional input. 
% Listings \ref{eye4} and \ref{cwm4} show the reasoning results of EYE respectively cwm. We clearly see that the two reasoners agree in their interpretation 
% and that this interpretation of formula (\ref{nest}) differs from the interpretation of the existential counterpart formula (\ref{eq1}). 
% 
% 
% 
% 
% 
% %Thus, our formalization has to carefully distinguish the interpretation of nested universals and existentials 
% %This particular difference
% %This difference has to be respected in the formalization in section \ref{formal}.\\
% Having considered the contrary behavior of the reasoners in the interpretation of formula (\ref{eq2}), the obvious question is: 
% how is this interpretation meant to be according to the official sources? The team submission \cite{Notation3} states the following:
% %
% %To be sure that this
% %result has the expected meaning, we did two additional tests. For the first one we added the formula
% %\begin{equation}\label{eq3}
% % \verb!{:c :p :o} => {:c :p :o}.!
% %\end{equation}
% %for the second the formula
% %\begin{equation}
% % \verb!{?x :p :o} => {?x :p :o}.!
% %\end{equation}
% %Listing \ref{forth} shows the result.
% 
% 
% 
% 
% 
% 
% 
% 
% 
% %\begin{lstlisting}[
% %  float=t,
% %  caption={Example of nested implications containing a universal variable \emph{(example2.n3)}},
% %  label=first]
% 
% %§\textcolor{gray}{@prefix : <http://example.org/test\#>.}§
% 
% %{
% %  {?x :p :o.} => {?x :p :o.}.
% %} 
% %=> 
% %{
% %  {?x :pp :oo.} => {?x :pp :oo.}.
% %}.
% %\end{lstlisting}
% 
% 
% 
% 
% 
% 
% 
% \begin{lstlisting}[
%   float=t,
%   caption={Output of EYE for formulas (\ref{nest}) and (\ref{spo}) },
%   label=eye4]  
% §\textcolor{gray}{@prefix : <http://example.org/test\#>.}§
% 
% :s :p :o.
% :s :pp {:s :ppp :ooo}.
% {?U0 :p :o} => {?U0 :pp {?U0 :ppp :ooo}}.
% 
% \end{lstlisting}
% 
% \begin{lstlisting}[
%   float=t,
%   caption={Output of cwm for formulas (\ref{nest}) and (\ref{spo}) },
%   label=cwm4]  
% §\textcolor{gray}{@prefix : <http://example.org/test\#>.}§
% §\textcolor{gray}{   @prefix ex: <\#> .}§
% 
% @forAll ex:x .   
% :s :p :o;
%    :pp {:s :ppp :ooo .}.
% { ex:x :p :o .} => {ex:x :pp {ex:x :ppp :ooo.}.}.
% \end{lstlisting}
% 
% 
% 
% 
%  %
% This quote strengthens the position of cwm but also makes the formalization and implementation of Notation3 challenging, 
% %The scope of a variable is its parent formula, but 
% especially considering it together with our observation on equations (\ref{nest}) and (\ref{spo}): %, this gets more complicated, 
% %as the scoping also includes nested formulas.
% %What if this parent formula
% %depends of another formula containing the same variable?
% If a universal variable occurs in a deeply
% nested formula, the scope of this particular variable can either be its direct parent, the parent of any predecessor containing a variable with the same name
% or even the direct parent of a predecessor's sibling containing the same variable on highest level. %Take for example the formula
% Consider for example the formula
% %\small
% \begin{multline}\label{twovars}
%  \texttt{\{?x :p :o.\}=> \{%} \\ \texttt{
%  \{\{?x :p2 ?y.\} => \{?x :p3 ?y.\}.\}}\\ \texttt{=>\{\{?x :p4 ?y.\} => \{?x :p5 ?y.\}.\}.\}.}
% \end{multline}
% %\normalsize
% Which, according to (III), has to be interpreted as the first order formula 
% \[\forall x: p(x,o)\rightarrow ((\forall y_1: p_2(x, y_1) \rightarrow p_3(x, y_1))\rightarrow(\forall y_2: p_4(x, y_2)\rightarrow p_5(x, y_2))) \]
% Note that in this example, there are two different scopes for \verb!?y!, but only one for \verb!?x!. One can easily think of more  complicated cases. 
% %From here on we invite the interested reader to invent more complicate examples and to test them in both reasoners.
% %There is another aspect which is interesting 
% %in this context. It is not mentioned how nested formulas are expected to  be treated.
% %If universal quantification really behaved as existential quantification with the one and only exception that 
% %the quantifier is on the parent level of the formula, the behavior of the reasoners with respect to formula (\ref{nest}) should be similar to 
% %their handling of formula (\ref{eq1}). 
% %As both reasoners agree in their interpretation of the formulas, we understand this rather as a vagueness 
% %in the formulation and see once again the need to formalize Notation3 logic.
% 
% %\clearpage
% 
% %\FloatBarrier 
% 

\subsection{Explicit Quantification}\label{remarkExplicitQuantification}
We already saw above that apart from implicit quantification, \nthreelogic also provides the possibility to explicitly quantify over variables.
To do so, the quantifiers \verb!@forSome!
and \verb!@forAll! are used. With explicit quantification, Interpretation \ref{both}a of Formula \ref{both} can be expressed as follows:
\begin{multline}\notag
 \texttt{@forAll :y. @forSome :x.}\\\texttt{:x :thinks \{:y :is :pretty\}.}
\end{multline}
Seeing this example, the reader might think that the misunderstandings described above could be avoided by only using explicit quantification and that this notation could 
even be used to explain the differences. Unfortunately, that is not the case. Independently of the order they appear in the formula, universal quantifiers are always understood to
be outside of  existential quantifiers in \nthree~\cite{Notation3}.
The formula
\begin{multline}\notag
 \texttt{@forSome :x. @forAll :y. }\\\texttt{:x :thinks \{:y :is :pretty\}.}
\end{multline}
has the exact same meaning as the previous one, namely Interpretation \ref{both}a:\footnote{We suppose that this has historic reasons: 
When \nthree was created, everything, including quantifiers,  was represented in triples. The order of triples should not matter in any context.}
$ % oder auch align
\forall y \exists x : \text{thinks}(x,\text{is}(y,\text{pretty})) % \tag{\ref{both}a}
$.
To express Interpretation \ref{both}b,
$ %\begin{equation} % oder auch align
\exists x \forall  y : \text{thinks}(x,\text{is}(y,\text{pretty})) %\tag{\ref{both}b}
$, %\end{equation}
a more complicated 
construction is needed, which could then again lead to different interpretations. 

The peculiarity described above leads to open questions. %is also one of the reasons we exclude explicit quantification from the considerations in this paper. 
For example, what does the following valid \nthree formula mean?
\begin{multline}
 \texttt{@forSome :x. :x :knows \underline{:y}.}\\ \texttt{@forAll :y. :y :knows :x.}
\end{multline}
The scope of the \texttt{@forAll} is outside of the scope of the \texttt{@forSome}, but what about the first occurrence of \texttt{:y} (underlined in the example)? Is \texttt{:y} a universally 
quantified variable or a constant?
This formula is not supported by Cwm and its intended meaning is not specified in any source we are aware of. 
This and many similar examples make explicit quantification in \nthree a complex topic on its own, which is outside 
the scope of this dissertation. Here we focus on implicit quantification.