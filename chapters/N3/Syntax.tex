
\section{Syntax}
Before coming to the main topic of this paper, implicit quantification, we start by defining the general syntax of N3Logic. We exclude built-ins %, as their 
%meaning is mostly defined on the corresponding website, %\footnote{See \url{http://www.w3.org/2000/10/swap/doc/cwmBuiltins}.},  
and explicit quantification (for more information see section \ref{expl}). %\footnote{We explain this choice in section \ref{expl}. }. 
%The latter is, as well as its implicit counterpart, not trivial. We will give a quick explanation of that at the end of this paper.\\
The syntax-definition below is oriented on the context-free grammar as provided at the team submission web page \cite{Notation3}. 
%We limit our developments to the %, at least in our opinion, 
%most important properties of \nthree---quantification, implication and quoting---and skip most inbuilt-predicates of \nthree and \rdf, whose meaning 
%can be found on the corresponding websites. %We start with the vocabulary:


\begin{definition}[Basic N3 vocabulary]\label{voc}
An \emph{N3 alphabet~$A$} consists of the following disjoint classes of symbols:
\begin{itemize}
\item A set $U$ of \uri symbols.
\item A set $V=V_E\mathbin{\dot{\cup}} V_U$  of (quantified) variables, with $V_E$ being the set of existential variables
and $V_U$ the set of universal variables.
\item A set $L$  of literals.
%\item A boolean literal \verb!false!
\item Brackets \verb!{!, \verb!}!, \verb!(!, \verb!)!
\item Logical implication $\verb!=>!$ 
\item Period \verb!.!
\end{itemize}
\end{definition}

%We define the elements of $U$ in the same way as \rdf \cite{rdf}. %in %the corresponding specification~\cite{iri}.
%Like Turtle, \nthree allows to abbreviate URLs as prefixed names~\cite{turtle}.
%Literals are strings beginning and ending with quotation marks `\verb!"!';
%existentials start with `\verb!_:!', universals with `\verb!?!'.\\
%As \nthree does not clearly distinguish between predicates and constants---a single \uri symbol can stand for both at the same time---we have to change the first-order-concept of a 
%term to something similar:

We define the elements of $U$ as in the corresponding specification~\cite{iri}. As for example in Turtle \cite{turtle},
\nthree allows  to abbreviate URLs by using prefixes.
%\nthree allows to abbreviate URLs as prefixed names~\cite{turtle}.
Literals are strings beginning and ending with quotation marks `\verb!"!';
existentials start with `\verb!_:!', universals with `\verb!?!'.
%
Unlike first order logic, \nthree does not distinguish between predicates and constants---%
a single \uri symbol can stand for both at the same time---%
so the first-order-concept of a~\emph{term}
has a~slightly different counterpart in \nthree: an \emph{expression}.
Since the definition of expressions (Definition~\ref{expression})
is closely related to the concept of a~formula (Definition~\ref{formula}),
the two following definitions should be considered together.

%We include the possibility of quoting formulas by introducing \emph{formula expressions}. Additionally to the direct citation of a formula, there are two other 
%kinds of formula-expressions: the empty expression (true) and \verb!false!. As the latter is inherited from \rdf, it can't be understood as a formula itself as the 
%similar expression in first order logic normally is. The declarations as formula expressions enable us to keep that kind of meaning in controlled situations, as will
%be shown in the following chapters.


\begin{definition}[Expressions]\label{expression}
  Let $A$ be an \nthree alphabet.
  The set of \textit{expressions} $E \subset A^{*}$ is
  defined as follows:
  \begin{enumerate}
    \item Each \uri is an expression.
    \item Each variable is an expression.
    \item Each literal is an expression.
    \item \label{list} If $e_1,\ldots,e_n$ are expressions, $\verb!(!e_1 \ldots e_n\verb!)!$ is an expression. 
    %\item \label{false} \verb!false! is an expression.
   % \item $\verb!{ }!$ is an expression.
    \item \label{fe} If $f\in F$ is a formula, then $\verb!{!f\verb!}!$ is an expression. 
  \end{enumerate}
  The expression defined by \ref{list} is called a \textit{list}.
  We call the expressions defined by %\ref{false}--
  \ref{fe}
  \textit{formula expressions} and denote the set of all formula expressions by $\mbox{\textit{FE}}$.
\end{definition}


Note that point \ref{fe} of the definition above makes use of formulas, which are defined as follows:

\begin{definition}[\nthree Formulas]
    \label{formula}
    The set $F$ of \textit{\nthree formulas} over an alphabet $A$ is recursively defined as follows:
    \begin{enumerate}  
      \item \label{1} If $e_1, e_2, e_3$ are expressions, $e_1~ e_2~ e_3$. is a formula, an \textit{atomic} formula.
      \item \label{2} If $t_1, t_2$ are formula expressions, $t_{1} \verb!=>!~t_{2}.$ is a formula, an \textit{implication}.
      \item \label{n} If $f_1$ and $f_2$ are formulas, $f_1 f_2$ is a formula, a \textit{conjunction}.
    \end{enumerate}
\end{definition}

We will refer to a~formula without any variables as a~\textit{ground formula}.
Analogously, we call such kind of expressions \textit{ground expressions}.
We denote the corresponding sets by $F_g$ respectively $E_g$.\\
%Note that 
%Althouh we see this rarely used in practice, t
The definition explicitly allows all expressions in all positions of atomic formulas. 
Literals or even formula expressions can be subjects, objects or predicates.

%Whether a triple with a 
%
%It is rather unlikely th
%It is rather unlikely that any
%Whether an element 
%The meaning of those expressions depends on the respective interpretation.

%We will refer to a~formula without any variables as a~\textit{ground formula}.
%Analogously, we call expressions without any variables \textit{ground expressions}.
%We denote the corresponding sets by $F_g$ respectively $E_g$.
% We are going to define the semantics of a formula. To enable the reader to distinguish between mathematical symbols describing the language and \nthree-constructs (expressions and formulas), 
% we mark the latter by underlinement. 

In the examples in the remainder of this paper, we will use the common \rdf shortcuts:

\begin{remark}[Syntactic variants]\label{rem}
\begin{itemize}
\item A formula consisting of two triple subformulas starting with the same element \verb!<d> <p> <e>. <d> <q> <f>.! can be abbreviated using a semicolon:\newline 
\verb!<d> <p> <e>; <q> <f>.!
\item Two triple formulas sharing the first two elements  \verb!<d> <p> <e>. <d> <p> <f>.! can be abbreviated using a comma: \verb!<d> <p> <e>, <f>.!
%  \item \verb![]! can be used as an expression and is a shortcut for a new existential variable. So \verb![] <p> <d>.! stands for \verb!_:x <p> <d>.!
%  If $[~]$ occurs in a formula $f$ instead of an expression, each instance of $[~]$  can be translated by a new existential variable.
 \item An expression of the form \verb![<p> <o>]! is a shortcut for a new existential variable \verb!_:x!,
   which is subject to the statement \verb!_:x <p> <o>.! %\linebreak
   So \verb! <s> <p> [<q> <o>].! stands for \verb!<s> <p> _:x. _:x <q> <o>.!
 %\item \verb!a! is a~shortcut for \verb!rdf:type!~\cite{RDF}.
 \end{itemize}
\end{remark}



To emphasize the difference between brackets which form part of the \nthree vocabulary, i.e. ``\verb!(!'', ``\verb!)!'', ``\verb!{!'', and ``\verb!}!'', 
and the brackets occurring in mathematical language, we will underline the \nthree brackets in all definitions where both kinds of brackets occur.
