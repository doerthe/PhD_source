\section{N3 Core Logic}\label{core}
The examples in the last section explain how \nthree formulas are interpreted differently by different reasoners. To point out these variations,
we used natural language together with a
first-order-logic-like structure. However, both the natural language and logical structure do not have a fixed definition of their semantics and can 
thus still be understood in different 
ways. In order to dispose of this ambiguity when comparing interpretations, 
 we  now define a new \emph{core logic} of \nthree. 
This logic  supports all important features of \nthree such as the possibility to refer to formulas or to use quantified variables in predicate position, 
but only allows \emph{explicit} quantification. This logic can then be used to make \nthree's implicit quantification explicit.


\begin{figure}

\begin{tabular}{llr}
\hline
Syntax: &&\\
&&\\
\texttt{t ::=}&&                    terms:\\
      & \texttt{v}\hspace{0.15\textwidth} &                variables\\
      & \texttt{c} &                constants\\
      & \texttt{e} &                 expressions\\
      & \texttt{(k)}& lists\\
      & \texttt{()}& empty list\\
      &&\\
\texttt{k ::=}&&                    listcontent\\  
       &\texttt{t}  &\\
       &\texttt{t k}&\\
\texttt{e ::=}&&                    expressions:\\
%      &\texttt{<>} &                true\\
       &\texttt{<f>} &               formula expression\\
       &\texttt{<>} & true\\
       &\texttt{false}       &               false\\
       &&\\
\texttt{f ::= } & &                   formulas:\\  
    &  \texttt{t t t}&                atomic formula\\
    &  \texttt{e} $\rightarrow$ \texttt{e}& implication\\
%    &  \texttt{f} $\rightarrow \bot$ &\\
%    &  \texttt{f} $\rightarrow \top$ &\\
%    &  $\top \rightarrow $ \texttt{f} &\\
%    &   $\bot \rightarrow $ \texttt{f} &\\
    &  \texttt{f f} &                 conjunction\\
    &  \(\forall\)\texttt{v.f}     & universal formula\\
    &  \(\exists\)\texttt{v.f}     & existential formula\\
    \hline
\end{tabular}
\caption{Syntax of the core language $\mathcal{L}$ over $\mathcal{V}\cup\mathcal{C}$.\label{syntax}}
\end{figure}

\subsection{Syntax}
Given disjoint countable sets of variables $\mathcal{V}$ and constants $\mathcal{C}$ we define the core language $\mathcal{L}$ of \nthree over \linebreak
$\mathcal{C}\cup\mathcal{V}$ as displayed in Figure \ref{syntax}. 
In core logic interpretation \ref{zwei} is expressed as:
\[
 \exists \texttt{x. x says <x knows Albert>}
\]
And Interpretation \ref{zw}:
\[
 \exists \texttt{x1. x1 says <}\exists\texttt{x2. x2 knows Albert>} 
\]
%\rv{Nit, but unfortunate choice of the <> symbols, which are used in N3 and elsewhere on the Web to denote IRIs}
%\da{what do you recommend? [] and () are also taken and I need brackets.}
Note that this notation is close to the original \nthree notation. To make a clear distinction between core logic and \nthreelogic, 
we use angle brackets instead of curly brackets and
a different kind of arrow. 
For the same reason, we do not represent constants and variables using \iri{}s (ie we write \texttt{x} instead of \texttt{:x}) in our examples.%
\footnote{Note that the representation of the constants and variables only depends on the choice of $\mathcal{C}$ and $\mathcal{V}$.} 
The main difference between \nthreelogic and core logic is the symbol used for explicit quantification in the latter, 
which is taken from first order logic to emphasize 
that the quantifiers here are interpreted in the order they occur.






Variables in a formula can either occur free or bound:
\begin{definition}[Free variables]\label{free}
The set of free variables of a language element~$\texttt{l}\in\mathcal{L}$, written $\text{FV}(\texttt{l})$ is defined as follows:

\begin{align*}
 \text{FV}(\texttt{v}) &= \{\texttt{v}\}\\
 \text{FV}(\texttt{c}) &= \emptyset\\
 \text{FV}(\texttt{<f>}) &= \text{FV}(\texttt{f})\\
 \text{FV}(\texttt{<>}) &= \emptyset\\
 \text{FV}(\texttt{false}) &= \emptyset\\
 \text{FV}(\texttt{(t}_1\ldots\texttt{t}_n\texttt{)})&=\text{FV}(\texttt{t}_1)\cup\ldots\cup\text{FV}(\texttt{t}_n)\\
 \text{FV}(\texttt{()})&= \emptyset\\
 \text{FV}(\texttt{t}_1\texttt{t}_2\texttt{t}_3) &= \text{FV}(\texttt{t}_1)\cup\text{FV}(\texttt{t}_2)\cup \text{FV}(\texttt{t}_3)\\
 \text{FV}(\texttt{e}_1\rightarrow \texttt{e}_2) &= \text{FV}(\texttt{e}_1)\cup\text{FV}(\texttt{e}_2)\\
 \text{FV}(\texttt{f}_1\texttt{f}_2) &= \text{FV}(\texttt{f}_1)\cup\text{FV}(\texttt{f}_2)\\
 \text{FV}(\forall \texttt{v}.\texttt{f}) &= \text{FV}(\texttt{f})\setminus\{\texttt{v}\}\\
 \text{FV}(\exists \texttt{v}.\texttt{f}) &= \text{FV}(\texttt{f})\setminus\{\texttt{v}\}\\
\end{align*}
We call every language element $\texttt{l}\in \mathcal{L}$ with $\text{FV}(\texttt{l})=\emptyset$ \emph{ground}.
% 
% The set of ground elements of $\mathcal{L}$ is defined as:
% \[
%  \mathcal{L}_g=\{\texttt{l}\in \mathcal{L}| \text{FV}(\texttt{l})=\emptyset\}
% \]
%If a formula $\texttt{f} \in \mathcal{F}$ does not contain free variables, ie $\text{FV}(\texttt{f})=\emptyset$, we call this formula \emph{ground}. 
%Accordingly, we call a term 
%$\texttt{t}\in \mathcal{T}$ with $\text{FV}(\texttt{t})=\emptyset$ \emph{ground}. 
We denote the set of ground language elements by $\mathcal{L}_g$, 
for the set $\mathcal{T}$ of terms as defined in Figure~\ref{syntax}, we define  $\mathcal{T}_g:= \mathcal{L}_g\cap \mathcal{T}$ as the set of ground terms.% By 
%$\mathcal{E}_g := \mathcal{E}\cap \mathcal{T}_g$ we denote the set of ground expressions.
\end{definition}
For example,
in the formula
\[
 \forall \texttt{x. x knows y}
\]
the variable \texttt{x} occurs bound, while \texttt{y} is free.



\subsection{Semantics}
To define the semantics of the core language $\mathcal{L}$ over $\mathcal{V}\cup\mathcal{C}$ we first introduce the notion of structure. 

\begin{definition}[Structure]
A structure over a language $\mathcal{L}$ %alphabet $\mathcal{A}$ 
%\nthree care language
is a triple $\mathfrak{A}=(\mathcal{D}, \mathfrak{a}, \mathfrak{p})$ containing:
\begin{itemize}
 \item A non empty set $\mathcal{D}$ %is a non empty set, 
 called the \emph{domain}. %of $\mathfrak{A}$.
 \item A mapping $\mathfrak{a}: \mathcal{T}_g\rightarrow \mathcal{D}$ 
 called the \emph{object mapping}.
 %is a mapping from the set of ground expressions into the domain of discourse. %is mapping defined on $\mathcal{A}$ with the following properties.
 %\begin{itemize}
 
%  \item For every constant $c\in\mathcal{C}\subset \mathcal{A}$, $\mathfrak{a}(c)\in \mathcal{D}$.
%  \item For every ground expression $e\in \mathcal{E}_{g}$, $\mathfrak{a}(e)\in \mathcal{D}$. 
% %  \Remark{Not sure yet, whether I need some additional restrictions or interpretation functions here. 
% %  Maybe I want to treat two formulas which are logically equivalent equally? N3 reasoners don't do that as far as I know.
% %  N3 paper only says the I shouldn't use knowledge from other formulas.
% %  }
%  %\item For the relation $\text{tr}$, $\mathfrak{a}(\text{tr})\subset \mathcal{D}^3$.
%  \end{itemize}
\item A mapping $\mathfrak{p}: \mathcal{T}_g\rightarrow 2^{\mathcal{D}\times\mathcal{D}}$ called the \emph{predicate mapping}.  %is a mapping on  the domain elements 
%$d\in \mathcal{D}$ such that $\mathfrak{p}(d)\in 2^{\mathcal{D}\times\mathcal{D}}$.
%$\mathfrak{p}:\mathcal{T}_g\rightarrow 2^{\mathcal{D}\times\mathcal{D}}$ is a mapping which maps every ground term into a set of domain pairs.
\end{itemize}
\end{definition}
Similarly to a structure in the classical first order sense, a core logic structure 
consists of a domain of discourse and maps from the terms of the language into that domain. There are two main differences:

Firstly, there is not \emph{one} mapping into the domain of discourse, but \emph{two}. The reason is that in \rdf and \nthree -- and thereby also in its core logic -- the 
same symbol 
can represent a predicate, a subject or an object:
\[
 \texttt{knows a predicate. Albert knows Kurt.}
\]
is a formula of the core language. A structure should map the first occurrence of the term ``\texttt{knows}'' to one element of the domain and its second occurrence to 
a set of pairs of domain elements, since the latter 
represents a relation. 
% \rdf semantics~\cite{RDFSemantics} deals with that case in a similar way: One interpretation function maps from the terms into the set of resources (IR) and properties (IP) 
% and a second mapping assigns 
% a set of resource pairs to all properties. Even though we do not distinguish between resources and properties, the two definitions are compatible. 
% For a resource in the \rdf sense which is not a property, the predicate mapping $\mathfrak{p}$ simply maps into the empty set.  

The second difference is that our structure is only defined for ground terms and not, as it is in other logics, for terms containing variables. 
To define the semantics for these we make use of a grounding function:


\begin{definition}[Grounding Function]
A \emph{grounding function} $\gamma_g$ over a language $\mathcal{L}$ is a mapping from the set of variables $\mathcal{V}$ %in $\mathcal{A}$
into the set of ground terms $\mathcal{T}_g$. % of $\mathcal{A}$. 
By $\gamma_g [x\mapsto t]$ 
 we denote the grounding function which is identical to $\gamma_g$ except for $x\mapsto t$. 
% Likewise we denote a mapping which behaves like a gounding function $\gamma$ 
% but does not change the variable $x$ by $\gamma\{x/x\}$.
\end{definition}

This function can be extended to all elements of the language:

\begin{definition}[Extended Grounding Function]
For a language $\mathcal{L}$ and a  \emph{grounding function} $\gamma_g: \mathcal{V}\rightarrow \mathcal{T}_g$ we define its extension
%is a mapping $\gamma_b:\mathcal{V}\rightarrow \mathcal{T}_g$ which can be extended to a function 
$\gamma:\mathcal{L}\rightarrow \mathcal{L}_g$ as follows: 
\begin{align*}
 \gamma(\texttt{v}) &= \gamma_g(\texttt{v})\\
 \gamma(\texttt{c}) &= c\\
 \gamma(\texttt{<f>}) &= \texttt{<}\gamma(\texttt{f})\texttt{>}\\
 \gamma(\texttt{<>}) &= \texttt{<>}\\
 \gamma(\texttt{false}) &= \texttt{false}\\
 \gamma(\texttt{(t}_1\ldots\texttt{t}_n\texttt{)})&= \texttt{(}\gamma(\texttt{t}_1)\ldots\gamma(\texttt{t}_n)\texttt{)}\\
 \gamma(\texttt{()})&=\texttt{()}\\
 \gamma(\texttt{t}_1\texttt{t}_2\texttt{t}_3) &= \gamma(\texttt{t}_1)\gamma(\texttt{t}_2)\gamma(\texttt{t}_3)\\
 \gamma(\texttt{e}_1\rightarrow \texttt{e}_2) &= \gamma(\texttt{e}_1)\rightarrow\gamma(\texttt{e}_2)\\
 \gamma(\texttt{f}_1\texttt{f}_2) &= \gamma(\texttt{f}_1)\gamma(\texttt{f}_2)\\
 \gamma(\forall \texttt{v}.\texttt{f}) &=  \forall \texttt{v}. \gamma[\texttt{v}\mapsto \texttt{v}](\texttt{f})\\
  \gamma(\exists \texttt{v}.\texttt{f}) &=  \exists \texttt{v}. \gamma[\texttt{v}\mapsto \texttt{v}](\texttt{f})
\end{align*}
\end{definition}

For the definition of an interpretation and meaning we now use the grounding function instead of a classical 
valuation function which maps the set of variables into the domain of discourse. We have two reasons for that:
%We have two reasons for using a grounding function instead of a classical valuation function from the set of variables into the domain of discourse:

The first reason is again related to the lack of distinction 
between objects and predicates. In \nthree it is possible to quantify over predicates. 
These predicates could also be used in subject or object position as in the example formula\footnote{Here and for the remainder of this paper, we use the symbols $\lhexbrace$ and $\rhexbrace$ as auxiliary brackets 
to clarify the structure of the formulas shown.}
%and this aspect needs to be covered by the core logic. A formula of the form
\[
 \forall \texttt{x.}\lhexbrace\texttt{ x a predicate. Albert x Kurt.}\rhexbrace
\]
%is part of the core language.
We need make sure that the universal quantifier takes the relation between the 
two occurrences of \texttt{x} into account. 
We do that by using the same mapping into the set of ground terms for both of them.
%both occurrences of \texttt{x} get assigned 
%to the same constant and
%that a quantifier over a variable occurring 
By doing so we furthermore make sure that a variables in predicate position cannot refer to all elements of $2^{\mathcal{D}\times\mathcal{D}}$ and thereby lead us into second order logic.
Following the idea of  Henkin Semantics \cite{henkin_1950}, we ensure by grounding first that it can only be quantified over 
the countable subset of those relations 
which have a name in $\mathcal{L}$. 
A mapping into first order logic could then be done similarly as it has been performed for KIF \cite{skif}. %\todo{not two paragraphs over grounding, but one}
%To also assign a meaning to variables, we define the following mapping:
%To
%As explained above, the use of a grounding function instead of a direct mapping from the variables into the domain of discourse ensures that we stay in first order logic.

An alternative to perform grounding would have been to use the functions IS and IEXT as defined in \rdf Semantics \cite{RDFSemantics}. 
This brings us to the second reason to perform grounding first: %The reason for not doing that 
%not directly 
%use these functions here and use a grounding function and a predicate mapping instead 
%
%has to do with 
Many of the built-ins defined 
for \nthreelogic\footnote{\url{https://www.w3.org/2000/10/swap/doc/CwmBuiltins}} 
%Many of these functions 
act on the representation layer of the logic. The predicate \texttt{log:equalTo}, for example, 
compares concrete \uri{}s and literals:
\[
 \texttt{dbpedia:Bern log:equalTo \texttt{dbpedia:Bern}.} 
\]
is \texttt{true} while
\[
 \texttt{dbpedia:Bern log:equalTo dbpedia-nl:Bern.}
\]
results in \texttt{false}, even though we can assume that both \uri{}s refer to the same place in Switzerland.
While we do not explicitly define the meaning of built-ins in this paper 
we still want to make sure that our theory is able to cover such cases.
%To be able to define the semantics of such built-ins in our theory, we do the grounding step.

An interpretation for core formulas therefore makes use of a grounding function:


\begin{definition}[Core logic interpretation]
A core logic interpretation $\mathfrak{I}$ is a pair $(\mathfrak{A}, \gamma)$ consisting of a structure $\mathfrak{A}$ and a grounding function $\gamma$.
\end{definition}


With the present definitions we can now define the meaning  of formulas. 


\begin{definition}[Meaning of formulas]
Let $\mathfrak{I}=(\mathfrak{A}, \gamma)$ be a core logic interpretation of a language $\mathcal{L}$. %and $\texttt{f}_1$ and $\texttt{f}_2$ formulas. 
Then:
 \begin{enumerate}
 \item $\mathfrak{I}\models \texttt{t}_1\texttt{t}_2 \texttt{t}_3$ iff 
 $(\mathfrak{a}(\gamma(\texttt{t}_1)), \mathfrak{a}(\gamma(\texttt{t}_3))\in \mathfrak{p}(\gamma(\texttt{t}_2)))$
  %\item $\mathfrak{I}\models \text{tr}(t_1, t_2, t_3)$ iff $(\mathfrak{I}(t_1), \mathfrak{I}(t_3))\in \mathfrak{p}(\mathfrak{I}(t_2))$. %$(\mathfrak{I}(t_1), \mathfrak{I}(t_2), \mathfrak{I}(t_3))\in \mathfrak{a}(tr)$.
  \item $\mathfrak{I}\models \texttt{<f}_1\texttt{>}\rightarrow \texttt{<f}_2\texttt{>}$ iff 
  $\mathfrak{I}\models \texttt{f}_2$ if $\mathfrak{I}\models \texttt{f}_1$.
  \item $\mathfrak{I}\models \texttt{false} \rightarrow \texttt{<f>}$
  \item $\mathfrak{I}\models \texttt{<f>}\rightarrow \texttt{<>}$
  %\item $\mathfrak{I}\models \bot \rightarrow (\phi)$ and $\mathfrak{I}\models (\phi)\rightarrow \top$.
  \item $\mathfrak{I}\models \texttt{<f>} \rightarrow \texttt{false}$ iff $\mathfrak{I}\not \models \texttt{f}$.
  \item $\mathfrak{I}\models \texttt{<>} \rightarrow \texttt{<f>}$ iff $\mathfrak{I}\models \texttt{f}$.
  \item $\mathfrak{I}\models \texttt{f}_1\texttt{f}_2$ iff $\mathfrak{I}\models \texttt{f}_1$ and $\mathfrak{I}\models \texttt{f}_2$.
  \item $\mathfrak{I}\models \forall \texttt{v.f}$ iff $(\mathfrak{A}, \gamma[\texttt{v}\mapsto\texttt{t}])\models\texttt{f}$ for all $t\in \mathcal{T}_g$.
  \item $\mathfrak{I}\models \exists \texttt{v.f}$ iff $(\mathfrak{A}, \gamma[\texttt{v}\mapsto\texttt{t}])\models \texttt{f}$ for some $t\in \mathcal{T}_g$.
 % \item $\mathfrak{I}\models \exists x (\phi)$ iff $(\mathfrak{A}, \gamma\{x/t\})\models \phi$ for some $t\in \mathcal{T}_g$.
 \end{enumerate}
%
We call a formula $\texttt{f}$ \emph{true} in $\mathfrak{I}$ if $\mathfrak{I}\models \texttt{f}$. 
\end{definition}

%Next to the fact that we do not use the functions IS and IEXT from \rdf Semantics, there is another detail which the reader familiar to these definitions might consider 
Note that in \nthree lists are treated as ``first-class citizens'', this means that \nthree does not use \rdf's 
\texttt{rdf:first}-\texttt{rdf:rest} notation to construct lists but allows the treatment of lists as a proper data type (see also \cite[p.6]{N3Logic}). 
This is reflected
in our core logic and its semantics, where we directly map ground lists into the domain of discourse. We do the same for cited formulas since such a treatment allows later
refinement by specifying the mapping function further. %We see this refinement as future work.


The definition of a model is similar to first order logic:

%This enables us to define the 
%concept of a model of or 

\begin{definition}[Model]
 Let $\Phi$ be a set of formulas. We call a core logic interpretation $\mathfrak{I}$ a \emph{model} of $\Phi$ iff every formula in $\Phi$ is true in $\mathfrak{I}$.
\end{definition}

% \Remark{Problem:
%  I think I should guarantee that two
% ``expression-formulas'' which only differ in their variable names, are treated equally. How?
% Using logical equivalence? 
% }

We finish this section by defining the classical concepts of logical consequence and equivalence:

\begin{definition}[Logical consequence]
 Let $\Phi$ be a set of ground formulas. A ground formula $\texttt{f}$ is called a \emph{logical consequence} of $\Phi$ (written: $\Phi\models\texttt{f}$) 
 iff $\texttt{f}$ is true in every model 
 of $\Phi$.
\end{definition}
 Instead of $\{\texttt{f}_1\}\models \texttt{f}_2$ we sometimes write $\texttt{f}_1\models\texttt{f}_2$.

\begin{definition}[Logical equivalence]
 Two formulas $\phi$ and $\psi$ are called logically equivalent ($\phi \equiv \psi$) iff $\phi \models \psi$ and $\psi\models \phi$.
\end{definition}

