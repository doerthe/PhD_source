\section{A direct semantics for N3}

When defining the Semantics of our Core Logic in Section~\ref{core}, we took the freedom to choose the details of our interpretation function. We, for example
chose to apply a grounding function on quantified variables before mapping these to the domain of discourse which makes the formal 
definition of built-in functions easier but also results in the fact, that we for example cannot represent the set of real numbers in
our version of \nthree since the set over which we can quantify is countable. When defining a direct semantics for N3 which extends the semantics of \rdf, we do not have this freedom:



\subsection{Formalization of quantification}\label{formal}
%When it comes to semantics of \nthree one aspect where the interpretations chosen by the implementers of reasoning engines differ is the understanding 
%of quantified variables. It is not always clear in which cases the variables with the same name actually mean the same and when they have to be interpreted differently.
%We propose a solution for that problem, which is oriented on the implementation of the EYE-reasoner as it was in December 2013. We start with auxiliary definitions:
%\restdesc descriptions are expressed in the Notation3 (\nthree) rule language~\cite{N3Logic,Notation3}.
%We will introduce the \nthree language and its logic,
%focusing on the aspects relevant to our purposes.
After having discussed the characteristics of implicit quantification in Notation3 in the last section, we now formalize our observations. 
Where possible, we will follow the team submission \cite{Notation3} as this is the most official source indicating how the language is meant to be understood.
%As the dominant reference 
%we will use the team submission as this is the most official source available, which specifies how the logic is meant to be interpreted.
%Where both reasoners differ
%we are following cwm as this reasoner seems to be the closest to the actual specification.

To enable us to distinguish between variables occurring directly in a formula and variables only occurring in formula expressions which are dependent of a formula---as 
it is necessary to interpret for example formula (\ref{eq1})---we give the following definition: 




%%%%%%%%%%%%%%%%%%%%%%%%%%%%%%%%%%%%%%%%%%%%%%%%%%%%%%%%%%%%%%%%%%%%%%%%%%%%%%%%%%%%%%%%%%%%%%%%%%%%%%%%%%%%%%%%%%%%%%%%%%%%%%%%%%%%%%%%%
%Semantics
%%%%%%%%%%%%%%%%%%%%%%%%%%%%%%%%%%%%%%%%%%%%%%%%%%%%%%%%%%%%%%%%%%%%%%%%%%%%%%%%%%%%%%%%%%%%%%%%%%%%%%%%%%%%%%%%%%%%%%%%%%%%%%%%%%%%%%%%%




\begin{definition}[Components of a formula]
Let $f\in F$ be a formula and $c: E \rightarrow 2^E$ a~function such that:
\[c(e)=\begin{cases}
  
  c(e_1)\cup\ldots\cup c(e_n) & \text{if }e=\underline{\texttt{(}}e_1 \ldots e_n\underline{\texttt{)}}\text{ is a list,}\\
  \{e\}  & \text{otherwise.}
\end{cases}\]



We define the set $\textit{comp}(f)\subset E$ of components of $f$ as follows:
 \begin{itemize}
  \item If $f$ is an atomic formula of the form $e_1~ e_2~ e_3.$, $\textit{comp}(f)=c(e_1)\cup c(e_2)\cup c(e_3)$.
  \item If $f$ is an implication of the form $t_{1} \verb!=>!~t_{2}.$, then $\textit{comp}(f)=\{t_1, t_2\}$.
  \item If $f$ is a conjunction of the form $f_1 f_2$, then $\textit{comp}(f)=\textit{comp}(f_1)\cup \textit{comp}(f_2)$.
 \end{itemize}
 Likewise, for $n\in \mathbb{N}_{>0}$, we define the components of level $n$ as:
 \begin{flalign*} 
  \textit{comp}^n(f):= &  
  \{e\in E|\exists f_1,\ldots, f_{n-1}\in F: 
   e\in \textit{comp}(f_1)\wedge  \underline{\texttt{\{}}f_1\underline{\texttt{\}}}\in \textit{comp}(f_2)\wedge \ldots\\& \wedge  \underline{\texttt{\{}}f_{n-1}\underline{\texttt{\}}}\in 
  \textit{comp}(f)\} 
\end{flalign*} 
\end{definition}


The definition allows us to distinguish between direct components and 
nested components. As an example take the following \nthree formula:
%\begin{Verbatim}[fontsize=\normalsize] 
\begin{equation}
\label{cit}		\verb!:John :says {:Kurt :knows :Albert.}.! \end{equation}
%\end{Verbatim}
Direct components are \verb!:John!, \verb!:says! and \verb!{:Kurt :knows :Albert.}! while \verb!:Kurt!,  \verb!:knows! and  \verb!:Albert! are nested components of
level two.

%The definition allows us to distinguish between direct components and 
%nested components. We clarify the deepness of a nesting:

For variables, we can now clarify the deepness of a nesting:



\begin{definition}[Nesting level]
Let $f\in F$ be an \nthree formula and $v\in V$ a variable. The nesting level $n_f(v)$ of $v$ in $f$ 
is defined as follows:
\[
n_f(v):= 
\begin{cases} 
\min\{n\in \mathbb{N}|v\in \textit{comp}^n(f)\} & \text{if } v\in \textit{comp}^n(f) \text{ for some } n.\\
0 & \text{otherwise.}    
         \end{cases}
\]
The nesting level of a formula is defined as: $n(f) = \max_{v\in V} (n_f (v))$.
\end{definition}


As an illustration of the definition, consider the following formula:
\[
 f=\verb! _:x :says {_:y :says {_:x :knows ?z.}.}.!
\]
Here we have $n_f(\verb!_:x!)=1$, $n_f(\verb!_:y!)=2$, $n_f(\verb!?z!)=3$ and $n_f(\verb!?v!)=0$, as the latter does not occur in the formula. The nesting level of that formula is 
$n(f)=3$, as the deepest nested variable \verb!?z! has nesting level 3.

%This definition seems to be a little bit counter-intuitive, especially because we count the nesting level starting from the top of the formula and not, as one might expect, recursively from the bottom.
%The reason for this lies in the way nested universal quantifiers are expected to be interpreted: coming form the top formula, we are always interested in the less nested occurrence of a variable and its 
%direct parent formula, the variables with the same name which depend on this formula and are  are nested deeper are not important for the 


%Variables occurring in \nthree formulas are always implicitly quantified (either universally or existentially). 
%For the interpretation it is crucial to clearly define the scope of each variable. 
%Translated to first order logic, the formula 
%\[
%\verb! {{?x :p :a.} => {?x :q :b.}.} => {{?x :r :c.} => {?x :s :d.}.}.! 
%\]
%should be understood as 
%\[
%(\forall x_1: p(x_1,a)\rightarrow q(x_1,b))\rightarrow (\forall x_2: r(x_2,c) \rightarrow s(x_2,d))
%\]
%and \emph{not} as, for example,
%\[\forall x: ((p(x,a)\rightarrow q(x,b))\rightarrow ( r(x,c) \rightarrow s(x,d)))\]
At first glance this definition might seem counter-intuitive as we count the nesting level starting from the top of the formula and not, as one might expect, from the bottom. This has to do with the
intended meaning of implicitly quantified universals. Only the parent formula of an expression containing a universal is important for its interpretation (see formula (\ref{eq2})). 
All subordinated formulas depending on that formula are handled equally regardless of the degree of their dependency (see formula (\ref{nest})).

The different treatment of universal and existential variables in the interpretation of formulas (\ref{nest}) and (\ref{eq2}) also makes it necessary to define two ways to apply a substitution: 
%one which only acts on the component level and one, 
%which replaces also all nested variable by their substitute:

%it indicates which is the lowest depth a variable occurs and not, as one might expect, the most nested occurrence.  


%\noindent
%To reach this goal, we need to take a closer look to the possible applications of substitutions:

\begin{definition}[Substitution]

Let $A$ be an \nthree alphabet %, $S\subset V$ a set of variables 
and $f\in F$ an \nthree formula over $A$. 
\begin{itemize}
 \item A \emph{substitution} is a finite set of pairs of expressions $\{v_1/e_1, \ldots, v_n/e_n\}$ where each $e_i$ is an expression and each $v_i$ 
 a variable such that $v_i\neq e_i$ and 
 $v_i \neq v_j$,
 if $i\neq j$. 
 %We call each mapping $\sigma: S\cap \text{comp}(f) \rightarrow E$
 %a \textit{component substitution} of $f$. 
 \item %The application $f\sigma$ of a component substitution $\sigma$ 
 For a formula $f$ and a substitution $\sigma=\{v_1/e_1, \ldots, v_n/e_n\}$, we obtain the \emph{component application} of $\sigma$ to $f$, $f\sigma^c$, by simultaneously replacing each $v_i$ 
 which occurs as a \emph{direct component} in $f$ by the corresponding expression $e_i$. 
 \item %The application $f\sigma$ of a component substitution $\sigma$ 
 For a formula $f$ and a substitution $\sigma=\{v_1/e_1, \ldots, v_n/e_n\}$, we obtain the \emph{total application} of $\sigma$ to $f$, $f\sigma^t$, by simultaneously replacing each $v_i$ 
 which occurs as a \emph{direct or nested component} in $f$ by the corresponding expression $e_i$. 
 \end{itemize}
%A substitution $\sigma: S \rightarrow E_g$ is called \textit{ground substitution}, a substitution  $\sigma: S\rightarrow V_E$ \textit{existential substitution}.
\end{definition}

As the definition states, component application of a substitution only changes the direct components of a formula. For a substitution $\sigma=\{\verb!?x!/ \verb!:Kurt!\}$ we obtain:
\begin{multline}
(\texttt{?x :says \{?x :knows :Albert.\}.})\sigma^c  =\nonumber \\ (\texttt{ :Kurt :says \{?x :knows :Albert.\}.})\nonumber
\end{multline}
A total application, in contrast, replaces each \emph{occurrence} of a variable in a formula. with the same example as above, by applying $\sigma$ as a total substitution, we get:
\begin{multline}
(\texttt{?x :says \{?x :knows :Albert.\}.})\sigma^t =\\ (\texttt{:Kurt :says \{:Kurt :knows :Albert.\}.})\nonumber
\end{multline}


%For the following definition, we count the components of each level of a formula $f$ from left to right according to their position in $f$.
This enables us to state how the different kinds of variables in a formula should be treated by an interpretation:




\begin{definition}[Replacements]\label{repl}
Let $f\in F$ be an \nthree formula over an alphabet $A$. Let $v\in V_E$ be an existential and $u\in V_U$ a universal variable. 
For each level $i\in \mathbb{N}$, $i>1$, of $f$, we count with the index $j$ the formula expressions $\underline{\texttt{\{}} f_{ij}\underline{\texttt{\}}}\in \text{comp}^{(i-1)}(f)\cap \text{FE}$ according to
their appearance in $f$ from left to right.


%For each expression $e\in E$ of $A$ let
%$\sigma_e:V\rightarrow E$ be the constant map with $\sigma_e(x)=e$ for all $x\in V$.  
\begin{enumerate}
% \item A \emph{replacement} of $v$ is list $(\sigma_i)_{i\in \mathbb{N}}$ of substitutions $\{x/e_i\}$ for $v$.


 \item 
 An \emph{existential replacement} $\rho_v:F\rightarrow F$ of $v$ is a function consisting of single 
 substitutions $ \{v/e_{ij}\}=\sigma_{ij}$ with $i,j\in \mathbb{N}$ %$(\sigma_{ij})_{i,j \in \mathbb{N}}$ 
 which can be applied by simultaneously performing the component applications $f_{ij}\sigma_{ij}^c$ of the substitutions $\sigma_{ij}$ to the formulas $f_{ij}$. 
 %\\
 %Start with $f=f_{11}$ and perform the following steps:
 % \begin{enumerate}
  
 % \item If $n_f(v)=0$, then $\rho_v(f)=f$.
 %   \item If $n_f(v)>0$ perform the component application of substitution $\sigma_{11}$ on $f$ resulting in $f\sigma_{11}^c$  and perform for each level 
    %perform on each 
   %level 
  % $i>1$ for each formula $f_{ij}$ with $\underline{\texttt{\{}} f_{ij} \underline{\texttt{\}}}\in \text{comp}^{i-1}(f)\cap \text{FE}$ 
   %a component application 
    %of the substitution $\sigma_{ij}$.   
  %\end{enumerate}
  %To emphasize the different substitutions of an existential replacement, we sometimes write it as the set of all its substitutions 
  %$\rho = \{\sigma_{e_f}, \sigma_{e_{f_1}}, \ldots \sigma_{e_{f_n}},\sigma_{e_{f_{1_1}}}\ldots \sigma_{e_{f_{1_m}}}\ldots \sigma_{e_{f_{n_1}}}\ldots \sigma_{e_{f_{n_o}}} $
 \item \label{ur} A \emph{universal replacement}  $\mu_u:F\rightarrow F$ of $u$ is a function consisting of 
 single substitutions $\{u/e_{ij}\}=\sigma_{ij}$ with $i,j \in \mathbb{N}$ %$(\sigma_{ij})_{i,j\in \mathbb{N}}$ 
 which can be applied as follows:\\
 Start with $f=f_{11}$ and perform one of the following steps:
 \begin{enumerate}
 
  %\item If $n_f(u)=0$, then $\mu_u(f)=f$.
  \item \label{a} If $n_{f_{ij}}(u)\leq 2$, perform a total application $ f_{ij}\sigma_{ij}^t$. %of $\sigma_{(i+1)j}$ to $f_{ij}$. %Apply the same componenet substitution to all subformulas  
  %$\underline{\texttt{\{}}f_i\underline{\texttt{\}}}\in \textit{comp}(f)$ and all 
  %to all $u$ with $u\in \textit{comp}^n(f)$ for some $n\in \mathbb{N}$.
  \item \label{b} If $n_{f_{ij}}(u)>2$ 
  repeat %the above mentioned 
  %step (a) or (b) 
  this process
  for each $f_{(i+1) k}$\\with  
  $\underline{\texttt{\{}} f_{(i+1)k} \underline{\texttt{\}}}\in \text{comp}(f_{ij})\cap \text{FE}$. % if $n_{f_{ij}}(u)\leq 2$. 
  %whether 
  %each level 
  %$i\in \mathbb{N}$ formula $f_{ij}$ with $\underline{\texttt{\{}} f_{ij} \underline{\texttt{\}}}\in \text{comp}^{i-1}(f)$ if $n_{f_{ij}}(u)\leq 2$ and perform, if this is the case, a total
  %application $f_{ij}\sigma_{ij}^t$ of $\sigma_{ij}$.
  %, apply this procedure for all $f_i$ with $\underline{\texttt{\{}}f_i\underline{\texttt{\}}}\in \textit{comp}(f)$.
 \end{enumerate}

\end{enumerate}
%For both kinds of replacement
%the subformula index $j$ of a formula $f_{ij}$ is counted from left to right depending on its appearance in $f$.\\
If all substitutions $\sigma_{ij}$ are ground substitutions, i.e. $\sigma_{ij}:\{x\}\rightarrow E_g$, we call the respective replacement \emph{ground}. %\\
%If $V_f$ is the set of variables occurring in $f$ we call all replacements which only consist of substitutions of the form 
%$\sigma_{ij} :\{v\}\rightarrow V_E\setminus V_f$, $v\in V_E$ an %\emph{variable renaming}. Analogously we call it 
%\emph{existential renaming}. 
%Analogously we call $\sigma_{ij} :\{u\}\rightarrow V_U\setminus V_f$, $u\in V_U$
%if the range is $V_E\setminus V_f$ and 
%a \emph{universal renaming}.
%in the case that it is $V_U\setminus V_f$.
 \end{definition}


Note that existential and universal replacements are uniquely defined by the set of their substitutions.\\
%Note, that both definitions can be understood as a set of substitutions which can be applied to 
%Both replacements allow different substitutions for the same variable. While in existential replacement a substitution is always only applied 
%to the \emph{components} of one level of a formula, the substituions used within universal replacement change each \emph{occurrence} of a variable in a (part-)formula. 
The definition states that only existential variables occurring as components in the same (sub-)formula should be treated equally. 
This is different for universal variables. 
To understand how nesting level two is of importance, we come back to our previous example, formula (\ref{eq2}):
\begin{small}\[
f=(\verb!{{?x :p :a.} => {?x :q :b.}.} => {{?x :r :c.} => {?x :s :d.}.}.!) 
\]
\end{small}
For this formula the indexing $f_{ij}$ explained at the beginning of the definition is as follows:\\

\noindent
 \begin{tabular}{ p{3cm}    p{3cm}   p{3cm}  p{3cm} }
%\hline
\multicolumn{4}{l}{ $f_{11}=$}\\
\multicolumn{4}{l}{ $(\texttt{\{\{?x :p :a.\} => \{?x :q :b.\}.\} => \{\{?x :r :c.\} => \{?x :s :d.\}.\}.})$}\\
&&&\\
%\hline
\multicolumn{2}{l}{$f_{21} = $} & \multicolumn{2}{l}{$f_{22} = $}\\
\multicolumn{2}{l}{$(\texttt{\{?x :p :a.\} => \{?x :q :b.\}.})$} & \multicolumn{2}{l}{$(\texttt{\{?x :r :c.\} => \{?x :s :d.\}.})$}\\
&&&\\
  $f_{31}= $& $f_{32}=$& $f_{33}= $& $f_{34}= $\\
    $(\verb!?x :p :a.!)$& $(\verb!?x :q :b.!)$& $( \verb!?x :r :c.!)$& $(\verb!?x :s :d.!)$\\
&&&\\
 \end{tabular}



%$f_{11}=(\texttt{\{\{?x :p :a.\} => \{?x :q :b.\}.\} => \{\{?x :r :c.\} => \{?x :s :d.\}.\}.})$

%{$f_{22} = \texttt{\{?x :r :c.\} => \{?x :s :d.\}.\}.$}

%$f_{11} = f$

%$f_{21} = \verb!{?x :p :a.} => {?x :q :b.}.!$, $f_{22} = \verb!{?x :r :c.} => {?x :s :d.}.!$

%$f_{31}= \verb!?x :p :a.!$, $f_{32}=\verb!?x :q :b.!$, $f_{33}= \verb!?x :r :c.!$, $f_{34}= \verb!?x :s :d.!$.

\noindent Suppose we want to apply a universal replacement $\mu= (\sigma_{ij})_{i,j \in \mathbb{N}}$ for \verb!?x!:\\
As $n_f(\verb!?x!)=3>2$, i.e. \ref{b} is fulfilled, we have to consider the subformulas $f_{21}$ and $f_{22}$ separately.\\
%\[f_{21}=(\verb!{?x :p :a.} => {?x :q :b.}.!)\] \hspace{5cm} and \[f_{22}=(\verb!{{?x :r :c.} => {?x :s :d}.}.!)\] separately.
For $f_{21}$ we get $n_{f_{21}}(\verb!?x!)=2$, which means that \ref{a} is fulfilled. We apply the substitution $\sigma_{21}= \{\verb!?x!/e_{21}\}$ totally on $f_{21}$.\\
We do the same for $f_{22}$. As $n_{f_{22}}(\verb!?x!)=2$, thus \ref{a}, we totally apply the substitution $\sigma_{22}= \{\verb!?x!/e_{22}\}$.
As a result we get:
\begin{small}
\[
\mu(f)=(\verb!{{!e_{21}\verb! :p :a.} => {!e_{21}\verb! :q :b.}.} => {{!e_{22} \verb! :r :c.} => {! e_{22}\verb! :s :d.}.}.!) 
\]
\end{small}
%Both substitutions are applied via total applications such that, 
%if the respective variable occurred in sub-formulas, it would be handled the same way as it is as a~component of 
%$f_{21}$ and $f_{22}$. Note that $e_{21}$ and $e_{22}$ can be different. 
By the total application of the substitutions every nested occurrence of the variable is replaced, we do not have to consider the formulas $f_{3j}$ separately.\\ 
Note that $e_{21}$ and $e_{22}$ can be different. 
Our definition only ensures, that, 
if the same variable occurs in direct sibling formulas, it is treated in the same way, just as in interpretation (\ref{eq2}a).



\subsection{Interpretations and models}\label{semn3}
In this section we are going to embed our concept for evaluation of implicit quantified variables into a definition for the semantics of Notation3.

We start be defining an interpretation function:

%We embed our concepts in the definition of interpretation and semantics:

\begin{definition}[Interpretation]
An interpretation $\mathfrak{I}$ of
an alphabet $A$ consists of:
\begin{enumerate}
\item A set $\mathcal{D}$ called the domain of $\mathfrak{I}$.
\item A function $\mathfrak{a}: 
E_g \rightarrow \mathcal{D}$ called the object function.
\item A function $\mathfrak{p}:
\mathcal{D} \rightarrow 2^{\mathcal{D} \times \mathcal{D}}$ called the predicate function.
\end{enumerate}
\end{definition}

Note that in contrast to the classical definition of \rdf-semantics \cite{rdf} our domain does not distinguish between properties (IP) 
and resources (IR). 
The definitions are nevertheless compatible, as we assume $\mathfrak{p}(p)=\emptyset\in 2^{\mathcal{D} \times \mathcal{D}}$
for all resources p which are not properties (i.e. $p \in \text{IR}\setminus \text{IP}$ in the \rdf-sense). 
By extending given \rdf ground interpretation functions to Notation 3 interpretation functions, the meaning of all valid \rdf triples can be kept in Notation3 Logic. %\\
%Note that 
The main necessary extension would be a function which assigns domain values to ground formula expressions. 
This manner of handling formula expressions makes the expressiveness of such quoting of formulas as in example (\ref{cit}) very similar to \rdf reification \cite{rdf}. 
%Evaluation of concepts within a formula expression can be made via rules. 
%By doing so, we don't allow any extra-evaluation of the ground formulas (as for example 
%By handling ground formulas in that way, we don't allow 
%To use the \rdf ground interpretation functions here 
%By only adding one extra interpretation-function which assigns domain values to formula expressions, \rdf interpretation functions could be used here.  
%Thereby any \rdf-interpretation is also expressible as a Notation3 interpretation.
 %So, the interpretation functions 
%Therefore every \rdf triple can also be interpreted in Notation3.
%Notation3 is more tolerant than \rdf in the sense, that it accepts every kind of expression in all positions (subject, predicate or object).\\ 
%Especially in the object position this makes sense, as it enables us to make statements about formulas as in example (\ref{cit}). We consider 
%the oportunity to use a formula expression in the predicate position as rather theoretical, but we wouln't exclude that there may be cases where 
%this could be a useful feature.\\
%REIFICATION


%This flexibility might raises the question, whether it is really necessary to allow a
%If necessary, the predicate position can easily be restricted.




The following definition combines the replacements of variables and the ground interpretation functions:






\begin{definition}[\label{sem_n3}Semantics of \nthree]
Let $\mathfrak{I}=(\mathcal{D},\mathfrak{a,p})$ be an interpretation of $A$.
Let  $f$ be a formula over $A$ which contains at least one variable, $f_1$ and $f_2$ be ground formulas and $c_1,c_2,p \in E$ be ground expressions.
Then the following holds:
\begin{enumerate}
 \item\label{quant} $\mathfrak{I}\models f$ iff for all universal ground replacements $\mu_1,\ldots,\mu_n$ of the universal variables contained in $f$,  there exist some existential ground replacements
 $\rho_1,\ldots,\rho_m$ of the existential variables in $f$, such that 
 \[\mathfrak{I}\models \rho_1\circ\ldots\circ \rho_m \circ \mu_1\circ\ldots\circ\mu_n(f)\]

 \item $\mathfrak{I} \models c_1\, p\, c_2$. iff $(\mathfrak{a}(c_1),\mathfrak{a}(c_2))\in\mathfrak{p}(\mathfrak{a}(p))$.
 \item \label{conj} $\mathfrak{I} \models f_1 f_2$ iff $\mathfrak{I}\models f_1$ and $\mathfrak{I} \models f_2$.
 \item \label{impl1} $\mathfrak{I} \models \verb!{! f_1 \verb!}! \verb!=>! \verb!{! f_2 \verb!}!$. iff $\mathfrak{I} \models f_2$ if $\mathfrak{I} \models f_1$. 
% \item $\mathfrak{I} \models \verb!{ }! \verb!=>! \verb!{! f_2 \verb!}!$. iff $\mathfrak{I} \models f_2$.
% \item  $\mathfrak{I} \models e \,\verb!=>!\, \verb!{ }!$. for all formula expressions $e$.
% \item \label{fal1} $\mathfrak{I} \models \verb!false! \,\verb!=>!\, e $. for all formula expressions $e$.
 
% \item \label {fal2}$\mathfrak{I} \models \verb!{! f_1 \verb!}! \,\verb!=>!\, \verb!false!$. iff $\mathfrak{I} \not\models f_1$.
\end{enumerate} 
\end{definition}

Number \ref{quant} of the definition respects the constraint explained at the beginning of section \ref{quantsec} and illustrated by example (\ref{both}). % that
%prescribed by the \nthree-website \cite{Notation32}, that in a formula the 
%the scope of all universal quantifications is outside the scope of all existentials. 
This constraint is also the reason, 
why we define two separate mappings which ``ground'' the formulas before further valuation is done.

We finally define a model:

\begin{definition}[Model]
Let $\Phi$ be a set of \nthree formulas. We call an interpretation $\mathfrak{I}=(\mathcal{D},\mathfrak{a,p})$ a \textit{model} of $\Phi$ iff $\mathfrak{I}\models f$ for every formula $f\in \Phi$.
\end{definition}
%
%As in first order logic, we can define the notion of logical implication:

%\begin{definition}[Logical implication]\label{log_impl}
%Let $\Phi$ be a set of \nthree formulas  and $\phi$ a formula over the same \nthree alphabet $A$. We say that $\Phi$ (logical) implies 
%$\phi$ ($\Phi \models \phi$) iff every
%model $\mathfrak{I}\models \Phi$ is also a model of $\phi$.
%\end{definition}


%\section{Findings}







%As a consequence, 
%the interpretation of an existential is slightly different than the of a exisitential quantified variable in first order logic, as it alsways requires the existence of a 
%witness 
%The interpretation of \texttt{false}
%as defined in \ref{fal1} and \ref{fal2} is not supported by every \nthree reasoner. %Nevertheless, we include it in our definition to clarify that not every formula posses a model. 
%Our approach follows the semantics of EYE~\cite{eye}.\\


%\section{Implicit Quantification in other Logics}



