
\subsection{Mappings from Representations to Core Logics}
The approach of translating a logical representation into a well defined core logic to explain its semantics has been inspired by the formal description of 
programming languages where this practice is quite 
common. In programming languages, normally the lambda calculus and its extensions are used as a core logic. The general idea is, for example, explained by Pierce~\cite{Pierce}
and has been implemented for several programming languages: Sulzmann et al~\cite{Sulzmann} define System $F_C$, an extension of the polymorphic lambda calculus System F, to provide 
a way to express even rather complicated constructs contained in Haskell and other functional languages, as for example generalised algebraic data types (GADTs) and associated types,
in terms of a well defined logic.
Next to the definition and discussion of the basic properties of that logic, the authors also show how to translate the above mentioned examples from a
source language  into System $F_C$.
Igarashi et al~\cite{Igarashi} follow a similar approach for the programming language Java. To have a logical representation of the core features of Java, they define 
Featherweight Java. This logic can be used to describe and prove essential properties of the programming language. They furthermore discuss how the formalism can be extended 
by adding generic types and methods.

In the Semantic Web context a similar approach to the one represented in this paper can be found for the definition of SPARQL's semantics: 
instead of defining the semantics directly 
on the language itself, expressions of SPARQL are first mapped to the SPARQL algebra %~\cite{cyganiak2005relational} 
for which an evaluation 
semantics is defined.\footnote{\url{https://www.w3.org/TR/2013/REC-sparql11-query-20130321/\#sparqlDefinition}}
Another similar, although slightly different, approach can be found for \rdf. De Bruijn et al~\cite{erdf} embed \rdf into F-Logic and first order logic. The difference to our
approach is that this embedding is not done to define the semantics of \rdf{} -- this is defined directly -- but to research its relation to other logics.
The authors show that \rdf can be represented in the two frameworks.
