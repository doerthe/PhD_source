\section{Conclusion}\label{concl}
We conclude this paper in two parts: First, we get back to the research questions which we raised in the introduction and 
summarize how they have been addressed in this paper. In a second part we give an outlook to future work and 
discuss open challenges for the formalisation of \nthreelogic.


\subsection{Review of the Research Questions}
We discuss each research question separately:
\begin{enumerate}
 \item[(i)] \emph{How can we formally express the difference between two interpretations of the same \nthree formula?}
 \end{enumerate}
 To express the interpretations of implicitly quantified variables in \nthreelogic, we defined a core logic (Section~\ref{core}). The syntax of this 
 core logic is very close to the syntax of \nthree and -- apart from some differences in the symbols -- only differs from it in the fact 
 that it only supports explicit 
 quantification. That way, it is easy to express at which position an interpretation sets the quantifier when being faced with implicit quantification
 and to compare different interpretations. 
 For the syntax we also provided a semantics which respects the characteristics 
 of \nthree{} -- like the fact that lists need to be treated as \emph{first-class citizens} -- and allows for further refinement of concepts which have
 not been treated in detail in this paper like the interpretation 
of cited formulas and the definition of \nthree's built-in functions.

To connect an \nthree formula to its core logic counterpart, we chose the concept of attribute grammars (Section~\ref{ag}) and can -- using this concept -- formally
define how an interpretation maps \nthree to core logic.
 \begin{enumerate}
 \item[(ii)] \emph{How do existing interpretations of \nthreelogic conceptually differ in their way of handling implicit quantification?}
% \item[(iii)] What are the consequences of this difference in practice?
% \begin{enumerate}
\end{enumerate}
To answer this research question we first explained that universal quantification is closely related to the concept of a \emph{parent formula} 
(Section~\ref{universals}) which is underspecified in the \wwwc team submission. 
We specified attribute grammars for two different interpretations (Section~\ref{n3}): one seeing the top formula as the \emph{parent formula} of all implicitly universally
quantified variables occurring in it regardless of their level of nesting -- this is the interpretation the reasoner EYE applies -- 
one for which this \emph{parent formula} of an implicitly universally quantified variable is not the direct formula either written in curly brackets \{\} 
or being the top formula
containing the variable
but the next higher formula fulfilling these conditions -- this approach is implemented in Cwm.
\begin{enumerate}
 \item[(iii)] \textit{How often does this conceptual difference lead to conflicting interpretations of formulas used in practical applications?}
 \end{enumerate}
To know whether the conceptual difference in the treatment of implicit universal quantification is not only of theoretical nature but does have impact 
on practical cases we implemented the previously defined attribute grammars in Haskell (Section~\ref{imp}). We applied the implementation to different datasets which were written for 
practical applications and discovered that in 31\% of
all our investigated files at least one construct which is understood differently by the different interpretations can be found (Sections~\ref{data} and \ref{results}).
 \begin{enumerate}
 \item[(iv)]\textit{Which kinds of constructs cause these conflicting interpretations in practice and how likely is it that a file containing 
 these constructs is actually 
 subject to the problem?}
% \item How likely is it that a file containing such constructs leads to conflicting interpretations?
% \end{enumerate}
\end{enumerate}
For the answer of this last question, we looked deeper into the datasets and identified three different categories of constructs which
caused conflicting interpretations, namely proofs, built-ins, and deeply nested formulas without the occurrence of proof-constructs or built-ins 
(Section~\ref{results}). For all these constructs we provided examples and discussed whether the differences in the interpretations are problematic which was especially
the case for nested constructs without built-ins which are not part of a proof. 
We then tested if, given one of these constructs is present in a file, how likely it is 
that this file is treated differently by the two different interpretations. While this was always the case if proofs are present, 
disagreements could also be found for a quarter of the cases containing built-ins and for 72\% of the files containing nested constructs of any kind.
As a last step, we divided the files showing differences in the interpretations into disjoint groups depending on the main reason  causing that difference.
We identified that 13\% of our disagreements were caused by simple nesting without built-ins or proof-constructs. These are the cases we consider as most dangerous 
since files are mostly manually written -- opposed to computer generated proofs -- by users which do not have a specific reasoner in mind.
\\

Especially these cases lead us to the conclusion that the problem of having different interpretations for deeply nested implicit universal quantification needs 
to be addressed by a formalised standard. 
In our opinion, this standard should be easy to understand for users and easy to implement which is why we favour to
the interpretation which understands the top formula as the parent for all universals. 
% But 
% also
% for anyone 
%Being easy to understand and implement, we favour 
% Of the possible solutions we favour the option to follow the the interpretation 
% understanding the top formula as the parent for all universals (Section~\ref{solution}).
% A first, but in our opinion very important, step towards such a standard is to
By providing a way to explicitly express 
the differences between existing interpretations as we did by defining the core logic and showing how attribute grammars can be used to map from
\nthree formulas to that logic, we provided tools which ease the discussion.
Having these tools at hand enables anyone involved to clearly define how he understands implicit universal quantification 
and to test where this interpretation differs
from others. We thereby set a step forward towards the standardisation of \nthree.
% and a mapping from \nthree formula t exemplified of Cwm and EYE in this paper. 
% The newly provided framework enables us and any user to identify the concrete cases where interpretations 
% differ in their understanding of a formula and, hopefully, 
% helps to come to an agreement between different implementers.
% Nested constructs occurring in proofs are often rather harmless. Many cases where built-in functions are involved lead 
% to different interpretations due
% to the fact that not all functions are supported by all reasoners such that the problem of differently defined implicit quantification adds 
% not that much extra damage. But we
% also found cases in almost all of our datasets where nested constructs were occurring without being used in a proof or together with built-in functions. This happened in 13\%
% of the files causing contradictory interpretations. Especially these cases need to be avoided. 

\subsection{Open Challenges and Future Directions} 
While this paper shows how the uncertainties about implicit quantification in \nthree can be clarified the are further challenges lying ahead  
which need to be solved in order to standardise \nthreelogic. 

Next to the clarification of the different ways to interpret implicit quantification in \nthree and a study whether the concrete meaning of existing \nthree files 
changes with a change of the 
interpretation applied, the expressivity of formulas using implicit quantification is an important factor when making a choice how this should be handled. 
Understanding the \emph{parent formula} as the top formula and not as a nested formula makes \nthree with only implicit quantification less expressive. 
Whether this limited expressivity is strong enough to support all the tasks \nthree is meant for and how it compares to other standards needs to be carefully studied 
in order to come to an agreement.

The expressivity of implicit quantification becomes less crucial if \nthree, as intended by the \wwwc team submission, also supports explicit quantification. 
Section~\ref{remarkExplicitQuantification} briefly discusses the problems and uncertainties in the current informal specification. The most important factors are
how the exact position of a quantifier should be taken into account (ie does it make sense that universal quantification always dominates existential quantification) 
and whether or not variables should be separated from constants and have for example a designated name space. These topics need to be discussed in the community. 
We plan to use our core logic to formalise the different options which then, hopefully, leads to an agreement.

Another important topic we excluded from this paper is the formalisation of built-in functions. The \wwwc team submission discusses different predicates which
are considered as part of \nthree like the list functions \texttt{rdf:first} and \texttt{rdf:last} which \nthree inherits from \rdf but also, for example, the predicate 
\texttt{log:equalTo} to state ground equality. We designed the core logic in such way that it can be easily extended to support these and other predicates 
and plan to provide this formalisation as future work.

The last, but probably most critical point, which needs to be tackled in order to standardise \nthree is a clear definition how cited formulas need to be handled. The 
need for such constructs but also the difficulties which come with this kind of standardisation can be observed in \rdf: \rdf reification is excluded 
from the definition of \rdf semantics \cite{RDFSemantics} and 
for the TriG syntax to express named graphs, there is a disagreement about its meaning \cite{TriGsemantics}. We plan to base our own
proposal for 
this part of \nthreelogic on KIF~\cite{kif} and ISO Common Logic~\cite{ICL}.

With the definition of an extendible core logic we provided a framework which can and will help us to clarify all properties of \nthreelogic and thereby
leverage this logic to become a standard for the Semantic Web.
