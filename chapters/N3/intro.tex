










In Section~\ref{n3examples} we already briefly introduced \nthreelogic by discussing several examples and their intended meaning. We furthermore explained 
that this \emph{intended meaning} 
is problematic: The semantics of \nthreelogic is only defined in an informal way by the \wwwc team submission~\cite{Notation3} and a journal paper~\cite{N3Logic}. 
This leaves room for interpretation and it actually leads to contradicting implementations. 
In this chapter, we take a closer look to this problem. % and the related \hyperlink{rq2}{Research Question 2}.
%In order to answer  and test \hyperlink{h2}{Hypothesis 2} we create several test cases .  
We first collect evidence for this problem by giving concrete example formulas which, when given to the reasoners Cwm~\cite{cwm} and EYE~\cite{eye}, lead to different results. The problems we encounter are mainly related to 
implicit quantification: \nthree allows the use of quantified variables whose universal or existential quantifier is not explicitly stated but implicitly assumed.
To formalise the difference in the interpretations of implicit quantification, we define a core logic for \nthree which is very close to the original but only allows explicit quantification. We then map the interpretations of Cwm and EYE to that core logic.
This mapping is then implemented in a program which we then use to test, whether the problem we encountered is only present in our constructed test cases or whether it can also be observed in files 
used for practical applications.
We discuss several solutions for the problem and then show how, once such a solution is selected, the semantics of \nthreelogic can be defined in alignment with \rdf.
We set our findings into the context of related work to then conclude the chapter.

%This chapter is partly based on the following papers:

%\basedpapers{arndt_ruleml_2015}.