









%When introducing \nthreelogic in Section~\ref{n3examples} we also
% In Section~\ref{n3examples}  we already briefly introduced \nthreelogic by discussing several examples and their intended meaning. We furthermore explained 
% that this \emph{intended meaning} 
% is problematic:
The semantics of \nthreelogic is only defined in an informal way by the \wwwc team submission~\cite{Notation3} and a journal paper~\cite{N3Logic}. 
This leaves room for interpretation and it actually leads to contradicting implementations. 
In this chapter, we take a closer look to this problem and its possible solutions. % and the related \hyperlink{rq2}{Research Question 2}.
%We first briefly discuss the semantics of \rdf, the logic \nthree extends, and have a look at possible challenges
%In order to answer  and test \hyperlink{h2}{Hypothesis 2} we create several test cases .  
We first collect evidence %for this problem 
by giving concrete example formulas which, when given to the reasoners Cwm~\cite{cwm} and EYE~\cite{eye}, lead to different results. 
The discrepancies we encounter are mainly related to 
implicit quantification: \nthree allows the use of quantified variables whose universal or existential quantifier is not explicitly stated but implicitly assumed.
% Having understood this difficulty To understand how such a concept can be formalised, we discuss other frameworks which support implicit quantification and their semantics. Here we especially 
% focus on \rdf, the logic which is extended by \nthree: blank node in \rdf are understood as implicitly existentially quantified.
The problem can thus be solved by providing a formalisation which clearly indicates how implicit quantification needs to be interpreted. To better understand how such a formalisation 
could look like we then discuss different formalisms supporting some kind of implicit quantification, in particular \rdf and other formats related to the Semantic Web.

% We encounter similar concepts in other logics and especially in 
% %Similar concepts can be found in other logics, in particular in 
% \rdf where blank nodes are understood as implicitly existentially quantified. 
% % To better understand how implicit quantification can be handled in a formal model a closer look to \rdf semantics and the conclude this section
% % by discussing implicit quantification in general.
% In order to find a way to formally describe the semantics of \nthree we take a closer look to \rdf{}'s semantics and implicit quantification 
% 


% To put our observations into context, we then discuss how the concept of implicit quantification is handled in related contexts. In particular we give an overview of 
% \rdf semantics and 
% To formalise the difference in the interpretations of implicit quantification, we define a core logic for \nthree which is very close to the original 
% but only allows explicit quantification. 
% One part of such implicitly quantified variables are also present in the logic \nthree is based on: \rdf understands blank nodes as existentially quantified variables.
% To better understand how the semantics of a language can deal with implicit quantification we next take a look into \rdf semantics
%This chapter is partly based on the following papers: 

%\basedpapers{arndt_ruleml_2015}.