\section{Conclusion}
In this chapter we investigated which impact the fact that we have different interpretations for implicit universal quantification on 
reasoning files used for practical applications.

In order to do so, we implemented the attribute grammar defined in the previous chapter. 
This implementations allows us to produce the concrete \nthree Core Logic equivalent of an \nthree formula following the different interpretations and 
used for testing: Whoever wants to provide \nthree rules in the Web, can test whether the rules he or she created are subject to ambiguity.  

We applied our implementation on different on the example use cases of EYE and on different sets of files which have been implemented in research projects 
to solve practical problems. We encountered that 31\% of our files were subject to contradictions. 
We further investigated the contexts in which we could observe these contradictions and identified one group of them as most harmful:
if universal variables occur in constructs which do not make use of built-in functions and which are not part of a proof they are most likely produced by users who 
are not aware of the different possibilities to interpret the formulas. 13\% of or formulas belonged to that group.

These investigations have shown that the the underspecified semantics of \nthree is an actual problem which needs to be addressed. We therefore ended the chapter 
by discussing possible solutions for the practitioner but also for the community. The latter needs to come to an agreement. 

With this and the previous two chapters we addressed research questions 1 and 2 which focus on the formal semantics of \nthree and on \nthree's relation to \rdf. In the following 
chapter, we take a more practical perspective and discuss the use cases \nthreelogic can solve, in particular: a Semantic nurse call system and data validation.