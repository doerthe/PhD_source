\chapter{Conclusions}\label{concl}\label{conclusion}
In this thesis we investigated  \nthreelogic and its potential to become the \emph{Unifying logic} for the Semantic Web. 
Many properties of \nthree make it a very strong candidate for this role: 
It is based on \rdf, the most established standard of the Web, and it extends this framework by rules and citations. This connection makes it possible to reason 
about the \rdf representation of \owl ontologies by using rules which support the different concepts forming part of that standard. 
%These rules sometimes rely on built-in functions which are still not officially included in \nthree as proposed by the \wwwc team submission~\cite{Notation3}.
\nthree furthermore supports querying. It is possible to define filter rules such that the reasoners only provide all valid consequences of these rules. \nthree supports the layer 
of proofs: \nthree reasoners provide proofs for their conclusions. These proofs are again written in \nthree which makes it easy to share and exchange them and 
to use them in further applications.

We also discussed the most important problem when it comes to \nthree:  As its semantics is not formally defined the implementations using this 
logic differ in their understanding of implicit quantification. We tackled this problem by introducing a format close to \nthree which only supports explicit 
quantification: \nthree Core Logic. For this Logic we provided the formal semantics and defined concrete mappings from \nthree syntax to that logic which followed 
the different interpretations. By doing that we made the differences between interpretations explicit. Our \nthree Core Logic makes it possible to discuss 
different ways to define \nthree's semantics which then -- hopefully -- leads to an agreement.

Below, we answer the research questions we raised in the beginning of this thesis in detail.


\section{Review of the research questions}
Since Research question 1 relies on the other, we postpone its answer till the end of this section and start with Research question 2:


\emph{``How do the implemented interpretations of \nthree following the \wwwc team submission~\cite{Notation3}
and the journal paper~\cite{N3Logic}
differ, and how can this difference be formally expressed?''} 

In order to answer the first part of the question we performed different tests with the reasoners Cwm and EYE (Chapter~\ref{problem}) and came to the conclusion 
that these two reasoners differ in their understanding of implicit universal quantification. We further analysed this problem by consulting the 
\wwwc team submission and could identify its source: The team submission states that universals are quantified on their \emph{parent level} 
but it is not further explained what this \emph{parent} is. For EYE it is the overall formula, for Cwm it is the second next formula in curly brackets
\texttt{\{ \}} surrounding the universal. For Cwm, we furthermore identified exceptions from this quantification on parent level: If a formula depends on another formula 
on which the universal is already quantified, this quantifier also counts for all descendants. 

Having analysed this problem we next focussed on the second part of the research question: To be able to express the difference between these two 
and possible other 
interpretations of \nthree we defined \nthree Core Logic (Chapter~\ref{semofn3}).
This logic supports the same constructs as \nthreelogic with the only exception of implicit quantification. Quantifiers in \nthree Core Logic need to be stated explicitly.
We defined an attribute grammar which maps formulas stated in \nthree to their \nthree Core Logic interpretations in Cwm and EYE. In the case of Cwm 
this grammar was rather complex which is an indication that reasoners following this interpretation are at least difficult to implement.
Using the attribute grammar we can assign the meanings according to EYE and to Cwm to any given \nthree formula. This enabled us to answer Research question 3:

\emph{``How big is the impact of having different interpretations for \nthree formulas in practice?''}

We implemented the attribute grammar defined before and applied our implementations on practical cases (Chapter~\ref{eva}). As test data sets we chose
the test set of the EYE reasoner -- mainly because of its variety -- and different 
rule files which were used in research projects. For all these files we generated the interpretations according to Cwm and EYE and compared the results. In 31\%
of the cases we encountered differences in the interpretations. This indicates that the problem we found needs to be addressed. We further analysed our findings and 
identified one group of files which were particularly problematic: If users use deeply nested universal variables outside the proof environment and 
not in connection with built-ins, it is very likely that they are not aware of the differences of the reasoners. 
13\% of the files causing differences belong to this group. We finished this section by proposing possible solutions. In our opinion, 
quantifying all universals on top level would be the solution which users understand best and which is the easiest to implement. In the context of the previous 
definitions we also tackled Research question 4:

\emph{``Can the the semantics of \nthreelogic be defined such that it is compatible to \rdf?''}

By defining the semantics of \nthree Core Logic (Chapter~\ref{semofn3}) and formalising two mappings from \nthree syntax to this logic we also gave two possible 
formalisations for the semantics of \nthree. These formalisations were mostly compatible with the semantics of \rdf. 
% To interpret predicates, the interpretation for \nthree Core Logic and RDF both map these directly to the one element of the set of properties which is a subset 
% of the domain of discourse. An additional interpretation function assigns the set of relations fulfilling the relation to that property.
The only bigger difference between the interpretation of \nthree Core Logic as we defined it and the interpretation of RDF
is that for the former we need to ground existentially quantified variables 
first before we can assign them to the resource they represent. This had to do with the fact that cited  \nthree formulas should not be referentially transparent:
Even if cited formulas refer to the same statement they should be handled differently if they contain different constants. This difference 
from RDF semantics is therefore unavoidable. The practical impact of this difference seems to be rather small but to be sure  further investigation is needed.

Having discussed the theory of \nthree, we next turned to its practical use. Here we investigated Research question 5:

\emph{ 
``In which aspects can Notation3 Logic cover and connect the building blocks \emph{Querying}, \emph{Ontologies/Taxonomies} and \emph{Rules} 
of the Semantic Web stack?''
}

In order to answer this research question we considered two use cases: a semantic nurse call 
system and a system to perform RDF-validation (Chapter~\ref{others}). Both use cases have been implemented previously by applying OWL DL reasoning and/or SPARQL querying.
We approached both use cases by applying rule-based reasoning instead. In our implementation we made use of \nthree, its reasoning mechanism and the different built-in 
predicates supported by the reasoner EYE. 
Our resulting systems were able to provide the same functionality as the original ones.
The performance of our nurse call system was faster in all cases.  
The RDF-validation system was faster for datasets below 100,000 triples. 
We can thus conclude: Equipped with the right set of built-in predicates, \nthree can solve the same practical problems as OWL DL reasoning and SPARQL querying with a comparable performance.
%but we also need to make an addition here:  

Having discussed the connection of \nthreelogic to the layers \emph{Ontology/Taxonomy} and \emph{Querying} we now continue elaborating on the connection
of \nthree to the layer of \emph{Proofs}. In this context we asked Research question 6:

\emph{``How can we verify that the proof steps defined by the SWAP vocabulary are correct?''}

In order to answer that research questions we first defined a set of formulas for which the reasoners EYE and Cwm both quantify all universal variables on top level (Chapter~\ref{proof}).
For these so-called \emph{simple formulas} we then defined the direct semantics. This semantics also covers the interpretation EYE assumes for all formulas.
Using that definition we then formally defined the proof calculus which is expressed by the SWAP vocabulary and proved its correctness.
After having achieved that result, we focussed on Research question 7:
% In order to prove the correctness of the steps included in the SWAP proof vocabulary we f gave the definition of \nthree's direct semantics . This definition 
% was following the assumption that all universal variables are quantified on top level. 


% In order to show that using \nthreelogic we can combine querying over \rdf graphs and reasoning over OWL ontologies in one single technology and
% solve the same use cases as \owl DL reasoning and SPARQL querying  with a comparable performance.

\textit{ 
``Are there applications for \nthree proofs which go beyond the establishment of \emph{Trust}?''
}

As a use case for \nthree proofs we studied automatic composition and execution of hypermedia APIs (Chapter~\ref{restdesc}). We presented a way 
to describe the operations such APIs can perform by means of existential rules which then could be combined in proof for a desired goal. 
As all rules contributing to that goal also appear in such a proof, we can understand that proof as a plan, execute and -- if needed -- update it.
We defined an algorithm to exactly do that and we showed that if the proof generation itself terminates, then our algorithm does as well.
We discussed the limits of our approach: by using \nthree which is based on FOL we cannot express change, we furthermore cannot provide different options to the user.
We concluded that use cases which do not require these two properties can be tackled by our method and presented one instance of such a use case: the localisation of critical 
sensors in complex set-ups where context is relevant. By discussing these two use cases we showed that there are many applications which can benefit from \nthree proofs.

Having answered all these research questions we can now return to our main research question, Research question 1:

\textit{ 
``To what extend can Notation3 Logic fulfil the role of a \emph{Unifying Logic} for the Semantic Web?''
}

The research we conducted in this doctoral project showed that \nthreelogic is a promising candidate to become the \emph{Unifying Logic} of the Semantic Web:
We were able to propose two definitions for \nthree's semantics which were mostly compatible with RDF. It depends on the agreement of the community which 
of these definitions -- if any -- is chosen.
We furthermore showed that \nthree can be used to combine \emph{Querying}, reasoning about \emph{Ontologies/Taxonomies}, and rule-based inferencing. 
However, to properly support querying we  needed 
built-in functions. As \nthree is not standardised (yet) there is also no agreement on the built-in functions which are part of the logic. This  issue again 
needs to be decided by the community. 
\nthree  furthermore does not only support the layer of proof, by providing the to possibility to cite formulas it even makes such proofs part of the Semantic Web.
Proofs can be used for further reasoning, they can be exchanged and they can be used in all kinds of applications.
If the Semantic Web community comes to an agreement about the meaning of implicit universal quantification and the built-ins which form part of the logic, this logic 
can serve as a \emph{Unifying Logic} for the Semantic Web.
%(here I am currently considering to change the requirement in that direction). Proofs should not only be 
%producible, the produced proofs should also be part of the web and therefore again have a strong connection to \rdf.




\section{Open challenges and future directions}
Having answered the research questions in the previous section, we now take an outlook to the future and discuss the challenges which lie ahead. 

The most important challenge 
has already been mentioned in the previous section: We should work on the standardisation of \nthree.
The research presented in this thesis has shown that \nthree is worth that effort. 
Even though rule-based reasoning is very powerful, it has so far been widely neglected in the Semantic Web in favour of other frameworks like OWL DL which 
scarify user-friendliness in order to guarantee decidability. %Semantic Web reasoning should be performed by both, rule-based and OWL DL-based reasoners, 
Here, we do not say that one framework is better than the other -- there are use cases in which it makes sense to use an OWL DL reasoner instead of an 
rule-based reasoner -- we just say that both frameworks should co-exist and that in cases where reasoning needs to be combined with querying, \nthree could be 
the logic of choice. 

% the semantic Web lacks a rule language to be used in practical applications and if we 
% choose \nthree here, we most probably come
We made a first step towards the standardisation of \nthree and started a 
\wwwc community group\footnote{\url{https://www.w3.org/community/n3-dev/}}.
We hope to soon be able to come to agreements in the questions discussed\footnote{The discussions can be found on the group's git repository: \url{https://github.com/w3c/N3/issues/}} 
 to then provide a formalisation of the logic.
% 
% the open questions
% %open questions mentioned here in this thesis but also other ones 
% in order to come to an agreement. 
The most important points which need to be addressed are the meaning of implicit quantification, the agreement on the 
built-in functions which form part of the logic -- here it is in particular important whether or not we want to include scoped negation as failure -- and 
the meaning of cited formulas. This last topic has not been discussed in detail here in this thesis even though the semantic definition of \nthree Core Logic gives a hint
how we want to understand cited formulas. 
%  here to 
% solve this problem, but to really formalise \nthree the community needs to come to an agreement. We already started this group with the \wwwc community group (link)
% and hope to soon be able to formalise our agreements. The most important open issues are: how do we want to deal with cited formulas and what do the \emph{mean}?
% Which built-ins should the standard include? Do we for example introduce predicates for scoped negation as failure? Should \nthree be expressibel in plain \rdf and 
% is that even possible?


Apart from the standardisation of \nthreelogic we also see the need to extend the SWAP proof vocabulary: 
Reasoners should at least have the option to express all proof steps they perform when deriving new knowledge.
Even with a more complex vocabulary the implementers of reasoners can decide to omit proof steps they consider trivial.
But the calculus such descriptions represent should be complete for a big part of the logic, if possible, even for the entire logic.
% to allow reasoners to express all proof steps 
% they apply when deriving new knowledge. It can then still be decided by the implementers of the reasoners whether they omit trivial proof steps in the proofs 
% they display but the calculus defined should reflect the reasoning steps 
% We showed clear ways to define the semantics of \nthree, the decision which specification we want to use depends on the community (here I want to also talk about our W3C community group)

% We also clarified the relation of \nthree and \rdf: \nthree is compatible with \rdf and truely extends it. 
% 
% \nthree can be used for ontology reasoning and querying. But practice also needs to show whether we encounter practical cases where we get problems.

% \nthree supports the layer of proof and makes this layer part of the semantic web (here I am currently considering to change the requirement in that direction). Proofs should not only be 
% producible, the produced proofs should also be part of the web and therefore again have a strong connection to \rdf.
% 
% We also showed  practical applications for proofs.

% The condition we added to our answer for Research question 5, the fact that we need built-in predicates, again points to the problem we discussed earlier: as long as \nthree 
% is not standardised our answer is conditional: at the current point we do not know which built-ins will be included in a possible standard.
% 

To provide a last suggestion for further research we come back to the limitations of RESTdesc: we saw that there are cases where it makes sense for a logic to
have a notion of change. Even though the concept of time and invalid information is already part of the Semantic Web, we do not have any Semantic Web
Logic which can deal with that concept and use it -- for example -- for planning. Of course the solution of that problem requires us to leave our comfort zone of 
monotonic logics and try something new. But establishing this new direction as an addition to the different logical frameworks we already have we bring the Semantic 
Web one step further towards its real materialisation.

% Further investigate the ontology reasoning.
% 
% Go beyond RESTdesc (here I want to mention or at least give a hint that we had GPS4IC).


% General conclusion: \nthree logic is a very promising candidate to fulfil the rule as the unifying logic of the Semantic Web: It is syntactically and semantically 
% compatible with \rdf, it supports querying and reasoning over ontologies, it supports the layer of proof.

%But there are also problems to solve to get there: \nthree is not standardised and this is partly because the semantics is not formally defined.




%Rules should finally take the role they deserve in the Web!

%Mention SPIN and the fact that we can use it to query. Or better: everything which has an \rdf representation can serve as input for \nthree reasoning.

%grounding Herbrand