\chapter{Samenvatting}\label{dutch-summary}

\selectlanguage{dutch}

Om de visie van het Semantisch Web - een machine verstaanbare versie van ons huidige web - volledig
te realiseren, moeten we gebruik maken van logica. Alleen een zuivere logische taal met een goed
gedefinieerde semantiek stelt ons in staat om de kennis die we over de wereld hebben met
computers te delen. De architectuur van het Semantisch Web voorziet het gebruik van logica voor drie
verschillende soorten toepassingen: (1) logica moet query's mogelijk maken, (2) zij moet het mogelijk
maken om ontologieën en taxonomieën uit te drukken en te begrijpen, en (3) zij moet regelgebaseerde redenering ondersteunen. 
Hoewel deze toepassingen tot nu toe met
behulp van verschillende technologieën gerealiseerd zijn - voornamelijk SPARQL, RDFS/OWL en RIF
– is de visie om deze functionele eigenschappen te ondersteunen en deze technologieën te verbinden
door gebruik te maken van één enkel framework: de verenigende logica. Deze logica moet bovendien
 bewijzen ondersteunen: het moet mogelijk zijn om de uitgevoerde bewijzen uit te
drukken en uit te wisselen door de verenigende logica toe te passen.

In dit proefschrift onderzoeken wij dit idee van een verenigende logica en testen we de geschiktheid
van een mogelijke kandidaat om die rol te vervullen: Notation3 Logic (N3). N3 is een regelgebaseerde
logica die ongeveer tien jaar geleden is geïntroduceerd. Het vult het Resource Description
Framework (RDF) - het meest gebruikte formaat van het Semantisch Web – aan met universele
kwantificatie, met de mogelijkheid om formules aan te halen en regels uit te drukken, en met een
aantal verschillende ingebouwde predicaten. N3 maakt het mogelijk te redeneren over elke input die
syntactisch kan worden weergegeven in RDF, en de meeste N3-reasoners produceren al bewijzen om
hun afleidingen te toe te lichten. Dit maakt N3 een veelbelovende kandidaat om de verenigende
logica te worden, maar er zijn ook obstakels: de semantiek van N3 is alleen informeel gedefinieerd,
wat het niet alleen moeilijk maakt om de formele eigenschappen te bestuderen, het heeft ook
praktische gevolgen: vaak interpreteren verschillende reasoning engines één en dezelfde formule op
verschillende manieren.

Het proefschrift begint met een discussie van het Semantisch Web, de architectuur ervan en de rol
die de verenigende logica daarin speelt: we begrijpen de verenigende logica als een praktisch
hulpmiddel dat de algemene taken moet ondersteunen die verbonden zijn aan querying, redeneren
over ontologieën en taxonomieën, en regelgebaseerde gevolgtrekking. Dit hulpmiddel moet zelf deel
uitmaken van het Semantisch Web - bijvoorbeeld doordat het diens taal, RDF, begrijpt - en het moet
bewijzen ondersteunen. Deze aanpak is anders dan vele andere benaderingen die vooral
proberen om een logica te definiëren die beschrijvingslogica en regelgebaseerde redenering
combineert en nog steeds beslisbar blijft. Hoewel we het erover eens zijn dat dit een interessant
doel is, zijn we er niet van overtuigd dat deze benadering het semantisch web verder brengt, vooral
als we bedenken dat de combinatie van RDF en OWL DL, OWL Full, al niet beslisbar is. De semantiek
van de verenigende logica moet bovendien goed gedefinieerd zijn en compatibel zijn met RDF.

Na identificatie van deze vereisten: (1) duidelijke semantiek, (2) compatibiliteit met RDF, (3)
ondersteuning van querying, redeneren over ontologieën/taxonomieën en regelgebaseerde afleiding,
en (4) ondersteuning van bewijzen - richten we onze aandacht op N3 om te onderzoeken in hoeverre
deze logica aan die vereisten kan voldoen.

Aangezien de semantiek van N3 niet duidelijk is gedefinieerd en implementaties zelfs verschillen in
hun interpretatie van bepaalde formules, beginnen we met het uitvoeren van tests om het probleem beter
te begrijpen: we stellen vast dat de reasoners EYE en Cwm het oneens zijn in hun interpretatie van
impliciet gekwantificeerde variabelen. In N3 is het mogelijk om universeel of existentieel
gekwantificeerde variabelen te gebruiken waarvoor de kwantoren niet expliciet worden vermeld
maar impliciet worden verondersteld. De documenten die de N3Logic introduceren, stellen datimpliciet universeel gekwantificeerde variabelen zijn gekwantificeerd op basis van hun
bovenliggende formule. Maar omdat deze term niet verder wordt uitgelegd, begrijpen Cwm en EYE
hem op verschillende manieren.

Om dit verschil en de gevolgen ervan beter te begrijpen, definiëren we vervolgens een logica die
vergelijkbaar is met N3, maar die alleen expliciete kwantificering ondersteunt: N3 Core Logic. We
bieden een modeltheorie voor N3 Core Logic aan die grotendeels compatibel is met RDF. Vervolgens
definiëren we een attribuutgrammatica die formules in N3-syntaxis toewijst aan hun twee N3 Core
Logic-interpretaties, de ene volgens Cwm en de andere volgens EYE. Daardoor hebben we twee
mogelijke semantieken voor N3 gedefinieerd. Bovendien hebben we een mechanisme ontwikkeld om
voor elke N3-formule te beslissen of het al dan niet leidt tot onenigheid tussen de reasoners.

Vervolgens hebben we de attributengrammatica in Haskell geïmplementeerd. 
Om de praktische impact van het probleem dat we hebben uitgemaakt te kunnen
meten, hebben we onze implementatie toegepast op verschillende N3-bestanden die zijn gebruikt in
sectorgerelateerde onderzoeksprojecten. Voor 31\% van deze bestanden konden we discrepanties
uitmaken tussen de N3 Core Logic-vertalingen. Vervolgens hebben we deze discrepanties verder
geanalyseerd en drie verschillende soorten bronnen geïdentificeerd: bewijzen, built-ins en geneste
formules die geen bewijs-predicaten of ingebouwde functies bevatten. Omdat bewijzen meestal door
de computer gegenereerd en ingebouwd kunnen worden, zijn wij van mening dat de laatste groep
constructies, geneste formules die geen built-ins of proof-predikaten bevatten, het gevaarlijkst is:
gebruikers die dergelijke regels schrijven, verwachten normaal gesproken interoperabiliteit. 13\% van
onze kritieke bestanden vallen onder die groep. We komen daarom tot de conclusie dat dit probleem
door de community moet worden opgelost. We bespreken de verschillende opties, waarvan we er de
voorkeur aan geven om gewoon de interpretatie van EYE te volgen, omdat het gemakkelijker te
implementeren is en - althans wat ons betreft - gemakkelijker te begrijpen is.

Nadat we de semantiek van N3 en zijn relatie tot RDF hebben besproken, concentreren we ons
daarna op de taken die kunnen worden uitgevoerd met behulp van N3. Om te weten of N3 dezelfde
praktische problemen als OWL DL-reasoning en SPARQL-query's met een vergelijkbare prestatie kan
oplossen, hebben we twee use-cases overwogen: een semantisch oproepsysteem voor
verpleegkundigen en een systeem voor het uitvoeren van RDF-validatie. Beide use-cases zijn eerder
geïmplementeerd door OWL DL-reasoning en/of SPARQL-query's toe te passen. We hebben beide
use-cases benaderd door in plaats daarvan rule-based reasoning toe te passen. De resulterende
systemen waren in staat om dezelfde functionaliteit te bieden als de originele. De prestaties van ons
verpleegoproepsysteem waren in alle gevallen sneller. Het RDF-validatiesysteem was sneller voor
datasets van minder dan 100.000 triples. We kunnen dus concluderen dat de implementaties
vergelijkbare resultaten opleveren.

Na het onderzoek naar het vermogen van N3 om ontologie-reasoning en -query’s te ondersteunen,
gaan we over op het volgende vereiste: bewijzen. Omdat bewijzen in de visie van het Semantisch
Web uitwisselbaar dienen te zijn tussen verschillende partijen op het web, mogen ze geen formules
bevatten waarvan de betekenis dubbelzinnig is. We introduceren daarom het begrip simpele
formules, dat zijn formules waarin de universele variabelen gelijk worden geïnterpreteerd door de
reasoners Cwm en EYE. Voor dergelijke formules definiëren we de directe semantiek. Vervolgens
formuleren we formeel de verschillende bewijsstappen die kunnen worden uitgedrukt met behulp
van het N3-bewijsvocabulaire en we bewijzen hun juistheid.

Naast die formele beschouwing van bewijzen, bestuderen we ook hoe N3-bewijzen in de praktijk
kunnen worden gebruikt. We beschouwen hier een use-case die verder gaat dan het eenvoudig
vaststellen van vertrouwen: in de context van automatische API-compositie gebruiken we bewijzen
als plannen. Als we mogelijke operaties beschrijven die kunnen worden uitgevoerd door gebruik van
een hypermedia-API aan te roepen door middel van regels die duidelijk aangeven onder welke
omstandigheden de bewerking kan worden uitgevoerd en welke situatie dat zal produceren, kunnen
deze regels worden gecombineerd in een bewijs van een gewenst doel. Het bewijs kan rekening
houden met kennis van de context, en elke bewerking die bijdraagt aan het doel wordt vermeld in
het bewijs en kan vervolgens worden uitgevoerd. Als de context verandert, kan het bewijs eenvoudig
worden aangepast. We definiëren het concrete formaat om Web API's, RESTdesc en een algoritme te
beschrijven om dergelijke plannen te gebruiken en bij te werken. We leveren een bewijs voor 
het stoppen van ons algoritme. We bespreken bovendien de limitaties en breiden het idee uit om
bewijzen als plannen voor een andere use-case te gebruiken.

Concluderend laat het onderzoek voor dit proefschrift zien dat N3 Logic inderdaad een
veelbelovende kandidaat is om de verenigende logica van het Semantisch Web te worden: N3 is een
uitbreiding van RDF, het ondersteunt redeneren over ontologieën/taxonomieën, querying en
regelgebaseerd afleiden en het ondersteunt de bewijslaag: N3-reasoners leveren niet alleen
bewijzen, N3 is zelfs expressief genoeg om de reasoner in staat te stellen om dergelijke bewijzen in
de logica zelf weer te geven. Het onderzoek heeft aangetoond dat het mogelijk is om de semantiek
van N3 zodanig te formaliseren dat het compatibel is met RDF. Voor een gestandaardiseerde
formalisering moet de semantic web community overeenstemming bereiken over de interpretatie
van impliciete universele kwantificatie. We hopen dat dit proefschrift een waardevolle stap is naar
deze overeenkomst.



\selectlanguage{english}

% vim: spell spelllang=nl syntax=tex tw=140 
