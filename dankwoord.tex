\cleardoublepage

\normalsize

\chapter{Preface}
\setlength{\epigraphrule}{0pt}
\setlength{\epigraphwidth}{0.75\textwidth}
%\epigraph{\textit{Er ist ein Mathematiker und also hartnäckig.}}{Johann Wolfgang von Goethe}
\epigraph{\textit{Mit Mathematikern ist kein heiteres Verhältnis zu gewinnen.}}{Johann Wolfgang von Goethe}

%Describe the ``culture shock'' and explain why you chose the quote above. 
Before I started this journey, I always thought that computer scientists and mathematicians are somehow similar: the former apply what the latter studied before in theory.
Now, after I did my research, participated in different projects, read many  papers  and talked to different people at conferences, I can tell: I was wrong! In mathematics many things 
are only done on a theoretical level and the validity 
of the results needs to be formally proven to be sure that they are correct. As a consequence, mathematicians focus a lot on details. They 
need to be able to exclude all possible problems. In (applied) computer science on the other hand, new ideas can be implemented.
The fact that the resulting program works properly is then a first indication that the initial idea was not too bad.
For many people this first indication is enough. They want to deliver working tools as fast as possible and therefore accept the risk
that problems can occur later. 

Being a mathematician who stepped over to computer science in order to learn more about this fascinating world, %this ``cultural difference''  was often a problem.
% 
% And even if both groups want to find the golden middle between these two approaches they always fall back to the way they were educated and this
% education 
% I think the best way to do things actually lies between these two extremes: details are important, but we can also get lost in them. Implementations are important as 
% well but they can be more sustainable if we take some time to really think about the theory behind.
%. %As this implementation step gives 
%For many people in computer science this implementation is enough and they do not want to be further concerned with the theory.
%and even more: for many people the implementation is the only thing what matters. They see it as the best strategy to tackle problems just when they occur.
%Mathematicians as ``what can go wrong'' while computer scientists ask ``what can go right''. To get optimal results, we should combine both questions and 
%
I often struggled with this \emph{cultural difference} (as many of the persons mentioned below can confirm).
It was not always easy to accept that some people consider theory a burden, it was even harder to understand that they do not enjoy discussing details even if
these details directly impact their implementations, but the hardest thing for me was that even by showing them the beauty of logic I could not change their mind. %do not want to discuss the details of %prefer to only be confronted with it if problems occur. %some people start implementing before having carefully considered all possible problems their algorithm might have and it was 
%even more difficult to accept that there are people who do not even see the importance of a proper formalisation 
On the other hand I also learned a lot by facing this new point of view: %But I also learned a lot:
I learned to always take practical aspects into account if it comes to computer programs, and I learned that sometimes (but just sometimes)
it is good to start implementing before you think a problem through, just to better understand it.
%In general, I think it is a very good experience to

Having the hope that everyone 
who  did not enjoy discussing  the scoping of variables as much as I do or who sometimes did not want to solve possible problems before even having written a single line of code
% or who 
% thinks that I am too negative when I always try to search for possible problems 
%had problems with my focus on logical details and my constant search for possible problems 
%in the past few years 
also sees \emph{some} positive aspects in having worked with me, 
I chose the above quote 
to let you know that you are not alone: There are many people who consider  mathematicians \emph{difficult} people among them the famous Johann Wolfgang von Goethe. 
% 
% \todo{Maybe add that I like to travel between because of cultural differences which always intrigued me. I also enjoyed this journey to another profession for the same reason: 
% the different spirit and research culture makes you feel that you just entered a ``new country of research''. An while I struggeld a lot in ``Computer science land'' I also learned to speak
% the language of its citizens and adopt to their habits. But I also need to add that the}
Taking up that quote from above, I thank everyone who tried to get ``ein heiteres Verh\"altnis'' with me despite the difficulties mentioned above.


In particular I thank Miel for guiding me through the writing process of this thesis, Femke for listening and helping with all kinds of problems, and the Knowledge on Web-Scale (KNoWS) team for 
sharing their knowledge with me.
As a team is always just as good as its members, I want to thank everyone individually (taking the risk of forgetting someone, sorry if this is the case), especially I thank my former colleagues Tom D.N., 
Sam, Hajar, Cristian, Laurens, Gayane, Dieter D.P., Dieter D.W.
and all current members of the team, in particular Joachim,
Pieter C.,
Pieter H.,
Ben, Sven, Gerald, Martin, Julian, Harm, Brecht, Anastasia, and Ruben T.
I also thank  all current and former members of IBCN, I had the pleasure to work with, in particular Pieter B., Mathias, Stijn and Alexander.
%Miel, Jos, Ruben, Tom, Pieter Pauwels, Femke, Agfa etc.

The research conducted in this thesis has mostly been done in the context of projects involving industry partners. I thank all these partners. In this context I want to give a special mention to Agfa healthcare.
I thank everyone at Agfa I worked with.
Visiting you was always the highlight of my week. In particular I thank Giovanni for always sharing the programmer's perspective with me,
I thank Hong  for his challenging questions, Boris for helping me with decisions, and Els for always trying to keep us focussed -- I know that this is not an easy 
task given that Jos and I really enjoy discussing logical details. 
A very special thanks goes to Jos for exactly those discussions. I think there is no one with whom talking about logics is more enjoyable.
It was you who reminded me how exciting research can be.

I furthermore thank all members of the jury for agreeing to review my thesis. 
Knowingly or unknowingly, you influenced the work presented in this thesis in some way:
You pointed me to papers, asked the right questions or introduced me to new fields.  
% for example by short conversations, by pointing me to papers, or simply by teaching me 
% that there are other ways to define semantics than defining a model theory. 
A very special thanks goes to Harold for making me feel welcome in the RuleML community, especially at my very first RuleML conference (2015 in Berlin).  
%I especially appreciate how you 
 
I also want to thank my supervisors.
I thank Tom S. for encouraging me to aim for \emph{good} papers instead of papers \emph{good enough to be accepted}.
I thank Erik for always backing me up. I thank Ruben for sharing his enthusiasm with me and for pointing me to my research topic:
I am very happy that you immediately anticipated my passion for \nthreelogic.  

I also thank all my friends who supported me during the writing of my thesis or even before. 
A special thanks goes to Stefanie, my former colleague, who encouraged (or forced?) me to start this journey, without her I would never have applied for my PhD position.
I thank my family, my parents, my brothers and, especially, my sister Sabine who helped me with the Dutch translation of my summary (it was a very last minute help, so please do not judge her language skills based on that summary).

Last, but not least, I thank Nicolas who gave me a lot practical and moral support during this whole process. He agreed to move to Belgium with me,  he accepted that I spent many weekends with \nthree instead of him 
(I agree that this is a questionable choice)
and he counselled me whenever I came home complaining about these \emph{``weird computer scientists''} who seemed not to care about the correctness of their implementations. %made sure that regardless of my stress level, I still got enough food and sleep.
During the writing process he furthermore took care of our daughter Paula and thereby gave me the time to finish this book. 
Without you this thesis would not have been possible.
Thank you!

\vspace*{\fill}

\begin{flushright}
D\"orthe Arndt \\
Gent, \today
\end{flushright}

\vspace*{\fill}



% vim: spell spelllang=nl syntax=tex tw=140 
